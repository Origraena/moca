% !TEX encoding = UTF-8 Unicode
% !TEX root = ../Rapport/rapport.tex

\section{Programmation dynamique}

\subsection{Sur le problème de la partition}
\subsubsection{Condition nécessaire sur la somme des poids des $n$ objets}

Les poids de chacun des objets étant des valeurs entières, pour qu'il existe deux sous-ensembles
distincts ayant le même poids, il faut que la somme des poids des objets soit paire.

\subsubsection{Récurrence}

Introduisons l'expression booléenne $T(i,j) $ : \emph{Etant donnés les $i$ premiers éléments de la
famille, il existe un sous-ensemble de ces $i$ éléments de poids $j$}. 

Dans un premier temps il paraît évident que si $j=0$, le sous-ensemble existe : il s'agit de
l'ensemble vide. \begin{equation}
\forall i \in \mathbb{[}0, n \mathbb{]}, \quad T(i,0) = \mbox{VRAI} 
\end{equation}

De plus, si $j$ est égal au poids de l'un des $i$ premiers éléments, alors l'existence du
sous-ensemble est avérée. Autrement dit, si l'on appelle $K_i$ l'ensemble formé par les poids des
objets $i$, on a : \begin{equation}
	\forall i \in \mathbb{[}0, n \mathbb{]}, \quad T(i,j \in K_i) = \mbox{VRAI}
\end{equation}

Troisièmement, l'ensemble des $i-1$ premiers objets est un sous ensemble de l'ensemble des $i$
premiers objets : $E_{i-1} \subset E_i$. On en déduit donc, que s'il existe un sous ensemble de
$E_{i-1}$ de poids $j$ tel que $T(i-1, j) = \mbox{VRAI}$, alors ce sous-ensemble est aussi un
sous-ensemble de $E_i$ tel que $T(i, j) = \mbox{VRAI}$ et ce par le principe d'inclusion.
\begin{equation}
	\forall i \in \mathbb{[}0, n \mathbb{]}, \quad \mbox{ si } T(i-1, j) = \mbox{VRAI}, \mbox{
	alors } T(i,j) = \mbox{VRAI}
\end{equation}

Enfin, considérons $S_j$ un sous-ensemble de $E_{i-1}$ de pois $j$ vérifiant donc $T(i-1, j) =
\mbox{VRAI}$, alors $S_j \cup a_i$ est un sous-ensemble de $E_i$ de poids $j+p(a_i)$ et donc
vérifiant $T(i,j+p(a_i)) = \mbox{VRAI}$. On en déduit donc :
\begin{equation}
	\forall i \in \mathbb{[}0, n \mathbb{]}, \quad \mbox{ si } T(i-1, j-p(a_i)) = \mbox{VRAI}, \mbox{
	alors } T(i,j) = \mbox{VRAI}
\end{equation}

En réunissant les équations ci dessus dans une seule expression booléenne, on obtient la formule de
la ligne $i$ en fonction de la ligne $i-1$ et $p(a_i)$ :
\begin{equation}
	T(i,j) = j == 0 \vee j == p(a_i) \vee T(i-1, j) \vee T(i-1, j-p(a_i))
\end{equation}

On peut alors donner un algorithme pour résoudre le problème de la partition : l'algorithme
\ref{solvepart}.

\begin{algorithm}
	\caption{solve-partition}
	\label{solvepart}
	\begin{algorithmic}[1]
		\STATE $p_{max} \leftarrow \sum_{i \in \mathbb{[}1, n \mathbb{]}} p(a_i)$
		\IF{$p_{max} \equiv 1 \pmod{2}$}
			\STATE Pas de solutions
		\ELSE
			\FOR {$j \in \mathbb{[}0,p_{max}\mathbb{]}$}
				\IF{$j = 0$}
					\STATE $T(0, j) \leftarrow \TRUE$
				\ELSE
					\STATE $T(0, j) \leftarrow \FALSE$
				\ENDIF
			\ENDFOR
			\FOR{$i \in \mathbb{[}1, n \mathbb{]}$}
				\FOR{$j \in \mathbb{[}0, p_{max}\mathbb{]}$}
					\STATE $T(i, j) \leftarrow j = 0 \vee j = p(a_i) \vee T(i-1, j) \vee T(i-1, j-p(a_i))$
				\ENDFOR
			\ENDFOR
			\IF{$T(n, p_{max} / 2) = \FALSE$}
				\STATE Pas de solutions
			\ELSE
				\STATE Construire le sous ensemble solution
			\ENDIF
		\ENDIF
	\end{algorithmic}
\end{algorithm}

Le sous ensemble solution $S_{sol}$ se construit à l'aide du principe suivant : pour une ligne $i>0$ et
un poids $j>0$ donnés, on a $a_i \in S_{sol}$ si $T(i-1, j) = \mbox{FAUX} \wedge T(i,j) =
\mbox{VRAI}$. Ceci n'est vrai que dans un sens et permet d'introduire l'implication suivante :
\begin{equation}
	\label{ineq}
	T(i-1, j) \oplus T(i, j) = \mbox{VRAI} \Rightarrow a_i \in S_{sol} 
\end{equation}

Ceci mérite quelques explications. Séparons les différents cas possibles et présentons les dans un
tableau : \\
\begin{center}
\begin{tabular}{|c|c|c|} \hline
	\backslashbox{$T(i-1, j)$}{$T(i,j)$} & VRAI & FAUX \\	\hline
	VRAI & Cas 1 & Cas 2 \\ \hline
	FAUX & Cas 3 & Cas 4 \\ \hline
\end{tabular}
\end{center}

Analysons les différents cas : \begin{enumerate}
	\item Dans ce cas, l'ensemble des $i-1$ premiers objets possède un sous-ensemble vérifiant les
		contraintes imposées. On peut donc en déduire qu'il existe un sous-ensemble ne contenant pas
		$a_i$ et vérifiant ces mêmes contraintes\footnote{Attention, celà ne veut pas dire qu'il
		n'existe pas de sous-ensembles vérifiant les contraintes et contenant $a_i$}. C'est cet ensemble
		qui nous intéresse
	\item Ce cas est un cas interdit par la formule de calcul de la ligne $i$. En effet, si $T(i-1, j)
		=$ VRAI, alors $T(i,j) =$ VRAI par construction.
	\item C'est ce cas qui est intéressant, puisqu'il nous indique que l'ajout de l'objet $a_i$ permet
		de trouver une solution au problème du sous-ensemble pour les valeurs données. On a donc $a_i$
		appartenant bel et bien à la solution.
	\item Ce cas indique jusque que la solution n'existe pas.
\end{enumerate}

De ces principes, on peut déduire l'algorithme \ref{recsol} de reconstruction de la solution.
\begin{algorithm}
	\caption{Construct-sol}
	\label{recsol}
	\begin{algorithmic}[1]
		\STATE $S \leftarrow \emptyset$
		\IF{$i = 0$ \OR $j=0$}
			\STATE Retourne $S$
		\ELSE
			\IF{$T(i-1,j) = \FALSE$}
				\STATE Retourne $a_i \cup$ Construct-sol($i-1$,$j-a_i$)
			\ELSE
				\STATE Retourne Construct-sol($i-1$, $j$)
			\ENDIF
		\ENDIF
	\end{algorithmic}
\end{algorithm}


\subsubsection{Complexité}


\subsubsection{Jeux d'essais}


\subsection{Le problème du sac à dos}
\subsubsection{Justification des formules}

\subsubsection{Exemple}

\subsubsection{Complexité}


\subsection{Le problème du voyageur de commerce}
\subsubsection{Exemple}

\subsubsection{Complexité}
