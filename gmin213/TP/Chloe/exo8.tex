% !TEX encoding = UTF-8 Unicode
% !TEX root = ../Rapport/rapport.tex

\section{Sur le produit matriciel}

\subsection{Nombre d'opérations nécessaires pour un produit}
Soit $M_i$ une $(p_i,p_{i+1})$-matrice. Soient $\pi _1$ et $\pi _2$ les produits matriciels à parenthésage symétrique suivants : 
$$\pi _1 = (M_1(M_2(M_3(M_4(...(M_{k-1}.M_k))))))$$
$$\pi _2 = ((((((M_1.M_2)M_3)M_4)...)M_{k-1})M_k)$$
Nous admettons que le produit d'une $(p,q)$-matrice et d'une $(q,r)$-matrice peut se faire à l'aide de $p.q.r$ opérations. Tous les produits matriciels envisagés par la suite sont définis car le nombre de colonnes de la première matrice est égal au nombre de lignes de la seconde.

Évaluons le nombre d'opérations nécessaires au produit $\pi _1$ grâce au tableau suivant :
\begin{center}
\resizebox{\textwidth}{!}{
$ \begin{array}{|c|c|c|c|c|c|c|} \hline
	\pi _l & \pi _{1,1} & … & \pi _{1,4} & … & \pi _{1,k-2} & \pi _{1,k-1} \\ \hline
	=  & M_1 \times & … & M_{4} \times  & … & M_{k-2} \times & (M_{k-1} \times M_k) \\
	Nb.\ d'operations & 1.2.(k+1) & … & 4.5.(k+1) & … & (k-2).(k-1).(k+1) & (k-1).k.(k+1) \\
	Taille\ de\ la\ matrice & (1,k+1) & … & (4,k+1) & … & (k-2,k+1) & (k-1,k+1) \\ \hline
 \end{array} $
 }
 \end{center}
 
 Nous pouvons alors déduire la formule donnant le nombre d'opérations nécessaires pour calculer le produit $\pi _1$ :
$$ \sum\limits_{i =1}^{i < k} {[i\times (i+1) \times (k+1)]} $$

De la même façon, évaluons le nombre d'opérations nécessaires au produit $\pi _2$ grâce au tableau suivant :
\begin{center}
$ \begin{array}{|c|c|c|c|c|c|c|} \hline
	\pi _l & \pi _{1,1} & \pi _{1,2} & \pi _{1,3} & … & \pi _{1,k-2} & \pi _{1,k-1} \\ \hline
	=  & M_1 \times M_2 & \times M_3 & \times M_{4}  & … & \times M_{k-1} & \times M_k \\
	Nb.\ d'operations & 1.2.3 & 1.3.4 & 1.4.5 & … & 1.(k-1).k & 1.k.(k+1) \\
	Taille\ de\ la\ matrice & (1,3) & (1,4) & (1,5) & … & (1,k) & (1,k+1) \\ \hline
 \end{array} $
 \end{center}
 
 Nous pouvons alors déduire la formule donnant le nombre d'opérations nécessaires pour calculer le produit $\pi _2$ :
$$ \sum\limits_{i =1}^{i < k} {[(i+1) \times (i+2)]} $$

Comparons maintenant le nombre d'opérations nécessaires aux produits $\pi_1$ et $\pi_2$ :

\begin{center}
$ \begin{array}{|c|C{4.5cm}|C{4.5cm}|C{4.5cm}|} \hline
	 & \sum\limits_{i =1}^{i < k} {[i\times (i+1) \times (k+1)]} & vs & \sum\limits_{i =1}^{i < k} {[(i+1) \times (i+2)]} \\ \hline
	k=1  & 4 & < & 6  \\
	k=2 & 24 & > & 18 \\
	k=3 & 80 & > & 38 \\ \hline
 \end{array} $
 \end{center}
 
Nous déduisons du tableau précédent (et du fait que la première expression comporte un terme de plus) que $\forall k \geq 2$ le nombre d'opérations nécessaires pour calculer le produit $\pi_1$ est supérieur au nombre d'opérations nécessaires pour calculer le produit $\pi_2$. Le nombre d'opérations nécessaires pour calculer un produit matriciel dépendant du parenthésage choisi.

 
 
\subsection{Nombre de parenthésages possibles d'un produit de $k$ matrices}


\subsection{Récurrence}


\subsection{Exemple}
