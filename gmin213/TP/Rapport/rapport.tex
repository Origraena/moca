% !TEX encoding = UTF-8 Unicode
\documentclass[a4paper]{report}

%%% Encodage, langues, guillemets, ... %%%
\usepackage[french]{babel}       
\usepackage[utf8]{inputenc}
\usepackage{setspace}
\usepackage{listings}
\usepackage{lscape}
%%% Inclusion images et feuilles pdf %%%
\usepackage{graphicx}
\usepackage{pdfpages}
\usepackage{wrapfig}
\usepackage{geometry}
%%% Inclusion url et liens %%%
\usepackage{hyperref}
\usepackage{url}
% Changement des puces de enumerate
\usepackage{enumerate}
%%% Symboles, Mathématique %%%
% Symbole euro
\usepackage{eurosym}
% Mathématiques
\usepackage{amsmath}
\usepackage{amssymb}
%%% Schémas
% Graphes, géométrie et autres
\usepackage{tikz}
\usetikzlibrary{trees}
%%% Algorithme
\usepackage{algorithm}
\usepackage{algorithmic}

\usepackage{slashbox}

\usepackage{array}
\newcolumntype{C}[1]{>{\centering\arraybackslash$}p{#1}<{$}}
\renewcommand\arraystretch{1.6}

\newcommand{\PLNE}[0]{ Programmation Linéaire en Nombres Entiers }
\newcommand{\card}{\mathrm{Card}}


\pagestyle{headings}
\thispagestyle{empty}
\geometry{a4paper,twoside,left=2.5cm,right=2.5cm,marginparwidth=1.2cm,marginparsep=3mm,top=2.5cm,bottom=2.5cm}
\begin{document}
\large
\setlength{\parskip}{5mm plus2mm minus2mm}
\lstset{language=C, showstringspaces=false, numbers=left, numberstyle=\tiny, tabsize=4}

 
 
{\setlength{\parindent}{0cm}
Chloé DESDOUITS \hfill M1 Informatique \\
Guillerme DUVILLIE \\
Swan ROCHER
}
\vfill
{\centering \Huge \bfseries TP de méthodes de résolution de problèmes NP-complets\par}
\vfill
2012 \hfill UM2

\setcounter{tocdepth}{1}
\tableofcontents
\thispagestyle{empty}
\pagenumbering{arabic}


\chapter{Partie théorique}
% !TEX root = ../../Rapport/rapport.tex
\section{Sur le problème de la couverture sommet minimale : trois approches différentes}

\subsection{Première approche : la programmation linéaire en nombres entiers}

\subsubsection{(a) Justification de l'utilisation de la Programmation Linéaire en Nombres Entiers}

On considère le problème de la couverture minimale sous la forme suivante : $$
\left \lbrace \begin{array}{l}
	\min z = \sum\limits_{j=1}^n x_j\\
	x_r + x_s \geq 1, \quad \forall \{v_r, v_s\} \in E \\
	x_j \in \{0,1\} \quad j = 1, \dot, n
\end{array} \right .
$$

La fonction objectif représente le nombre de sommets utilisés par la solution du problème. Le fait
de minimiser la fonction objectif permet d'assurer la couverture minimale. Chacune des clauses est
relative à une arête du graphe, et impose qu'au moins un des sommets adjacents à cette arête soit
dans la couverture.

On a donc bien un problème de Programmation Linéaire en Nombres Entiers permettant de résoudre le problème de la couverture minimale.

\subsubsection{(b) Justification des clauses}

Considérons le graphe donné par la figure \ref{triangle}. 

\begin{figure}
	\begin{center}
		\begin{tikzpicture}
			\tikzset{node/.style={circle, draw=black}};
			\node[node] (A) at (0,2) {$A$};
			\node[node] (B) at (-1, 0) {$B$};
			\node[node] (C) at (1, 0) {$C$};

			\draw (A) -- (B) -- (C) -- (A);
		\end{tikzpicture}
	\end{center}
	\label{triangle}
	\caption{Exemple}
\end{figure}

Sur ce graphe, le programme linéaire en nombres entiers est le suivant :$$
\left \{ \begin{array}{l}
		\min z = x_A + x_B + x_C\\
		x_A + x_B \geq 1 \\
		x_A + x_C \geq 1 \\
		x_B + x_C \geq 1 \\
		x_A, x_B, x_C \in \{0, 1\}
	\end{array} \right .
	$$
Il est très simple ici de comprendre pourquoi il est impossible de considérer le programme linéaire
suivant :$$
\left \{ \begin{array}{l}
		\min z = x_A + x_B + x_C\\
		x_A + x_B = 1 \\
		x_A + x_C = 1 \\
		x_B + x_C = 1 \\
		x_A, x_B, x_C \in \{0, 1\}
	\end{array} \right .
	$$

Ce programmme ne permet pas de résoudre la couverture minimale sur le graphe donné par la figure
\ref{triangle}. Quelque soit le sommet choisi dans un premier lieu pour appartenir à la couverture
minimale, il est impossible d'en choisir un second pour compléter cette dernière. Prenons un
exemple, nous forçons le sommet $A$ à appartenir à la couverture minimale (respectivement $B$ et
$C$). Ce choix force : $x_B = 0$ et $x_C = 0$ (respectivement, $x_A = 0$ et $x_C = 0$, et $x_A = 0$
et $x_B$ = 0). Il est donc impossible de respecter la clause $x_B + x_C = 1$, le problème (au vu de
sa modélisation) n'aurait donc pas de solution, or le graphe de la figure \ref{trisol} montre le
contraire.

\begin{figure}
	\begin{center}
		\begin{tikzpicture}
			\tikzset{node/.style={circle, draw=black}};
			\node[node,draw=red] (A) at (0,2) {$A$};
			\node[node,draw=red] (B) at (-1, 0) {$B$};
			\node[node] (C) at (1, 0) {$C$};

			\draw[green] (A) -- (B) -- (C) -- (A);
		\end{tikzpicture}
	\end{center}
	\label{trisol}
	\caption{Solution}
\end{figure}

\subsubsection{(c) Une borne inférieure des solutions optimales}

On cherche à montrer qu'une solution optimale du programme linéaire en nombres entiers est une borne
inférieure de toute solution optimale du programme relaxé. Raisonnons par l'absurde et considérons
une solution optimale du programme linéaire, notée $n^*$ telle qu'il existe $x^*$ solution optimale du problème
relaxé vérifiant $x^* < n^*$. Toute solution du programme linéaire est solution du programme
relaxé\footnote{Une solution appartenant à $\mathbb{N}$ appartient aussi à $\mathbb{R}$}. Ceci
implique : $n^*$ solution du programme relaxé, et donc $x^* < n^*$ impossible. On a donc : $n^* \leq
x^*$ ce qui est la définition d'une borne inférieure.

\subsubsection{(d) A propos de la relaxation de contrainte}

Pour démontrer que la relaxation des contraintes d'intégrité implique $x_r \geq \frac{1}{2}$ ou $x_s
\geq \frac{1}{2}$, le raisonnement par l'absurde sera utilisé. 
Soient $x_s$ et $x_r$ les variables relatives aux sommets $r$ et $s$ adjacents à l'arête $(rc)$ et
telles que, après relaxation des contraintes, on a : $x_r < \frac{1}{2}$ \underline{et} $x_s <
\frac{1}{2}$. On en déduit donc que $x_r + x_s < 1$ et donc la contrainte liée à l'arête $(rs)$ est
violée, l'hypothèse de départ est donc fausse. On a donc, $\forall (rs) \in V : x_r \geq \frac{1}{2}
\mbox{ et } x_s \geq{1}{2}$.

\subsubsection{(e) Une 2-approximation}

Mettons en évidence le pire des cas pouvant se présenter : pour une arête $(rs) \in V$, un seul
sommet est nécessaire pour la couverture de cette dernière dans le cas de la couverture minimale,
mais l'algorithme approché retourne : $x_r = x_s = \frac{1}{2}$. Après la phase d'arrondis, on a $x_r
= x_s = 1$ et donc les deux sommets appartiennent à la solution approchée, cette phase multiplie
donc au pire le nombre de sommets (pour chaque clause par 2), ce qui implique que le cardinal de la
solution approximée est au plus 2 fois la solution optimale.

Cet algorithme est donc une 2-approximation.

\subsubsection{(f) Dans le cas d'un graphe valué}
\paragraph{i. Programme linéaire}
$$
\left \lbrace \begin{array}{l}
	\min z = \sum\limits_{(i,j)\in E}{(x_i \times w_{ij} + x_j \times w_{ij})}\\
	x_r + x_s \geq 1, \quad \forall \{v_r, v_s\} \in E \\
	x_j \in \{0,1\} \quad j \in \{1, \ldots, n\}
\end{array} \right .
$$

\paragraph{ii. PLNE avec matrice\\}

Posons la matrice A, la $(n,n)$-matrice d'adjacente du graphe et X le vecteur contenant les variables existentielles définies précédemment. Le PLNE correspondant est le suivant :
$$
\left \lbrace \begin{array}{l}
	\min z = \sum\limits_{(i,j)\in E}{(x_i \times w_{ij} + x_j \times w_{ij})}\\
	A.X \geq 1\\
	x_j \in \{0,1\} \quad j \in \{1, \ldots, n\}
\end{array} \right .
$$

\paragraph{iii. Algorithme 2-approché}
% TODO


\subsection{Seconde approche : la recherche d'un couplage maximal}

\subsubsection{(a) Une 2-approximation}

Commençons par prouver que l'algorithme retourne une couverture des arêtes par les sommets.
Considérons donc une arête $(rs)$ non couverte par l'ensemble de sommets retourné par l'algorithme, par
définition du couplage, il serait donc possible d'ajouter $(rs)$ au couplage. Or le couplage calculé
par l'algorithme est maximal, on en déduit que l'arête $(rs)$ telle qu'elle est définie ne peut
exister et donc que l'ensemble de sommets obtenu couvre l'ensemble des arêtes du graphe.

Appelons $c$ le couplage calculé par l'algorithme et $x^*$ la solution optimale du problème de la
couverture par les sommets, on sait que $\card c \leq \card x^*$, or pour construire la solution
approchée, on ajoute à $C$ les deux extrêmités des arêtes utilisées pour le couplage. On a donc :$$
\card C = 2 \times \card c \leq 2 \times \card x^*$$ $$\Longrightarrow \frac{\card C}{\card x^*} =
2$$

\subsubsection{(b) Exemple de graphe pathologique}

Le graphe suivant met en évidence la borne 2 de l'algorithme.

\begin{tikzpicture}
	\tikzset{node/.style={circle}};
	\foreach \x/\xt/\col/\colb/\colc/\cold in {0/A/black/black/black/black,
	4/B/black/green/black/black, 8/C/black/black/red/black, 12/D/green/green/black/black}{
		\node[node, draw=\col] (\xt1) at (\x, 0) {$A$};
		\node[node, draw=\colb] (\xt2) at (\x+2, 0) {$B$};
		\node[node, draw=\col] (\xt3) at (\x+2, 2) {$C$};
		\node[node, draw=\colb] (\xt4) at (\x, 2) {$D$};

		\draw[-, \colc] (\xt1) -- (\xt2);
		\draw[-, \cold] (\xt2) -- (\xt3);
		\draw[-, \colc] (\xt3) -- (\xt4);
		\draw[-, \cold] (\xt4) -- (\xt1);
	}
	\node at (0, -2) {Le graphe};
	\node at (4, -2) {Une couverture minimale};
	\node at (9, -2) {Un couplage maximale};
	\node at (13, -2) {La solution renvoyée};
\end{tikzpicture}

\subsubsection{(c) Application de l'algorithme}
L'application de l'algorithme 3 au graphe d'exemple de la figure 1 donne une couverture $C = \{a_i\} \cup \{b_i\}$ alors que la solution optimale est $C = \{b_i\}$. Si le graphe de la figure 1 est bien le pire des cas pour cet algorithme alors c'est un algorithme $1.6$-approché.

\subsection{Utilisation d'un algorithme primal-dual}
% TODO

\section{Sur le problème de la couverture d'ensembles}

\subsection{Modélisation du problème à l'aide de la PLNE}

On considère le problème de la couverture d'ensemble, défini par :
Soit $E = \lbrace e_1, \dots, e_n \rbrace$, soient $S_1, \dots, S_m$ des sous-ensembles non vides
de $E$ tels que $\forall i \in \{1, \dots, m\}$, on a : $S_i \subset E$. On associe à chaque
ensemble $S_j$ un poids $w_j \geq 0$. Le problème consiste à trouver une collection de sous-ensemble
de poids minimum et telle que $\bigcup_{i=1}^m S_i = E$.

Ce problème peut s'exprimer à l'aide de la Programmation Linéaire en Nombres entiers de la manière
suivante : $$
\left \lbrace
\begin{array}{l}
	min z = \sum_{i=1}^m w_ix_i \\
	\sum_{j / e_i \in S_j} x_j \geq 1 \quad \forall e_i \in E \\
	x_i \in \{0, 1\}
\end{array} \right.
$$

\subsection{Procédure d'arrondis}

Soit $f = \max_{i=1,\dots,n} f_i$, avec $f_i$ le cardinal de l'ensemble des sous-ensembles de $E$
contenant $e_i$ : $ f_i = |\{j : e \in S_j\}|$.

Définissons la procédure d'arrondis suivante: $$
x_i = \left \{ \begin{array}{rcl}
		1 & \mbox{ si } & x_i \geq \frac{1}{f} \\
		0 & \mbox{ sinon } &
	\end{array} \right .
	$$

Cherchons à démontrer que cette procédure d'arrondis garantit une solution réalisable. Pour ce
faire, considérons l'ensemble $C$ des sous-ensembles sélectionnés par la procédure d'arrondis : $$ C
= \{ S_i : x_i \geq \frac{1}{f} \} $$

Soir un élément quelconque $c$, par définition de $f$, $c$ appartient à, au plus $f$ sous-ensembles.
De plus, par définition du problème, on a : $$
\sum_{j : c \in S_j} x_j \geq 1 $$

On a donc au moins un des $x_j \geq \frac{1}{f}$, et donc $c$ est couvert par la solution. Cette
procédure d'arrondis garantit une solution réalisable.

\subsection{Existence d'un algorithme $f$-approché}

Considérons l'algorithme \ref{algo_pourri}
\begin{algorithm}
	\caption{Approximation couverture par ensemble}
	\label{algo_pourri}
	\begin{algorithmic}[1]
		\STATE Exprimer le problème en programmation linéaire en nombres entiers
		\STATE Résoudre la version relaxée
		\STATE Réaliser la procédure d'arrondis
	\end{algorithmic}
\end{algorithm}

En utilisant la procédure d'arrondis étudiée plus haut, on sait que pour chaque clause, il existe
$x_k$ tel que $x_k \geq \frac{1}{f}$, au pire des cas \footnote{cas similaire à celui de l'exercice
1}, chaque $x_k$ est multiplié par $f$, ce qui implique que la solution obtenue est au plus $f$ fois
plus grande que la solution optimale. Il s'agit donc d'un algorithme f-approché.

\subsection{Cas ou $f=2$}

Si $f=2$ alors le problème devient une couverture minimum par sommets.

% !TEX root = ../../rapport/rapport.tex
% !TEX encoding = UTF-8 Unicode

\section{Sur le problème du couplage maximum de poids minimum}

\subsection{Modélisation du problème}
Soit un graphe $G = (V,E)$ avec $V$ l'ensemble de ses sommets et $E$ l'ensemble de ses arêtes. Soit $\{ \forall{(i,j) \in E}, X_{(i,j)}\}$ un ensemble de variables booléennes qui indiquent le choix de l'arête $(i,j)$ correspondante dans le couplage. Soit $P_{(i,j)}$ le poids de l'arête $(i,j)$.

La première modélisation intuitive est la suivante :

$ Maximiser $ $$ \sum\limits_{(i,j) \in E} (X_{(i,j)} \times H) - \sum\limits_{(i,j) \in E} (X_{(i,j)} \times P_{(i,j)}) $$
$ Sous\ contraintes $
$$ \forall{i \in V}, \sum\limits_{(i,j) \in E} X_{(i,j)} + \sum\limits_{(j,i) \in E} X_{(j,i)} \leq 1 $$
$$ \forall{(i,j) \in E}, X_{(i,j))} \in \{0,1\} $$
$$ \forall{(i,j) \in E}, P_{(i,j))} \geq 0 $$
avec H une constante très grande. Cependant, après quelques instants de réflexion, nous pouvons imaginer une modélisation plus élégante :

$ Minimiser $ $$ \sum\limits_{(i,j) \in E} (X_{(i,j)} \times P_{(i,j)}) $$
$ Sous\ contraintes $
$$ \forall{i \in V}, \sum\limits_{(i,j) \in E} X_{(i,j)} + \sum\limits_{(j,i) \in E} X_{(j,i)} = 1 $$
$$ \forall{(i,j) \in E}, X_{(i,j))} \in \{0,1\} $$
$$ \forall{(i,j) \in E}, P_{(i,j))} \geq 0 $$

En effet, les contraintes forcent le couplage à être maximum tandis que la fonction objectif le force à tendre vers le poids minimum.

\subsection{Modélisation du problème appliquée au graphe de la figure 2}
$ Minimiser $ $$ \epsilon \times X_{ab} + \epsilon \times X_{bc} + \epsilon \times X_{ac} + M \times X_{ae} + M \times X_{cd} + M \times X_{bf} + \epsilon \times X_{df} + \epsilon \times X_{de} + \epsilon \times X_{fe}$$
$ Sous\ contraintes $
$$ X_{ab} + X_{ac} + X_{ae} = 1 $$
$$ X_{ab} + X_{bc} + X_{bf} = 1 $$
$$ X_{ac} + X_{bc} + X_{cd} = 1 $$
$$ X_{cd} + X_{de} + X_{df} = 1 $$
$$ X_{ae} + X_{de} + X_{df} = 1 $$
$$ X_{ef} + X_{df} + X_{bf} = 1 $$
$$ \forall{(i,j) \in E}, X_{(i,j))} \in \{0,1\} $$
$$ \epsilon \geq 0 $$
$$ M \geq 0 $$

\subsection{Solution optimale entière $z(ILP)$}
Sur un exemple de cette taille, il est facile de trouver une solution à la main. Il y a plusieurs solutions optimales de poids total $M+2\epsilon$ sur cet exemple ; l'une d'entre elles est le couplage $\{(a,b), (c,d), (e,f)\}$.

La résolution de ce PLNE par $glpsol$ (solveur de $GLPK$) donne bien la même solution.

\subsection{Solution optimale $z(LP)$ pour le programme relaxé}
Relaxer le programme revient à transformer la contrainte d'intégrité des $X_{(i,j)}$ en la contrainte suivante :
$$ \forall{(i,j) \in E},  0 \leq X_{(i,j))} \leq 1 $$

En rentrant le PL relaxé dans $glpsol$, nous obtenons le couplage de poids total $3\epsilon$ : $$\{X_{ab} = 0.5 ; X_{ac} = 0.5 ; X_{bc} = 0.5 ; X_{df} = 0.5 ; X_{ef} = 0.5 ; X_{de} = 0.5\}$$ 

\subsection{Conclusion sur la pertinence de la formulation}
La solution trouvée au PLNE étant optimale, la formulation du problème semble être pertinente.



\section{Sur le problème de la coupe maximum}

\subsection{Complexité}

A chaque itération de l'algorithme, la valeur de la coupe maximale augmente au minimum d'une unité.
Or la valeur de la coupe maximale étant bornée par le nombre d'arêtes, on obtient donc que
l'algorithme effectue au plus $m$ opérations. On a donc un algorithme en $O(m)$.

\subsection{Un algorithme 2-approché}

Considérons $(Y_1, Y_2)$ la coupe renvoyée par l'algorithme, nous chercherons dans un premier temps
à montrer que chaque sommet dans $Y_1$ admet au moins autant d'arêtes dans $Y_2$ que dans $Y_1$.

Pour ce faire, considérons un sommet $v \in Y_1$, supposons que ce sommet possède plus d'arêtes dans
$Y_1$ que dans $Y_2$, nous noterons $a_1$ le nombre d'arêtes incidentes à $v$ dans $Y_1$ et $a_2$ le
nombre d'arêtes adjacentes à $v$ dans $Y_2$. Déplacer $v$ de $Y_1$ vers $Y_2$ reviendrait à diminuer
la coupe de $a_2$ et augmenter celle-ci de $a_1$, or d'après l'hypothèse de départ $a_2 < a_1$, on
observerait une augmentation de la valeur de la coupe, ce qui est impossible si $(Y_1, Y_2)$ est une
coupe retournée par l'algorithme. On en déduit donc que l'hypothèse de départ est fausse.

Autrement dit, pour un sommet $v \in V$, en notant $d_v$ le degré de $v$, $v_c$ le nombre d'arêtes
adjacentes à $v$ traversant la coupe et $v_s$ le nombre d'arêtes adjacentes à $v$ ne la traversant
pas, on peut écrire : $$
\begin{array}{lrcl}
	&v_c + v_s &=& d_v \quad \mbox{or on a } v_c \geq v_s \\
	\Rightarrow & v_c & \geq & \frac{d_v}{2}
\end{array} $$

Si l'on généralise sur l'ensemble du graphe, on peut en déduire : $$
\begin{array}{rcl}
	|(Y_1,Y_2)| & = & \frac{1}{2} \sum_{v \in V} v_c\\
							& \geq & \frac{1}{2} \sum_{v \in V} \frac{d_v}{2} \\
					 	& = & \frac{m}{2}
\end{array}
$$

On a donc : $|(Y_1,Y_2)| \geq \frac{m}{2}$, de plus comme vu précédemment, la valeur de la coupe
maximale (que nous noterons $OPT$) est bornée par le nombre d'arêtes. On peut donc en déduire : $$
\frac{OPT}{|(Y_1,Y_2)|} \leq \frac{m}{\frac{m}{2}} = 2 $$

L'algorithme donné est donc bien un algorithme 2-approché.

\subsection{Atteindre la borne}
%TODO

% !TEX root = ../Rapport/rapport.tex

\section{Sur le problème de partition}\label{ex5}

\subsection{D\'efinition}\label{ex5_def}
Soit un ensemble d'objets $A = \{a_i : i \in [1;n]\}$ et une fonction de poids $p : A
\rightarrow \mathbb{N}$.\\
Le problème de décision est le suivant : \\
existe-t-il $Y_1,Y_2 \subseteq A\ Y_1 \cap Y_2 = \emptyset: \sum_{y_{1} \in Y_1}p(y_1) = \sum_{y_2 
\in Y_2}p(y_2) = P$ ?\\
Son problème d'optimisation associé est défini comme suit :\\
$minimiser\ \sum_{y_1 \in Y_1}p(y_1) : \sum_{y_1 \in Y_1}p(y_1) - \sum_{y_2 \in Y_2}p(y_2) 
\geq 0$.

On note :
\begin{itemize}
	\item $w(E) = \sum_{e \in E}p(e)$ avec $E \subseteq A$,
	\item $L = \frac{w(A)}{2}$.
\end{itemize}

Soit l'algorithme suivant :

\begin{center}
\begin{algorithm}[H]
\caption{PTAS Partition}\label{ex5_algo}
\algsetup{indent=2em,linenodelimiter= }
\begin{algorithmic}[1]
\REQUIRE $A = \{a_i : i \in [1,n]\}$ un ensemble de n objets,
		 $p : A \rightarrow \mathbb{N}$ une fonction de poids sur ceux-ci,
		 $r > 1$ le rapport d'approximation désiré
	\IF {$r \geq 2$}
		\RETURN $A,\emptyset$
	\ENDIF
	\STATE Trier $A$ par ordre décroissant de $p$
	\STATE Soit $(a_1,...,a_n)$
	\STATE $k(r) \leftarrow \lceil\frac{2-r}{r-1}\rceil$
	\STATE {\em Première phase}
	\STATE Trouver une partition optimale $Y_1,Y_2$ de $a_1,...,a_{k(r)}$
	\FOR {$j = k(r)+1$ à $n$}
		\IF {$\sum_{y_1 \in Y_1}p(y_1) \leq \sum_{y_2 \in Y_2}p(y_2)$}
			\STATE $Y_1 \leftarrow Y_1 \cup \{a_j\}$
		\ELSE
			\STATE $Y_2 \leftarrow Y_2 \cup \{a_j\}$
		\ENDIF
	\ENDFOR
	\RETURN $Y_1,Y_2$
\end{algorithmic}
\end{algorithm}
\end{center}

\subsection{Question 1}\label{ex5_q1}
Montrons que $\forall r \geq 2$, la solution $A,\emptyset$ est une r-approximation.\\
On sait que : $w(A) = w(Y_1) + w(Y_2)$ donc $w(Y_1) \geq \frac{w(A)}{2}$. Ie,
$sol_{opt}(A) \geq \frac{w(A)}{2}$.
La valeur de la solution $A,\emptyset$ $sol_{algo}(A)$ vaut $w(Y_1) = w(A)$.
On a alors $w(Y_1) = 2 \times \frac{w_(A)}{2}$ et donc $w(Y_1) \leq 2 \times
sol_{opt}(A)$.\\
Nous avons bien $sol_{algo}(A) \leq 2 \times sol_{opt}(A) \leq r \times sol_{opt}(A)$.

\subsection{Question 2}\label{ex5_q2}
Soit $a_h$ le dernier \'el\'ement ins\'er\'e dans $Y_1$.\\

\subsubsection{(a)}\label{ex5_q2_a}
Nous devons montrer que $w(Y_1) - L \leq \frac{p(a_h)}{2}$.\\
Posons $w(Y_1)'$ et $w(Y_2)'$ les poids respectifs deux partitions avant l'ajout de
$a_h$.\\
Puisque $a_h$ est ajout\'e \`a $Y_1$, on sait que $w(Y_1)' \leq w(Y_2)'$.
De plus, $a_h$ est le dernier \'el\'ement ajout\'e \`a $Y_1$, donc $w(Y_1) = w(Y_1)' +
p(a_h)$.
Cela implique aussi que tous les \'el\'ements suivants sont ajoutés dans $Y_2$, donc
$w(Y_2)' \leq w(Y_2)$.\\
On a donc $w(Y_1)' = w(Y_1) - p(a_h) \leq w(Y_2)' \leq w(Y_2)$.\\
$w(Y_1) - w(Y_2) \leq p(a_h) \leftrightarrow$
$w(Y_1) + w(A) - w(Y_2) - w(A) \leq p(a_h)$\\
$\leftrightarrow$
$2 \times w(Y_1) - w(A) \leq p(a_h) \leftrightarrow$
$w(Y_1) - \frac{w(A)}{2} \leq \frac{p(a_h)}{2}$\\
$\leftrightarrow$
$w(Y_1) - L \leq \frac{p(a_h)}{2}$.

\subsubsection{(b)}\label{ex5_q2_b}
Si $a_h$ est ins\'er\'e durant la première phase, l'algorithme renverra la solution
optimale. En effet $a_h$ est le dernier élément de $Y_1$, donc tous les suivants seront
insérés dans $Y_2$, de plus $w(Y_1) \geq w(Y_2)$.
Or, $a_h$ est inséré dans la première phase, c'est à dire dans la phase de recherche
optimale. On note $Y_1',Y_2'$ les ensembles à la fin de celle-ci.
On sait donc que $w(Y_1)'$ est la solution optimale du sous problème $Y_1' \cup
Y_2'$.
On a alors $w(Y_1) = w(Y_1)' = sol_{opt}(A')$ avec $A' = Y_1' \cup Y_2'$, puisque tous
les éléments suivants sont insérés dans $Y_2$, $sol_{opt}(A') \leq sol_{opt}(A)$.\\
On a donc $w(Y_1) \leq sol_{opt}(A)$ et $w(Y_1) \geq sol_{opt}(A)$ par définition :
$sol_{algo}(A) = sol_{opt}(A)$.

\subsubsection{(c)}\label{ex5_q2_c}
Supposons maintenant que $a_h$ est inséré durant la seconde phase, et intéressons nous aux 
$p(a_j)\ \forall j \in [1;k(r)]$. 
Puisque $a_h$ est inséré dans la seconde phase ($h \geq k(r)+1$), et que les $a_i$ sont
rangés par ordre décroissant de poids, on a : $p(a_h) \leq p(a_j)\ \forall j \in [1;k(r)+1]$.\\
Ce qui implique : $\sum_{j = 1}^{k(r)+1}p(a_h) \leq \sum_{j = 1}^{k(r)+1} \times p(a_j)$.\\
Or, $\sum_{j=1}^{k(r)+1}p(a_j) \leq w(A)$. \\
On a donc $(k(r)+1) \times p(a_h) \leq w(A) = 2 \times L$.

\subsubsection{(d)}\label{ex5_q2_d}
Toute solution optimale est minorée par $\frac{w(A)}{2}$.

\subsubsection{(e)}\label{ex5_q2_e}
Nous allons montrer que le ratio est bien majorée par $r$.\\
D'après la question \ref{ex5_q2_a}, nous savons que $w(Y_1) - L \leq \frac{p(a_h)}{2}$.
Donc $2 \times w(Y_1) - 2 \times L \leq p(a_h)$.
Or, par \ref{ex5_q2_c}, nous avons : $p(a_h) \leq \frac{2\times L}{k(r)+1}$.\\
On obtient alors $2 \times w(Y_1) - 2 \times L \leq 2 \times (r - 1) \times L$
$\leftrightarrow w(Y_1) \leq r \times L$.\\
On a bien : $sol_{algo}(A) \leq r \times sol_{opt}(A)$.

Nous pouvons en conclure que l'algorithme \ref{ex5_algo} renvoit bien une solution
r-approchée.


\section{Sur le problème du sac à dos simple}

\subsection{Construction d'un algorithme approché}

\subsubsection{Complexité de l'algorithme}

La complexité de l'algorithme, une fois les nombres triés, est en $O(n)$. Cependant, il est possible
de démontrer que la complexité des algorithmes de tris basés sur des fonctions de comparaisons ne
peut être inférieures à $O(n\log n)$. la complexité de cet algorithme est donc similaire à la
compléxité de l'algorithme de tri utilisé.

\subsubsection{Minoration de $cost(T)$}

Dans un premier temps nous montrerons que l'existence d'un indice $j$ pour lequel $cost(T) + w_{j+1}
> b$ n'est pas systématique, mais que si ce dernier n'existe pas, le problème n'existe pas non plus.

En effet, si cet indice n'existe pas, on a alors : 
$$ \sum_{i=1}^n w_1 \leq b $$
On en déduit donc que la solution optimale est donnée par $ T = \{ 1, \dot, n \} $. Ce qui en soit
même n'a aucun intérêt\footnote{Le problème du sac à dos, dans cette configuration n'est plus
NP-Complet}. Nous admettrons donc l'existence de cet indice $j$.

Supposons, à l'indice $j$, que $cost(T) \leq \frac{b}{2}$. Par définition de l'indice $j$, on a
$cost(T) + w_{j+1} > b$, ce qui implique $w_{j+1} > \frac{b}{2}$ et donc : $$
\begin{array}{lrcl}
	& w_{j+1} & > & cost(T) \\
	\Rightarrow & w_{j+1} & > & \sum_{i=1}^j w_i \\
	\Rightarrow & w_{j+1} & > & w_i
\end{array}
$$

Or les éléments $w_i$ étant triés, on a $w_j \geq w_{j+1}$. L'hypothèse de départ est donc fausse.

\subsubsection{Performance relative de l'algorithme}

Notons $OPT$ la solution optimale au problème du sac à dos, et $A$ la solution approchée donnée par
l'algorithme. Rappelons, par définition du problème, que $OPT \leq b $. La performance relative nous
est donnée par : $$
\frac{OPT}{A} \leq \frac{b}{\frac{b}{2}} = 2 $$

La performance relative de l'algorithme est donc 2.

\subsection{Contruction d'un schéma d'approximation}

\subsubsection{Complexité de l'algorithme}

De façon très grossière, la construction de l'ensemble des sous-ensembles $S_k$ de $k$ éléments se
fait en $k\times n^k$ opérations. On a donc une construction des sous-ensembles réalisée en : $$
\sum_{i=1}^k i n^i $$. Soit une construction en $O(n^k)$.

L'ensemble étant déjà trié, l'algorithme glouton s'éxécute en $O(n)$, d'où une exécution totale de
l'algorithme en $O(n^{k+1})$.


% !TEX encoding = UTF-8 Unicode
% !TEX root = ../Rapport/rapport.tex

\section{Programmation dynamique}

\subsection{Sur le problème de la partition}
\subsubsection{Condition nécessaire sur la somme des poids des $n$ objets}

Les poids de chacun des objets étant des valeurs entières, pour qu'il existe deux sous-ensembles
distincts ayant le même poids, il faut que la somme des poids des objets soit paire.

\subsubsection{Récurrence}

Introduisons l'expression booléenne $T(i,j) $ : \emph{Etant donnés les $i$ premiers éléments de la
famille, il existe un sous-ensemble de ces $i$ éléments de poids $j$}. 

Dans un premier temps il paraît évident que si $j=0$, le sous-ensemble existe : il s'agit de
l'ensemble vide. \begin{equation}
\forall i \in \mathbb{[}0, n \mathbb{]}, \quad T(i,0) = \mbox{VRAI} 
\end{equation}

De plus, si $j$ est égal au poids de l'un des $i$ premiers éléments, alors l'existence du
sous-ensemble est avérée. Autrement dit, si l'on appelle $K_i$ l'ensemble formé par les poids des
objets $i$, on a : \begin{equation}
	\forall i \in \mathbb{[}0, n \mathbb{]}, \quad T(i,j \in K_i) = \mbox{VRAI}
\end{equation}

Troisièmement, l'ensemble des $i-1$ premiers objets est un sous ensemble de l'ensemble des $i$
premiers objets : $E_{i-1} \subset E_i$. On en déduit donc, que s'il existe un sous ensemble de
$E_{i-1}$ de poids $j$ tel que $T(i-1, j) = \mbox{VRAI}$, alors ce sous-ensemble est aussi un
sous-ensemble de $E_i$ tel que $T(i, j) = \mbox{VRAI}$ et ce par le principe d'inclusion.
\begin{equation}
	\forall i \in \mathbb{[}0, n \mathbb{]}, \quad \mbox{ si } T(i-1, j) = \mbox{VRAI}, \mbox{
	alors } T(i,j) = \mbox{VRAI}
\end{equation}

Enfin, considérons $S_j$ un sous-ensemble de $E_{i-1}$ de pois $j$ vérifiant donc $T(i-1, j) =
\mbox{VRAI}$, alors $S_j \cup a_i$ est un sous-ensemble de $E_i$ de poids $j+p(a_i)$ et donc
vérifiant $T(i,j+p(a_i)) = \mbox{VRAI}$. On en déduit donc :
\begin{equation}
	\forall i \in \mathbb{[}0, n \mathbb{]}, \quad \mbox{ si } T(i-1, j-p(a_i)) = \mbox{VRAI}, \mbox{
	alors } T(i,j) = \mbox{VRAI}
\end{equation}

En réunissant les équations ci dessus dans une seule expression booléenne, on obtient la formule de
la ligne $i$ en fonction de la ligne $i-1$ et $p(a_i)$ :
\begin{equation}
	T(i,j) = j == 0 \vee j == p(a_i) \vee T(i-1, j) \vee T(i-1, j-p(a_i))
\end{equation}

On peut alors donner un algorithme pour résoudre le problème de la partition : l'algorithme
\ref{solvepart}.

\begin{algorithm}
	\caption{solve-partition}
	\label{solvepart}
	\begin{algorithmic}[1]
		\STATE $p_{max} \leftarrow \sum_{i \in \mathbb{[}1, n \mathbb{]}} p(a_i)$
		\IF{$p_{max} \equiv 1 \pmod{2}$}
			\STATE Pas de solutions
		\ELSE
			\FOR {$j \in \mathbb{[}0,p_{max}\mathbb{]}$}
				\IF{$j = 0$}
					\STATE $T(0, j) \leftarrow \TRUE$
				\ELSE
					\STATE $T(0, j) \leftarrow \FALSE$
				\ENDIF
			\ENDFOR
			\FOR{$i \in \mathbb{[}1, n \mathbb{]}$}
				\FOR{$j \in \mathbb{[}0, p_{max}\mathbb{]}$}
					\STATE $T(i, j) \leftarrow j = 0 \vee j = p(a_i) \vee T(i-1, j) \vee T(i-1, j-p(a_i))$
				\ENDFOR
			\ENDFOR
			\IF{$T(n, p_{max} / 2) = \FALSE$}
				\STATE Pas de solutions
			\ELSE
				\STATE Construire le sous ensemble solution
			\ENDIF
		\ENDIF
	\end{algorithmic}
\end{algorithm}

Le sous ensemble solution $S_{sol}$ se construit à l'aide du principe suivant : pour une ligne $i>0$ et
un poids $j>0$ donnés, on a $a_i \in S_{sol}$ si $T(i-1, j) = \mbox{FAUX} \wedge T(i,j) =
\mbox{VRAI}$. Ceci n'est vrai que dans un sens et permet d'introduire l'implication suivante :
\begin{equation}
	T(i-1, j) \oplus T(i, j) = \mbox{VRAI} \Rightarrow a_i \in S_{sol} 
\end{equation}

\subsubsection{Complexité}


\subsubsection{Jeux d'essais}


\subsection{Le problème du sac à dos}
\subsubsection{Justification des formules}

\subsubsection{Exemple}

\subsubsection{Complexité}


\subsection{Le problème du voyageur de commerce}
\subsubsection{Exemple}

\subsubsection{Complexité}

% !TEX encoding = UTF-8 Unicode
% !TEX root = ../Rapport/rapport.tex

\section{Sur le produit matriciel}

\subsection{Nombre d'opérations nécessaires pour un produit}
Soit $M_i$ une $(p_i,p_{i+1})$-matrice. Soient $\pi _1$ et $\pi _2$ les produits matriciels à parenthésage symétrique suivants : 
$$\pi _1 = (M_1(M_2(M_3(M_4(...(M_{k-1}.M_k))))))$$
$$\pi _2 = ((((((M_1.M_2)M_3)M_4)...)M_{k-1})M_k)$$
Nous admettons que le produit d'une $(p,q)$-matrice et d'une $(q,r)$-matrice peut se faire à l'aide de $p.q.r$ opérations. Tous les produits matriciels envisagés par la suite sont définis car le nombre de colonnes de la première matrice est égal au nombre de lignes de la seconde.

Évaluons le nombre d'opérations nécessaires au produit $\pi _1$ grâce au tableau suivant :
\begin{center}
\resizebox{\textwidth}{!}{
$ \begin{array}{|c|c|c|c|c|c|c|} \hline
	\pi _l & \pi _{1,1} & … & \pi _{1,4} & … & \pi _{1,k-2} & \pi _{1,k-1} \\ \hline
	=  & M_1 \times & … & M_{4} \times  & … & M_{k-2} \times & (M_{k-1} \times M_k) \\
	Nb.\ d'operations & 1.2.(k+1) & … & 4.5.(k+1) & … & (k-2).(k-1).(k+1) & (k-1).k.(k+1) \\
	Taille\ de\ la\ matrice & (1,k+1) & … & (4,k+1) & … & (k-2,k+1) & (k-1,k+1) \\ \hline
 \end{array} $
 }
 \end{center}
 
 Nous pouvons alors déduire la formule donnant le nombre d'opérations nécessaires pour calculer le produit $\pi _1$ :
$$ \sum\limits_{i =1}^{i < k} {[i\times (i+1) \times (k+1)]} $$

De la même façon, évaluons le nombre d'opérations nécessaires au produit $\pi _2$ grâce au tableau suivant :
\begin{center}
$ \begin{array}{|c|c|c|c|c|c|c|} \hline
	\pi _l & \pi _{1,1} & \pi _{1,2} & \pi _{1,3} & … & \pi _{1,k-2} & \pi _{1,k-1} \\ \hline
	=  & M_1 \times M_2 & \times M_3 & \times M_{4}  & … & \times M_{k-1} & \times M_k \\
	Nb.\ d'operations & 1.2.3 & 1.3.4 & 1.4.5 & … & 1.(k-1).k & 1.k.(k+1) \\
	Taille\ de\ la\ matrice & (1,3) & (1,4) & (1,5) & … & (1,k) & (1,k+1) \\ \hline
 \end{array} $
 \end{center}
 
 Nous pouvons alors déduire la formule donnant le nombre d'opérations nécessaires pour calculer le produit $\pi _2$ :
$$ \sum\limits_{i =1}^{i < k} {[(i+1) \times (i+2)]} $$

Comparons maintenant le nombre d'opérations nécessaires aux produits $\pi_1$ et $\pi_2$ :

\begin{center}
$ \begin{array}{|c|C{4.5cm}|C{4.5cm}|C{4.5cm}|} \hline
	 & \sum\limits_{i =1}^{i < k} {[i\times (i+1) \times (k+1)]} & vs & \sum\limits_{i =1}^{i < k} {[(i+1) \times (i+2)]} \\ \hline
	k=1  & 4 & < & 6  \\
	k=2 & 24 & > & 18 \\
	k=3 & 80 & > & 38 \\ \hline
 \end{array} $
 \end{center}
 
Nous déduisons du tableau précédent (et du fait que la première expression comporte un terme de plus) que $\forall k \geq 2$ le nombre d'opérations nécessaires pour calculer le produit $\pi_1$ est supérieur au nombre d'opérations nécessaires pour calculer le produit $\pi_2$. Le nombre d'opérations nécessaires pour calculer un produit matriciel dépendant du parenthésage choisi.

 
 
\subsection{Nombre de parenthésages possibles d'un produit de $k$ matrices}


\subsection{Récurrence}


\subsection{Exemple}

\section{Résolution numérique}
a
\section{Seuil d'approximation pour le problème Bin Packing}\label{ex10}

\subsection{Question 1}\label{ex10_q1}
Soit un ensemble $O = \{o_i : i \in [1;n]\}$ de $n$ objets.\\
Soit une fonction $q : O \rightarrow \mathbb{N}$.\\
Soit une taille de boîte $Q$.

Le problème de décision Bin Packing est le suivant :\\
Existe-t-il une k-partition $O_1,...,O_k \subseteq O$ telle que :
\begin{itemize}
	\item $O = \cup_{i = 1}^{k}O_i$,
	\item $\forall i,j \in \mathbb{N}$, $i \neq j$, $O_i \cap O_j = \emptyset$,
	\item $\sum_{o_{ij} \in O_i} \leq Q\ \forall i \in [1;k]$.
\end{itemize}

Le problème d'optimisation associé est :\\
Minimiser $k \in \mathbb : \exists$ k-partition qui respecte les conditions ci-dessus.

Dans la suite on notera $B(O_i) = \sum_{o_{ij} \in O_i}q(o_i)$ la somme des poids des
éléments de $O_i$, $B = B(O)$, et $K^*$ la solution optimale au problème d'optimisation.

\subsection{Question 2}\label{ex10_q2}
Remarque : pour les définitions et notations concernant le problème de partition, voir
l'\href{ex5}{exercice \ref{ex5}}.

Soit $x$ une instance du problème partition, nous construisons
une instance $x'$ de Bin Packing de la manière suivante :
\begin{itemize}
	\item $O = A$,
	\item $q(o_i) = \frac{2 \times p(a_i)}{w(A)}$,
	\item $Q = 1$.
\end{itemize}
%TODO


\subsection{Question 3}\label{ex10_q3}
Nous devons montrer que si $x$ est positive alors $K^*(x') = 2$ et que $K^*(x') = 3$
sinon.
Commençons par calculer $B = \sum_{a_i \in A} \frac{2 \times p(a_i)}{w(a)} =
\frac{2}{w(a)} \times \sum_{a_{i} \in A} = \frac{2}{w(a)} \times w(a) = 2$.

Montrons tout d'abord que $K^*(x') \leq 3$.\\
Nous supposons que $K(x') = 4 > 3$. Nous savons alors qu'il existe quatre ensembles
$O_1,O_2,O_3,O_4$ tels que $B(O_i) \leq 1$.
De plus, puisque les $O_i$ sont disjoints, nous avons $B = B(O_1)+B(O_2)+B(O_3)+B(O_4)$,
et nous pouvons énumérer les $O_i$ de sorte que $B(O_i) \geq B(O_{i+1})$.\\
Or, $B(O_i) + B(O_j) > 1\ \forall i \neq j$, en effet si ce n'était pas le cas, alors on
pourrait unir les deux ensembles et $K^*(x')$ serait égal à 3.
En particulier nous avons : $B(O_1)+B(O_2) > 1$ et $B(O_3)+B(O_4) > 1$, c'est à dire :
$2 = B(O_1) + B(O_2) + B(O_3) + B(O_4) > 2$, ce qui est absurde.\\
Nous avons donc que $K(x') \leq 3$.

Supposons que $x$ soit une instance positive.\\
Soit la solution $P$ de $x$ $Y_1,Y_2$, on a donc $w(Y_1) = w(Y_2) = \frac{w(A)}{2} = P$.
Soit $O_1$ (resp. $O_2$) l'image de $Y_1$ (resp. $Y_2$) dans le problème $x'$.\\
On a donc : $B(O_2) = B(O_1) = \frac{2 \times w(Y_1)}{w(A)} = \frac{2 \times w(A)}{2
\times w(A)} = 1 \leq Q$.\\
Ainsi $K^* = 2$.

Supposons maintenant que $x$ soit une instance négative et on suppose que $K(x') < 3$.
Si $K(x') = 1$, il est trivial de voir que $B = 2$ et que $B \leq Q = 1$ ce qui est
absurde.\\
Si $K(x') = 2$, on a alors $O = O_1 \cup O_2$. 
De plus, $B = 2$, $B = B(O_1) + B(O_2)$ et $B(O_i) \leq 1$.
On a donc $B(O_1) = B(O_2) = 1$.\\
Soit $Y_1$ (resp. $Y_2$) l'ensemble correspondant à $O_1$ (resp. $O_2$) dans le problème
x. On a :
$1 = w(O_1) = \frac{2}{w(A)} \times w(Y_1) \leftrightarrow w(Y_1) = \frac{w(A)}{2}$.
Donc il existe une solution $Y_1,Y_2$ au problème $x$, ce qui est absurde.\\
Ainsi si l'instance $x$ est négative, $K^*(x') = 3$.

\subsection{Question 4}\label{ex10_q5}
Supposons qu'il existe un algorithme A qui donne une solution $(\frac{3}{2} -
\epsilon)$-approchée au problème Bin Packing.\\
Soit $x$ un problème de Partition.
Grâce à la réduction polynomiale étudiée ci-dessus, nous avons donc :
$sol_{A}(x') < \frac{3}{2}sol_{opt}(x')$.\\
Et si le problème partition admet une solution alors $sol_{opt}(x') = 2$, nous en
déduisons : $sol_{A}(x') < 3$, ie $sol_A(x') = 2 = sol_{opt}(x')$.
A permet donc de résoudre le problème Partition en temps polynomial, celui-ci étant
NP-complet, il n'existe pas un tel algorithme.\\
Ainsi le problème Bin Packing n'appartient pas à la classe d'approximation $PTAS$.


\section{Seuil d'approximation pour le problème de la coloration de sommets (resp.
d'arêtes)}\label{ex11}

\subsection{Définitions}\label{ex11_def}
Soit un graphe quelconque $G = (V,E)$.

\subsubsection{Coloration de sommets}\label{ex11_def_sommets}
Le problème de décision est le suivant :\\
soit $k \in \mathbb{N}$, est-il possible de trouver une fonction $\chi : V \rightarrow
[1,k]$ telle que $\forall (x,y) \in E,\ \chi(x) \neq \chi(y)$.\\
Son problème d'optimisation associé est de trouver $k$ le plus petit possible tel que $G$
soit k-colorable (on note $\chi(G)$ un tel $k$).

\subsubsection{Coloration d'arêtes}\label{ex11_def_aretes}
Le problème de décision est le suivant :\\
soit $k \in \mathbb{N}$, est-il possible de trouver une fonction $\chi' : E \rightarrow
[1,k]$ telle que $\forall (x,y_1),(x,y_2) \in E,\ \chi'((x,y_1)) \neq \chi'((x,y_2))$.\\
Son problème d'optimisation associé est de trouver $k$ le plus petit possible tel que $G$
soit k-arêtes-colorable (on note $\chi'(G)$ un tel $k$).

\subsection{Question 1}\label{ex11_q1}
Ces deux problèmes de coloration appartiennent à la classe $NP$, de plus, celui de
sommets est non approximable, tandis que celui d'arêtes l'est mais n'appartient pas à la 
classe d'approximation $PTAS$.\\
Nous nous intéressons à déterminer un seuil d'approximation pour le problème
de coloration d'arêtes.

\subsection{Question 2}\label{ex11_q2}
Supposons qu'il existe un algorithme $(\frac{4}{3} - \epsilon)$-approché pour le
problème de coloration des arêtes.
Nous avons alors $sol_A < \frac{4}{3} sol_{opt}$.
De plus, ce problème d'optimisation étant un problème de minimisation, nous avons $sol_A
\geq sol_{opt}$.

\subsubsection{(a)}\label{ex11_q2_a}
Si $sol_{opt} \leq 3$, on a : $sol_A < \frac{4}{3} \times sol_{opt} \leq 4$,
c'est à dire $sol_A \leq 3$.\\
Cet algorithme renvoie donc la solution optimale.

\subsubsection{(b)}\label{ex11_q2_b}
Si $sol_{opt} \geq 4$, l'algorithme renverra une solution strictement supérieure à 3.
Ainsi il est possible de déterminer si la solution optimale est inférieure à 3.

\subsection{Question 3}\label{ex11_q3}
Etant donné que si un algorithme de rapport strictement inférieur à $\frac{4}{3}$
existait, nous pourrions répondre en temps polynomial au problème de savoir si un graphe est
3-arêtes-colorable (qui est NP-complet),
le seuil d'approximation minimum pour le problème de coloration d'arêtes est donc
supérieur ou égal à $\frac{4}{3}$.


% !TEX encoding = UTF-8 Unicode
% !TEX root = ../Rapport/rapport.tex


\tikzstyle{end} = [circle, minimum width=3pt,fill, inner sep=0pt]
\definecolor{gris}{gray}{0.85}

\section{Comparaison de \emph{branch and bound} et \emph{branch and cut}}

Nous considérons le  programme linéaire suivant :
$$
PL_0 \begin{cases}
max\ z(x_1,x_2) = 2x_1 + x_2 \\
 2x_1 + 5x_2 \leq 17 \\
 3x_1 + 2x_2 \leq 10 \\
 x_1, x_2 \geq 0 \\
\end{cases}
$$

\subsection{Polytope associé à ${PL}_0$}
\begin{tikzpicture}[scale = 0.6]

        \filldraw [smooth,draw=gray!20,fill=gray!20] plot[id=f3,domain=0:16.0/11.0]  function {(17.0/5.0) - (2.0/5.0)*x}
            -- (10.0/3.0,0)  -- (0,0) -- cycle;

    \draw[very thin,color=gray] (-5,-5) grid (10, 10);
    \draw[->] plot[id=axeX1] (-5,0) -- (10,0) node[right] {$x_1$};
    \draw[->] plot[id=axeX2] (0,-5) -- (0,10) node[above] {$x_2$};
    
    
    \draw[color=red, domain=-5:2.5, dashed] plot[id=obj] function{-2*x} 
        node[below] {$max\ z = 2x_1 + x_2$};
    \draw[color=blue, domain=-5:10] plot[id=c1] function{(17.0/5.0) - (2.0/5.0)*x} 
        node[right] {$2x_1 + 5x_2 \leq 17$};
    \draw[color=orange, domain=-10.0/3.0:20.0/3.0] plot[id=c2] function{(5.0)-(3.0/2.0)*x} 
        node[right] {$3x_1+2x_2 \leq 10$};
   	
\end{tikzpicture}


\subsection{Résolution graphique}
\begin{tikzpicture}[scale = 0.6]

        \filldraw [smooth,draw=gray!20,fill=gray!20] plot[id=f3,domain=0:16.0/11.0]  function {(17.0/5.0) - (2.0/5.0)*x}
            -- (10.0/3.0,0)  -- (0,0) -- cycle;

    \draw[very thin,color=gray] (-5,-5) grid (10, 10);
    \draw[->] plot[id=axeX1] (-5,0) -- (10,0) node[right] {$x_1$};
    \draw[->] plot[id=axeX2] (0,-5) -- (0,10) node[above] {$x_2$};
    
    
    \draw[color=red, domain=-5:2.5, dashed] plot[id=obj] function{-2*x} 
        node[below] {$max\ z = 2x_1 + x_2$};
    \draw[color=red, domain=(-5+1.7):(2.5+1.7), dashed] plot[id=obj] function{17.0/5.0-2*x};
    \draw[color=red, domain=(-5+63.0/22.0):(2.5+63.0/22.0), dashed] plot[id=obj] function{63.0/11.0-2*x};
    \draw[color=red, domain=(-5+10.0/3.0):(2.5+10.0/3.0), dashed] plot[id=obj] function{20.0/3.0-2*x};
        
    \draw[color=blue, domain=-5:10] plot[id=c1] function{(17.0/5.0) - (2.0/5.0)*x} 
        node[right] {$2x_1 + 5x_2 \leq 17$};
    \draw[color=orange, domain=-10.0/3.0:20.0/3.0] plot[id=c2] function{(5.0)-(3.0/2.0)*x)} 
        node[right] {$3x_1+2x_2 \leq 10$};
   	
\end{tikzpicture}

Nous obtenons grâce à la résolution graphique : 
$$ \begin{cases}
x_1 = \frac{10}{3} \\
x_2 = 0 \\
z = \frac{20}{3}
\end{cases} $$


\subsection{Résolution par la méthode du simplexe}
Programme linéaire :
$$
PL_0 \begin{cases}
max\ z(x_1,x_2) = 2x_1 + x_2 \\
 2x_1 + 5x_2 \leq 17 \\
 3x_1 + 2x_2 \leq 10 \\
 x_1, x_2 \geq 0 \\
\end{cases}
$$

Forme standard :
$$
PL_0 \begin{cases}
max\ z(x_1,x_2) = 2x_1 + x_2 \\
 2x_1 + 5x_2 + y_1 = 17 \\
 3x_1 + 2x_2 +y_2 = 10 \\
 x_1, x_2, y_1, y_2 \geq 0 \\
\end{cases}
$$


Tableaux :

$$ \begin{array}{|C{1cm}|C{2cm}|C{1cm} C{1cm} C{1cm} C{1cm}|} \hline
	 &  & 2 & 1 & 0 & 0 \\ \hline
	0 & y_1 = 17 & 2 & 5 & 1 & 0 \\ 
	0 & y_2 = 10 & \textcolor{red}{3} & 2 & 0 & 1 \\ \hline
	 & z = 0 & -2 & -1 & 0 & 0 \\ \hline
 \end{array} $$
 
 $y_2$ sort de la base ; $x_1$ rentre dans la base ; le pivot devient $a_{1,2} = 3$.
 
 $$ \begin{array}{|C{1cm}|C{2cm}|C{1cm} C{1cm} C{1cm} C{1cm}|} \hline
	 &  & 2 & 1 & 0 & 0 \\ \hline
	0 & y_1 = \frac{31}{3} & 0 & \frac{11}{3} & 1 & -\frac{2}{3} \\ 
	2 & x_1 = \frac{10}{3} &1 &  \frac{2}{3} & 0 & \frac{1}{3} \\ \hline
	 & z = \frac{20}{3} & 0 & \frac{1}{3} & 0 & \frac{2}{3} \\ \hline
 \end{array} $$
 
 Tous les coûts réduits sont positifs ou nuls donc le simplexe est fini. La solution correspond bien à celle trouvée graphiquement :
 $$ \begin{cases}
x_1 = \frac{10}{3} \\
x_2 = 0 \\
z = \frac{20}{3}
\end{cases} $$


\subsection{Recherche d'une solution à valeur entière}

\subsubsection{Branch and bound}
Pour la méthode du Branch and bound, nous partons de la solution optimale réelle trouvée précédemment.

\begin{center}
\begin{tikzpicture}[level/.style={sibling distance=7cm,level distance=3.5cm,
			growth parent anchor=south}]
\node [circle,draw] (z){	$ \begin{cases}
						x_1 = \frac{10}{3} \\
						x_2 = 0 \\
					\end{cases}
					z = \frac{20}{3} $}
;
\end{tikzpicture}
\end{center}

Puis nous ajoutons la contrainte $x_1 \leq 3$ pour forcer $x_1$ à être entier.

\begin{center}
\begin{tikzpicture}[level/.style={sibling distance=7cm,level distance=1cm, growth parent anchor=south}]
\node [circle,draw] (z){	$ \begin{cases}
						x_1 = \frac{10}{3} \\
						x_2 = 0 \\
					\end{cases}
					z = \frac{20}{3} $}
child {node [end] (a) {}
	edge from parent 
	node[right] {$x_1 \leq 3$}
	}
child {node [] (a) {}
	edge from parent[draw=none]
	}
;
\end{tikzpicture}
\end{center}

Nous calculons alors les tableaux du simplexe avec la nouvelle contrainte.

$$ \begin{array}{|C{1cm}|C{2cm}|C{1cm} C{1cm} C{1cm} C{1cm} C{1cm}|} \hline
	 &  & 2 & 1 & 0 & 0 & 0 \\ \hline
	0 & y_1 = 17 & 2 & 5 & 1 & 0 & 0 \\ 
	0 & y_2 = 10 & 3 & 2 & 0 & 1 & 0 \\ 
	0 & y_3 = 3 & \textcolor{red}{1} & 0 & 0 & 0 & 1 \\ \hline
	 & z = 0 & -2 & -1 & 0 & 0 & 0 \\ \hline
 \end{array} $$
 
 $y_3$ sort de la base ; $x_1$ rentre dans la base ; le pivot devient $a_{1,3} = 1$.
 
 $$ \begin{array}{|C{1cm}|C{2cm}|C{1cm} C{1cm} C{1cm} C{1cm} C{1cm}|} \hline
	 &  & 2 & 1 & 0 & 0 & 0 \\ \hline
	0 & y_1 = 11 & 0 & 5 & 1 & 0 & -2 \\ 
	0 & y_2 = 1 & 0 & \textcolor{red}{2} & 0 & 1 & -3 \\ 
	2 & x_1 = 3 & 1 & 0 & 0 & 0 & 1 \\ \hline
	 & z = 6 & 0 & -1 & 0 & 0 & 2 \\ \hline
 \end{array} $$

$y_2$ sort de la base ; $x_2$ rentre dans la base ; le pivot devient $a_{2,2} = 2$.

$$ \begin{array}{|C{1cm}|C{2cm}|C{1cm} C{1cm} C{1cm} C{1cm} C{1cm}|} \hline
	 &  & 2 & 1 & 0 & 0 & 0 \\ \hline
	0 & y_1 = \frac{17}{2} & 0 & 0 & 1 & -\frac{5}{2} & 2 \\ 
	1 & x_2 = \frac{1}{2} & 0 & 1 & 0 & \frac{1}{2} & -\frac{3}{2} \\ 
	2 & x_1 = 3 & 1 & 0 & 0 & 0 & 1 \\ \hline
	 & z = \frac{13}{2} & 0 & 0 & 0 & \frac{1}{2} & \frac{1}{2} \\ \hline
 \end{array} $$

Le simplexe est fini, nous ajoutons le résultat à l'arbre de branch and cut.

\begin{center}
\begin{tikzpicture}[level/.style={sibling distance=7cm,level distance=2.5cm,
			growth parent anchor=south}]
\node [circle,draw] (z){	$ \begin{cases}
						x_1 = \frac{10}{3} \\
						x_2 = 0 \\
					\end{cases}
					z = \frac{20}{3} $}
child {node [circle,draw] (a) {	$ \begin{cases}
							x_1 = 3 \\
							x_2 = \frac{1}{2} \\
						\end{cases}
						z = \frac{13}{2} $}
	edge from parent 
	node[left] {$x_1 \leq 3$}
	}
child {node [] (a) {}
	edge from parent[draw=none]
	}
;
\end{tikzpicture}
\end{center}

$x_1$ a bien une valeur entière mais $x_2$ est devenu réel, nous ajoutons donc la contrainte $x_2 \leq 0$. Comme $x_2 \geq 0$ par définition, nous obtenons la solution suivante :

\begin{center}
\begin{tikzpicture}[level/.style={sibling distance=7cm,level distance=2.5cm,
			growth parent anchor=south}]
\node [circle,draw] (z){	$ \begin{cases}
						x_1 = \frac{10}{3} \\
						x_2 = 0 \\
					\end{cases}
					z = \frac{20}{3} $}
child {node [circle,draw] (a) {	$ \begin{cases}
							x_1 = 3 \\
							x_2 = \frac{1}{2} \\
						\end{cases}
						z = \frac{13}{2} $}
	child { node [circle,draw] (b) {$ \begin{cases}
								x_1 = 3 \\
								x_2 = 0 \\
							\end{cases}
							z = 6 $ }
		edge from parent 
		node[left] {$x_2 \leq 0$}
						}
	child {node [] (c) {}
		edge from parent[draw=none]
		}
	edge from parent 
	node[left] {$x_1 \leq 3$}
	}
child {node [] (d) {}
	edge from parent[draw=none]
	}
;
\end{tikzpicture}
\end{center}

Nous obtenons une feuille de l'arbre et une solution entière admissible. Nous devons maintenant développer le reste de l'arbre pour vérifier s'il est possible d'obtenir une meilleure solution entière. Posons donc la contrainte suivante : $x_2 \geq 1$.

\begin{center}
\begin{tikzpicture}[level/.style={sibling distance=7cm,level distance=2.5cm,
			growth parent anchor=south}]
\node [circle,draw] (z){	$ \begin{cases}
						x_1 = \frac{10}{3} \\
						x_2 = 0 \\
					\end{cases}
					z = \frac{20}{3} $}
child {node [circle,draw] (a) {	$ \begin{cases}
							x_1 = 3 \\
							x_2 = \frac{1}{2} \\
						\end{cases}
						z = \frac{13}{2} $}
	child { node [circle,draw] (b) {$ \begin{cases}
								x_1 = 3 \\
								x_2 = 0 \\
							\end{cases}
							z = 6 $ }
		edge from parent 
		node[left] {$x_2 \leq 0$}
						}
	child {node [end] (c) {}
		edge from parent 
		node[right] {$x_2 \geq 1$}
		}
	edge from parent 
	node[left] {$x_1 \leq 3$}
	}
child {node [] (d) {}
	edge from parent[draw=none]
	}
;
\end{tikzpicture}
\end{center}

Afin de trouver une solution, nous appliquons la phase 1 du simplexe dont voici les tableaux.

$$ \begin{array}{|C{1cm}|C{2cm}|C{1cm} C{1cm} C{1cm} C{1cm} C{1cm} C{1cm} C{1cm}|} \hline
	 &  & 0 & 0 & 0 & 0 & 0 & 0 & -1 \\ \hline
	0 & y_1 = 17 & 2 & 5 & 1 & 0 & 0 & 0 & 0 \\ 
	0 & y_2 = 10 & 3 & 2 & 0 & 1 & 0 & 0 & 0 \\ 
	0 & y_3 = 3 & 1 & 0 & 0 & 0 & 1 & 0 & 0 \\ 
	-1 & y_5 = 1 & 0 & \textcolor{red}{1} & 0 & 0 & 0 & -1 & 1 \\ \hline
	 & z = -1 & 0 & -1 & 0 & 0 & 0 & 1 & 0 \\ \hline
 \end{array} $$
 
 $y_5$ sort de la base ; $x_2$ rentre dans la base ; le pivot devient $a_{2,4} = 1$.
 
 $$ \begin{array}{|C{1cm}|C{2cm}|C{1cm} C{1cm} C{1cm} C{1cm} C{1cm} C{1cm} C{1cm}|} \hline
	 &  & 0 & 0 & 0 & 0 & 0 & 0 & -1 \\ \hline
	0 & y_1 = 12 & 2 & 0 & 1 & 0 & 0 & 5 & -5 \\ 
	0 & y_2 = 8 & 3 & 0 & 0 & 1 & 0 & 2 & -2 \\ 
	0 & y_3 = 3 & 1 & 0 & 0 & 0 & 1 & 0 & 0 \\ 
	0 & x_2 = 1 & 0 & 1 & 0 & 0 & 0 & -1 & 1 \\ \hline
	 & z = 0 & 0 & 0 & 0 & 0 & 0 & 0 & 1 \\ \hline
 \end{array} $$

$z = 0$ donc la phase $1$ du simplexe est finie. Passons à la phase $2$...

$$ \begin{array}{|C{1cm}|C{2cm}|C{1cm} C{1cm} C{1cm} C{1cm} C{1cm} C{1cm}|} \hline
	 &  & 2 & 1 & 0 & 0 & 0 & 0 \\ \hline
	0 & y_1 = 12 & 2 & 0 & 1 & 0 & 0 & 5 \\ 
	0 & y_2 = 8 & \textcolor{red}{3} & 0 & 0 & 1 & 0 & 2 \\ 
	0 & y_3 = 3 & 1 & 0 & 0 & 0 & 1 & 0 \\ 
	1 & x_2 = 1 & 0 & 1 & 0 & 0 & 0 & -1 \\ \hline
	 & z = 1 & -2 & 0 & 0 & 0 & 0 & -1 \\ \hline
 \end{array} $$
 
  $y_2$ sort de la base ; $x_1$ rentre dans la base ; le pivot devient $a_{1,2} = 3$.

$$ \begin{array}{|C{1cm}|C{2cm}|C{1cm} C{1cm} C{1cm} C{1cm} C{1cm} C{1cm}|} \hline
	 &  & 2 & 1 & 0 & 0 & 0 & 0 \\ \hline
	0 & y_1 = \frac{20}{3} & 0 & 0 & 1 & -\frac{2}{3} & 0 & \frac{11}{3} \\ 
	2 & x_1 = \frac{8}{3} &1 & 0 & 0 & \frac{1}{3} & 0 & \frac{2}{3} \\ 
	0 & y_3 = \frac{1}{3} & 0 & 0 & 0 & -\frac{1}{3} & 1 & -\frac{2}{3} \\ 
	1 & x_2 = 1 & 0 & 1 & 0 & 0 & 0 & -1 \\ \hline
	 & z = \frac{19}{3} & 0 & 0 & 0 & \frac{2}{3} & 0 & \frac{1}{3} \\ \hline
 \end{array} $$
 
Nous ajoutons la solution obtenue à l'arbre de branch and cut. Comme dans cette solution, $x_1$ n'est pas entier, nous ajoutons également la contrainte : $x_1 \leq 2$.

\begin{center}
\begin{tikzpicture}[level/.style={sibling distance=7cm,level distance=2.5cm,
			growth parent anchor=south}]
\node [circle,draw] (z){	$ \begin{cases}
						x_1 = \frac{10}{3} \\
						x_2 = 0 \\
					\end{cases}
					z = \frac{20}{3} $}
child {node [circle,draw] (a) {	$ \begin{cases}
							x_1 = 3 \\
							x_2 = \frac{1}{2} \\
						\end{cases}
						z = \frac{13}{2} $}
	child { node [circle,draw] (b) {$ \begin{cases}
								x_1 = 3 \\
								x_2 = 0 \\
							\end{cases}
							z = 6 $ }
		edge from parent 
		node[left] {$x_2 \leq 0$}
						}
	child {node [circle,draw] (c) {$ \begin{cases}
								x_1 = \frac{8}{3} \\
								x_2 = 1 \\
							\end{cases}
							z = \frac{19}{3} $ }
		child {node [end] (e) {}
			edge from parent
			node[left] {$x_1 \leq 2$}
			}
		edge from parent 
		node[right] {$x_2 \geq 1$}
		}
	edge from parent 
	node[left] {$x_1 \leq 3$}
	}
child {node [] (d) {}
	edge from parent[draw=none]
	}
;
\end{tikzpicture}
\end{center}

Nous lançons alors la résolution de ce nouveau problème par le simplexe.

$$ \begin{array}{|C{1cm}|C{2cm}|C{1cm} C{1cm} C{1cm} C{1cm} C{1cm} C{1cm}|} \hline
	 &  & 2 & 1 & 0 & 0 & 0 & 0 \\ \hline
	0 & y_1 = 12 & 2 & 0 & 1 & 0 & 0 & 5 \\ 
	0 & y_2 = 8 & 3 & 0 & 0 & 1 & 0 & 2 \\ 
	0 & y_3 = 2 & \textcolor{red}{1} & 0 & 0 & 0 & 1 & 0 \\ 
	1 & x_2 = 1 & 0 & 1 & 0 & 0 & 0 & -1 \\ \hline
	 & z = 1 & -2 & 0 & 0 & 0 & 0 & -1 \\ \hline
 \end{array} $$
 
$y_3$ sort de la base ; $x_1$ rentre dans la base ; le pivot devient $a_{1,3} = 1$.
 
$$ \begin{array}{|C{1cm}|C{2cm}|C{1cm} C{1cm} C{1cm} C{1cm} C{1cm} C{1cm}|} \hline
	 &  & 2 & 1 & 0 & 0 & 0 & 0 \\ \hline
	0 & y_1 = 8 & 0 & 0 & 1 & 0 & -2 & 5 \\ 
	0 & y_2 = 2 & 0 & 0 & 0 & 1 & -3 & \textcolor{red}{2} \\ 
	2 & x_1 = 2 & 1 & 0 & 0 & 0 & 1 & 0 \\ 
	1 & x_2 = 1 & 0 & 1 & 0 & 0 & 0 & -1 \\ \hline
	 & z = 3 & 0 & 0 & 0 & 0 & 2 & -1 \\ \hline
 \end{array} $$
 
$y_2$ sort de la base ; $y_4$ rentre dans la base ; le pivot devient $a_{6,2} = 2$.
 
$$ \begin{array}{|C{1cm}|C{2cm}|C{1cm} C{1cm} C{1cm} C{1cm} C{1cm} C{1cm}|} \hline
	 &  & 2 & 1 & 0 & 0 & 0 & 0 \\ \hline
	0 & y_1 = 3 & 0 & 0 & 1 & -\frac{5}{2} & \frac{11}{2} & 0 \\ 
	0 & y_4 = 1 & 0 & 0 & 0 & \frac{1}{2} & -\frac{3}{2} & 1 \\ 
	2 & x_1 = 2 & 1 & 0 & 0 & 0 & 1 & 0 \\ 
	1 & x_2 = 2 & 0 & 1 & 0 & \frac{1}{2} & -\frac{3}{2} & 0 \\ \hline
	 & z = 6 & 0 & 0 & 0 & \frac{1}{2} & \frac{1}{2} & 0 \\ \hline
 \end{array} $$

La résolution par le simplexe est finie et nous obtenons une solution entière égale à celle précédemment trouvée. Cette solution est donc une feuille de l'arbre ; nous remontons d'un niveau et ajoutons la contrainte $x_1 \geq 4$.

Nous remarquons que cette contrainte force $x_1$ à violer la deuxième contrainte du problème originel : $3x_1 + 2x_2 \leq 10$. En effet $3 \times 4 > 10$, cette contrainte ne mène donc à aucune nouvelle solution et l'arbre de branch and cut est terminé.

Nous avons donc trouvé deux solutions optimales entières au problème :\\
$ \begin{cases}
	x_1 = 2 \\
	x_2 = 2 \\
	z = 6
\end{cases} $
et
$ \begin{cases}
	x_1 = 3 \\
	x_2 = 0 \\
	z = 6
\end{cases}$

 
\begin{center}
\begin{tikzpicture}[level/.style={sibling distance=7cm,level distance=2.5cm,
			growth parent anchor=south}]
\node [circle,draw] (z){	$ \begin{cases}
						x_1 = \frac{10}{3} \\
						x_2 = 0 \\
					\end{cases}
					z = \frac{20}{3} $}
child {node [circle,draw] (a) {	$ \begin{cases}
							x_1 = 3 \\
							x_2 = \frac{1}{2} \\
						\end{cases}
						z = \frac{13}{2} $}
	child { node [circle,draw] (b) {$ \begin{cases}
								x_1 = 3 \\
								x_2 = 0 \\
							\end{cases}
							z = 6 $ }
		edge from parent 
		node[left] {$x_2 \leq 0$}
						}
	child {node [circle,draw] (c) {$ \begin{cases}
								x_1 = \frac{8}{3} \\
								x_2 = 1 \\
							\end{cases}
							z = \frac{19}{3} $ }
		child {node [circle,draw] (e)  {$ \begin{cases}
									x_1 = 2 \\
									x_2 = 2 \\
								\end{cases}
								z = 6 $ }
			edge from parent
			node[left] {$x_1 \leq 2$}
			}
		edge from parent 
		node[right] {$x_2 \geq 1$}
		}
	edge from parent 
	node[left] {$x_1 \leq 3$}
	}
child {node [end] (d) {}
	edge from parent
	node[left] {$x_1 \geq 4$}
	}
;
\end{tikzpicture}
\end{center}


\subsubsection{Coupes de Gomory}

La méthode des coupes de Gomory nécessite une solution optimale réelle, nous partons donc de celle trouvée précédemment :
$$ \begin{cases}
x_1 = \frac{10}{3} \\
x_2 = 0 \\
z = \frac{20}{3}
\end{cases} $$

et du dernier tableau du simplexe associé :

$$ \begin{array}{|C{1cm}|C{2cm}|C{1cm} C{1cm} C{1cm} C{1cm}|} \hline
	 &  & 2 & 1 & 0 & 0 \\ \hline
	0 & y_1 = \frac{31}{3} & 0 & \frac{11}{3} & 1 & -\frac{2}{3} \\ 
	2 & x_1 = \frac{10}{3} &1 &  \frac{2}{3} & 0 & \frac{1}{3} \\ \hline
	 & z = \frac{20}{3} & 0 & \frac{1}{3} & 0 & \frac{2}{3} \\ \hline
 \end{array} $$

Nous avons choisi la deuxième ligne du tableau pour en déduire la contrainte de Gomory suivante :
 $$ <\frac{2}{3}>x_2 + <\frac{1}{3}>y_2 \geq <\frac{10}{3}> $$
 $$ \Leftrightarrow $$
 $$ \frac{2}{3}x_2 + \frac{1}{3}y_2 \geq \frac{1}{3} $$
Nous devons maintenant l'exprimer uniquement en fonction de $x_1$ et $x_2$. Remplaçons donc $y_2$ par son équivalent dans la contrainte 2, cela donne :
$$ x_1 \leq 3 $$
Nous devons maintenant utiliser le simplexe pour résoudre le problème avec la nouvelle contrainte. L'ayant déjà fait lors de la question précédente, sautons directement au résultat :

$ \begin{cases}
	x_1 = 3 \\
	x_2 = \frac{1}{2} \\
	z = \frac{13}{2}
\end{cases} $

et le dernier tableau associé :

$$ \begin{array}{|C{1cm}|C{2cm}|C{1cm} C{1cm} C{1cm} C{1cm} C{1cm}|} \hline
	 &  & 2 & 1 & 0 & 0 & 0 \\ \hline
	0 & y_1 = \frac{17}{2} & 0 & 0 & 1 & -\frac{5}{2} & 2 \\ 
	1 & x_2 = \frac{1}{2} & 0 & 1 & 0 & \frac{1}{2} & -\frac{3}{2} \\ 
	2 & x_1 = 3 & 1 & 0 & 0 & 0 & 1 \\ \hline
	 & z = \frac{13}{2} & 0 & 0 & 0 & \frac{1}{2} & \frac{1}{2} \\ \hline
 \end{array} $$

Nous choisissons la ligne 2 du tableau pour exprimer la contrainte de Gomory :
 $$ <\frac{1}{2}>y_2 + <-\frac{3}{2}>y_3 \geq <\frac{1}{2}> $$
 $$ \Leftrightarrow $$
 $$ \frac{1}{2}y_2 + \frac{1}{2}y_3 \geq \frac{1}{2} $$
Nous devons maintenant l'exprimer uniquement en fonction de $x_1$ et $x_2$. Remplaçons donc $y_2$ par son équivalent dans la contrainte 2, cela donne :
$$ x_2 \leq 2y_3 $$
Remplaçons maintenant $y_3$ par son équivalent dans la contrainte 3, cela donne :
$$ 2x_1 + x_2 \leq 6 $$

Résolvons le problème avec cette contrainte en plus :
$$ \begin{array}{|C{1cm}|C{2cm}|C{1cm} C{1cm} C{1cm} C{1cm} C{1cm} C{1cm}|} \hline
	 &  & 2 & 1 & 0 & 0 & 0 & 0 \\ \hline
	0 & y_1 = 17 & 2 & 5 & 1 & 0 & 0 & 0 \\ 
	0 & y_2 = 10 & 3 & 2 & 0 & 1 & 0 & 0 \\ 
	0 & y_3 = 3 & 1 & 0 & 0 & 0 & 1 & 0 \\ 
	0 & y_4 = 6 & \textcolor{red}{2} & 1 & 0 & 0 & 0 & 1 \\ \hline
	 & z = 0 & -2 & -1 & 0 & 0 & 0 & 0 \\ \hline
 \end{array} $$
 
 $y_4$ sort de la base ; $x_1$ rentre dans la base ; le pivot devient $a_{1,4} = 2$.
 
 $$ \begin{array}{|C{1cm}|C{2cm}|C{1cm} C{1cm} C{1cm} C{1cm} C{1cm} C{1cm}|} \hline
	 &  & 2 & 1 & 0 & 0 & 0 & 0 \\ \hline
	0 & y_1 = 11 & 0 & 4 & 1 & 0 & 0 & -1 \\ 
	0 & y_2 = 1 & 0 & \frac{1}{2} & 0 & 1 & 0 & -\frac{3}{2} \\ 
	0 & y_3 = 0 & 0 & -\frac{1}{2} & 0 & 0 & 1 & -\frac{1}{2}  \\ 
	2 & x_1 = 3 & 1 & \frac{1}{2} & 0 & 0 & 0 & \frac{1}{2} \\ \hline
	 & z = 6 & 0 & 0 & 0 & 0 & 0 & 1 \\ \hline
 \end{array} $$
 
Et voici donc la solution finale de la méthode des coupes de Gomory :
$ \begin{cases}
	x_1 = 3 \\
	x_2 = 0 \\
	z = 6
\end{cases} $.





\chapter{Partie pratique}
% !TEX encoding = UTF-8 Unicode
% !TEX root = ../Rapport/rapport.tex

\section{Programmation dynamique}

% !TEX encoding = UTF-8 Unicode
% !TEX root = ../Rapport/rapport.tex

\section{Branch and bound}


\vfill
{\raggedleft Réalisé avec \LaTeX{} \par}

\end{document}
