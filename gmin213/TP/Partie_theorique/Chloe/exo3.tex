% !TEX encoding = UTF-8 Unicode
% !TEX root = ../Rapport/rapport.tex

\section{Sur le problème du couplage maximum de poids minimum}

\subsection{Modélisation du problème}
Soit un graphe $G = (V,E)$ avec $V$ l'ensemble de ses sommets et $E$ l'ensemble de ses arêtes. Soit $\{ \forall{(i,j) \in E}, X_{(i,j)}\}$ un ensemble de variables booléennes qui indiquent le choix de l'arête $(i,j)$ correspondante dans le couplage. Soit $P_{(i,j)}$ le poids de l'arête $(i,j)$.

La première modélisation intuitive est la suivante :

$ Maximiser $ $$ \sum\limits_{(i,j) \in E} (X_{(i,j)} \times H) - \sum\limits_{(i,j) \in E} (X_{(i,j)} \times P_{(i,j)}) $$
$ Sous\ contraintes $
$$ \forall{i \in V}, \sum\limits_{(i,j) \in E} X_{(i,j)} + \sum\limits_{(j,i) \in E} X_{(j,i)} \leq 1 $$
$$ \forall{(i,j) \in E}, X_{(i,j))} \in \{0,1\} $$
$$ \forall{(i,j) \in E}, P_{(i,j))} \geq 0 $$
avec H une constante très grande. Cependant, après quelques instants de réflexion, nous pouvons imaginer une modélisation plus élégante :

$ Minimiser $ $$ \sum\limits_{(i,j) \in E} (X_{(i,j)} \times P_{(i,j)}) $$
$ Sous\ contraintes $
$$ \forall{i \in V}, \sum\limits_{(i,j) \in E} X_{(i,j)} + \sum\limits_{(j,i) \in E} X_{(j,i)} = 1 $$
$$ \forall{(i,j) \in E}, X_{(i,j))} \in \{0,1\} $$
$$ \forall{(i,j) \in E}, P_{(i,j))} \geq 0 $$

En effet, les contraintes forcent le couplage à être maximum tandis que la fonction objectif le force à tendre vers le poids minimum.

\subsection{Modélisation du problème appliquée au graphe de la figure 2}
$ Minimiser $ $$ \epsilon \times X_{ab} + \epsilon \times X_{bc} + \epsilon \times X_{ac} + M \times X_{ae} + M \times X_{cd} + M \times X_{bf} + \epsilon \times X_{df} + \epsilon \times X_{de} + \epsilon \times X_{fe}$$
$ Sous\ contraintes $
$$ X_{ab} + X_{ac} + X_{ae} = 1 $$
$$ X_{ab} + X_{bc} + X_{bf} = 1 $$
$$ X_{ac} + X_{bc} + X_{cd} = 1 $$
$$ X_{cd} + X_{de} + X_{df} = 1 $$
$$ X_{ae} + X_{de} + X_{df} = 1 $$
$$ X_{ef} + X_{df} + X_{bf} = 1 $$
$$ \forall{(i,j) \in E}, X_{(i,j))} \in \{0,1\} $$
$$ \epsilon \geq 0 $$
$$ M \geq 0 $$

\subsection{Solution optimale entière $z(ILP)$}
Sur un exemple de cette taille, il est facile de trouver une solution à la main. Il y a plusieurs solutions optimales de poids total $M+2\epsilon$ sur cet exemple ; l'une d'entre elles est le couplage $\{(a,b), (c,d), (e,f)\}$.

La résolution de ce PLNE par $glpsol$ (solveur de $GLPK$) donne bien la même solution.

\subsection{Solution optimale $z(LP)$ pour le programme relaxé}
Relaxer le programme revient à transformer la contrainte d'intégrité des $X_{(i,j)}$ en la contrainte suivante :
$$ \forall{(i,j) \in E},  0 \leq X_{(i,j))} \leq 1 $$

En rentrant le PL relaxé dans $glpsol$, nous obtenons le couplage de poids total $3\epsilon$ : $$\{X_{ab} = 0.5 ; X_{ac} = 0.5 ; X_{bc} = 0.5 ; X_{df} = 0.5 ; X_{ef} = 0.5 ; X_{de} = 0.5\}$$ 

\subsection{Conclusion sur la pertinence de la formulation}
La solution trouvée au PLNE étant optimale, la formulation du problème semble être pertinente.


