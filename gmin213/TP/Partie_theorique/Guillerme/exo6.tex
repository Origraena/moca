% !TEX root = ../../Rapport/rapport.tex
\section{Sur le problème du sac à dos simple}

\subsection{Construction d'un algorithme approché}

\subsubsection{(a) Complexité de l'algorithme}

La complexité de l'algorithme, une fois les nombres triés, est en $O(n)$. Cependant, il est possible
de démontrer que la complexité des algorithmes de tris basés sur des fonctions de comparaisons ne
peut être inférieures à $O(n\log n)$. la complexité de cet algorithme est donc similaire à la
compléxité de l'algorithme de tri utilisé.

\subsubsection{(b) Minoration de $cost(T)$}

Dans un premier temps nous montrerons que l'existence d'un indice $j$ pour lequel $cost(T) + w_{j+1}
> b$ n'est pas systématique, mais que si ce dernier n'existe pas, le problème n'existe pas non plus.

En effet, si cet indice n'existe pas, on a alors : 
$$ \sum_{i=1}^n w_1 \leq b $$
On en déduit donc que la solution optimale est donnée par $ T = \{ 1, \dot, n \} $. Ce qui en soit
même n'a aucun intérêt\footnote{Le problème du sac à dos, dans cette configuration n'est plus
NP-Complet}. Nous admettrons donc l'existence de cet indice $j$.

Supposons, à l'indice $j$, que $cost(T) \leq \frac{b}{2}$. Par définition de l'indice $j$, on a
$cost(T) + w_{j+1} > b$, ce qui implique $w_{j+1} > \frac{b}{2}$ et donc : $$
\begin{array}{lrcl}
	& w_{j+1} & > & cost(T) \\
	\Rightarrow & w_{j+1} & > & \sum_{i=1}^j w_i \\
	\Rightarrow & w_{j+1} & > & w_i
\end{array}
$$

Or les éléments $w_i$ étant triés, on a $w_j \geq w_{j+1}$. L'hypothèse de départ est donc fausse.

\subsubsection{(c) Performance relative de l'algorithme}

Notons $OPT$ la solution optimale au problème du sac à dos, et $A$ la solution approchée donnée par
l'algorithme. Rappelons, par définition du problème, que $OPT \leq b $. La performance relative nous
est donnée par : $$
\frac{OPT}{A} \leq \frac{b}{\frac{b}{2}} = 2 $$

La performance relative de l'algorithme est donc 2.

\subsection{Construction d'un schéma d'approximation}

\subsubsection{(a) Complexité de l'algorithme}

De façon très grossière, la construction de l'ensemble des sous-ensembles $S_k$ de $k$ éléments se
fait en $k\times n^k$ opérations. On a donc une construction des sous-ensembles réalisée en : 
$$\sum_{i=1}^k i . n^i $$
Soit une construction en $O(n^k)$.

L'ensemble étant déjà trié, l'algorithme glouton s'exécute en $O(n)$, d'où une exécution totale de
l'algorithme en $O(n^{k+1})$.


\subsubsection{(b) Comparaison à l'optimal}
% TODO

