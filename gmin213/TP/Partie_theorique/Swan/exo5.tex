% !TEX root = ../../Rapport/rapport.tex

\section{Sur le problème de partition}\label{ex5}

Soit un ensemble d'objets $A = \{a_i : i \in [1;n]\}$ et une fonction de poids $p : A
\rightarrow \mathbb{N}$.\\
Le problème de décision est le suivant : \\
existe-t-il $Y_1,Y_2 \subseteq A\ Y_1 \cap Y_2 = \emptyset: \sum_{y_{1} \in Y_1}p(y_1) = \sum_{y_2 
\in Y_2}p(y_2) = P$ ?\\
Son problème d'optimisation associé est défini comme suit :\\
$minimiser\ \sum_{y_1 \in Y_1}p(y_1) : \sum_{y_1 \in Y_1}p(y_1) - \sum_{y_2 \in Y_2}p(y_2) 
\geq 0$.

On note :
\begin{itemize}
	\item $w(E) = \sum_{e \in E}p(e)$ avec $E \subseteq A$,
	\item $L = \frac{w(A)}{2}$.
\end{itemize}
~\\
Soit l'algorithme suivant :

\begin{center}
\begin{algorithm}[H]
\caption{PTAS Partition}\label{ex5_algo}
\algsetup{indent=2em,linenodelimiter= }
\begin{algorithmic}[1]
\REQUIRE $A = \{a_i : i \in [1,n]\}$ un ensemble de n objets,
		 $p : A \rightarrow \mathbb{N}$ une fonction de poids sur ceux-ci,
		 $r > 1$ le rapport d'approximation désiré
	\IF {$r \geq 2$}
		\RETURN $A,\emptyset$
	\ENDIF
	\STATE Trier $A$ par ordre décroissant de $p$
	\STATE Soit $(a_1,...,a_n)$
	\STATE $k(r) \leftarrow \lceil\frac{2-r}{r-1}\rceil$
	\STATE {\em Première phase}
	\STATE Trouver une partition optimale $Y_1,Y_2$ de $a_1,...,a_{k(r)}$
	\FOR {$j = k(r)+1$ à $n$}
		\IF {$\sum_{y_1 \in Y_1}p(y_1) \leq \sum_{y_2 \in Y_2}p(y_2)$}
			\STATE $Y_1 \leftarrow Y_1 \cup \{a_j\}$
		\ELSE
			\STATE $Y_2 \leftarrow Y_2 \cup \{a_j\}$
		\ENDIF
	\ENDFOR
	\RETURN $Y_1,Y_2$
\end{algorithmic}
\end{algorithm}
\end{center}

\subsection{r-approximation ; cas où $r \geq 2$}\label{ex5_q1}
Montrons que $\forall r \geq 2$, la solution $A,\emptyset$ est une r-approximation.\\
On sait que : $w(A) = w(Y_1) + w(Y_2)$ donc $w(Y_1) \geq \frac{w(A)}{2}$. Ie,
$sol_{opt}(A) \geq \frac{w(A)}{2}$.
La valeur de la solution $A,\emptyset$ $sol_{algo}(A)$ vaut $w(Y_1) = w(A)$.
On a alors $w(Y_1) = 2 \times \frac{w_(A)}{2}$ et donc $w(Y_1) \leq 2 \times
sol_{opt}(A)$.\\
Nous avons bien $sol_{algo}(A) \leq 2 \times sol_{opt}(A) \leq r \times sol_{opt}(A)$.

\subsection{Cas où $r < 2$}\label{ex5_q2}
Soit $a_h$ le dernier \'el\'ement ins\'er\'e dans $Y_1$.\\

\subsubsection{(a)}\label{ex5_q2_a}
Nous devons montrer que $w(Y_1) - L \leq \frac{p(a_h)}{2}$.\\
Posons $w(Y_1)'$ et $w(Y_2)'$ les poids respectifs deux partitions avant l'ajout de
$a_h$.\\
Puisque $a_h$ est ajout\'e \`a $Y_1$, on sait que $w(Y_1)' \leq w(Y_2)'$.
De plus, $a_h$ est le dernier \'el\'ement ajout\'e \`a $Y_1$, donc $w(Y_1) = w(Y_1)' +
p(a_h)$.
Cela implique aussi que tous les \'el\'ements suivants sont ajoutés dans $Y_2$, donc
$w(Y_2)' \leq w(Y_2)$.\\
On a donc $w(Y_1)' = w(Y_1) - p(a_h) \leq w(Y_2)' \leq w(Y_2)$.\\
$w(Y_1) - w(Y_2) \leq p(a_h) \leftrightarrow$
$w(Y_1) + w(A) - w(Y_2) - w(A) \leq p(a_h)$\\
$\leftrightarrow$
$2 \times w(Y_1) - w(A) \leq p(a_h) \leftrightarrow$
$w(Y_1) - \frac{w(A)}{2} \leq \frac{p(a_h)}{2}$\\
$\leftrightarrow$
$w(Y_1) - L \leq \frac{p(a_h)}{2}$.

\subsubsection{(b)}\label{ex5_q2_b}
Si $a_h$ est ins\'er\'e durant la première phase, l'algorithme renverra la solution
optimale. En effet $a_h$ est le dernier élément de $Y_1$, donc tous les suivants seront
insérés dans $Y_2$, de plus $w(Y_1) \geq w(Y_2)$.
Or, $a_h$ est inséré dans la première phase, c'est à dire dans la phase de recherche
optimale. On note $Y_1',Y_2'$ les ensembles à la fin de celle-ci.
On sait donc que $w(Y_1)'$ est la solution optimale du sous problème $Y_1' \cup
Y_2'$.
On a alors $w(Y_1) = w(Y_1)' = sol_{opt}(A')$ avec $A' = Y_1' \cup Y_2'$, puisque tous
les éléments suivants sont insérés dans $Y_2$, $sol_{opt}(A') \leq sol_{opt}(A)$.\\
On a donc $w(Y_1) \leq sol_{opt}(A)$ et $w(Y_1) \geq sol_{opt}(A)$ par définition :
$sol_{algo}(A) = sol_{opt}(A)$.

\subsubsection{(c)}\label{ex5_q2_c}
Supposons maintenant que $a_h$ est inséré durant la seconde phase, et intéressons nous aux 
$p(a_j)\ \forall j \in [1;k(r)]$. 
Puisque $a_h$ est inséré dans la seconde phase ($h \geq k(r)+1$), et que les $a_i$ sont
rangés par ordre décroissant de poids, on a : $p(a_h) \leq p(a_j)\ \forall j \in [1;k(r)+1]$.\\
Ce qui implique : $\sum_{j = 1}^{k(r)+1}p(a_h) \leq \sum_{j = 1}^{k(r)+1} p(a_j)$.\\
Or, $\sum_{j=1}^{k(r)+1}p(a_j) \leq w(A)$. \\
On a donc $(k(r)+1) \times p(a_h) \leq w(A) = 2 \times L$.

\subsubsection{(d)}\label{ex5_q2_d}
Toute solution optimale est minorée par $\frac{w(A)}{2}$.

\subsubsection{(e)}\label{ex5_q2_e}
Nous allons montrer que le ratio est bien majorée par $r$.\\
D'après la question \ref{ex5_q2_a}, nous savons que $w(Y_1) - L \leq \frac{p(a_h)}{2}$.
Donc $2 \times w(Y_1) - 2 \times L \leq p(a_h)$.
Or, par \ref{ex5_q2_c}, nous avons : $p(a_h) \leq \frac{2\times L}{k(r)+1}$.\\
On obtient alors $2 \times w(Y_1) - 2 \times L \leq 2 \times (r - 1) \times L$
$\leftrightarrow w(Y_1) \leq r \times L$.\\
On a bien : $sol_{algo}(A) \leq r \times sol_{opt}(A)$.

Nous pouvons en conclure que l'algorithme \ref{ex5_algo} renvoit bien une solution
r-approchée.

\subsection{Complexité}
% TODO
La complexité de l'algorithme est : $n.log(n) + 2^{k(r)+1} + n$

