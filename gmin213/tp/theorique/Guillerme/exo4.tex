\section{Sur le problème de la coupe maximum}

\subsection{Complexité}

A chaque itération de l'algorithme, la valeur de la coupe maximale augmente au minimum d'une unité.
Or la valeur de la coupe maximale étant bornée par le nombre d'arêtes, on obtient donc que
l'algorithme effectue au plus $m$ opérations. On a donc un algorithme en $O(m)$.

\subsection{Un algorithme 2-approché}

Considérons $(Y_1, Y_2)$ la coupe renvoyée par l'algorithme, nous chercherons dans un premier temps
à montrer que chaque sommet dans $Y_1$ admet au moins autant d'arêtes dans $Y_2$ que dans $Y_1$.

Pour ce faire, considérons un sommet $v \in Y_1$, supposons que ce sommet possède plus d'arêtes dans
$Y_1$ que dans $Y_2$, nous noterons $a_1$ le nombre d'arêtes incidentes à $v$ dans $Y_1$ et $a_2$ le
nombre d'arêtes adjacentes à $v$ dans $Y_2$. Déplacer $v$ de $Y_1$ vers $Y_2$ reviendrait à diminuer
la coupe de $a_2$ et augmenter celle-ci de $a_1$, or d'après l'hypothèse de départ $a_2 < a_1$, on
observerait une augmentation de la valeur de la coupe, ce qui est impossible si $(Y_1, Y_2)$ est une
coupe retournée par l'algorithme. On en déduit donc que l'hypothèse de départ est fausse.

Autrement dit, pour un sommet $v \in V$, en notant $d_v$ le degré de $v$, $v_c$ le nombre d'arêtes
adjacentes à $v$ traversant la coupe et $v_s$ le nombre d'arêtes adjacentes à $v$ ne la traversant
pas, on peut écrire : $$
\begin{array}{lrcl}
	&v_c + v_s &=& d_v \quad \mbox{or on a } v_c \geq v_s \\
	\Rightarrow & v_c & \geq & \frac{d_v}{2}
\end{array} $$

Si l'on généralise sur l'ensemble du graphe, on peut en déduire : $$
\begin{array}{rcl}
	|(Y_1,Y_2)| & = & \frac{1}{2} \sum_{v \in V} v_c\\
							& \geq & \frac{1}{2} \sum_{v \in V} \frac{d_v}{2} \\
					 	& = & \frac{m}{2}
\end{array}
$$

On a donc : $|(Y_1,Y_2)| \geq \frac{m}{2}$, de plus comme vu précédemment, la valeur de la coupe
maximale (que nous noterons $OPT$) est bornée par le nombre d'arêtes. On peut donc en déduire : $$
\frac{OPT}{|(Y_1,Y_2)|} \leq \frac{m}{\frac{m}{2}} = 2 $$

L'algorithme donné est donc bien un algorithme 2-approché.

\subsection{Atteindre la borne}
%TODO
