\section{Sur le problème de la couverture d'ensembles}

\subsection{Modélisation du problème à l'aide de la PLNE}

On considère le problème de la couverture d'ensemble, défini par :
Soit $E = \lbrace e_1, \dots, e_n \rbrace$, soient $S_1, \dots, S_m$ des sous-ensembles non vides
de $E$ tels que $\forall i \in \{1, \dots, m\}$, on a : $S_i \subset E$. On associe à chaque
ensemble $S_j$ un poids $w_j \geq 0$. Le problème consiste à trouver une collection de sous-ensemble
de poids minimum et telle que $\bigcup_{i=1}^m S_i = E$.

Ce problème peut s'exprimer à l'aide de la Programmation Linéaire en Nombres entiers de la manière
suivante : $$
\left \lbrace
\begin{array}{l}
	min z = \sum_{i=1}^m w_ix_i \\
	\sum_{j / e_i \in S_j} x_j \geq 1 \quad \forall e_i \in E \\
	x_i \in \{0, 1\}
\end{array} \right.
$$

\subsection{Procédure d'arrondis}

Soit $f = \max_{i=1,\dots,n} f_i$, avec $f_i$ le cardinal de l'ensemble des sous-ensembles de $E$
contenant $e_i$ : $ f_i = |\{j : e \in S_j\}|$.

Définissons la procédure d'arrondis suivante: $$
x_i = \left \{ \begin{array}{rcl}
		1 & \mbox{ si } & x_i \geq \frac{1}{f} \\
		0 & \mbox{ sinon } &
	\end{array} \right .
	$$

Cherchons à démontrer que cette procédure d'arrondis garantit une solution réalisable. Pour ce
faire, considérons l'ensemble $C$ des sous-ensembles sélectionnés par la procédure d'arrondis : $$ C
= \{ S_i : x_i \geq \frac{1}{f} \} $$

Soir un élément quelconque $c$, par définition de $f$, $c$ appartient à, au plus $f$ sous-ensembles.
De plus, par définition du problème, on a : $$
\sum_{j : e \in S_j} x_j \geq 1 $$

On a donc au moins un des $x_j \geq \frac{1}{f}$, et donc $c$ est couvert par la solution. Cette
procédure d'arrondis garantit une solution réalisable.

\subsection{Existence d'un algorithme $f$-approché}

Considérons l'algorithme \ref{algo_pourri}
\begin{algorithm}
	\caption{Approximation couverture par ensemble}
	\label{algo_pourri}
	\begin{algorithmic}[1]
		\STATE Exprimer le problème en programmation linéaire en nombres entiers
		\STATE Résoudre la version relaxée
		\STATE Réaliser la procédure d'arrondis
	\end{algorithmic}
\end{algorithm}

En utilisant la procédure d'arrondis étudiée plus haut, on sait que pour chaque clause, il existe
$x_k$ tel que $x_k \geq \frac{1}{f}$, au pire des cas \footnote{cas similaire à celui de l'exercice
1}, chaque $x_k$ est multiplié par $f$, ce qui implique que la solution obtenue est au plus $f$ fois
plus grande que la solution optimale. Il s'agit donc d'un algorithme f-approché.

\subsection{Cas ou $f=2$}

Si $f=2$ alors le problème devient une couverture minimum par sommets.
