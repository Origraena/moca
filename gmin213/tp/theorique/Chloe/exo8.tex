% !TEX root = ../../Rapport/rapport.tex
% !TEX encoding = UTF-8 Unicode


\section{Sur le produit matriciel}

\subsection{Nombre d'opérations nécessaires pour un produit}
Soit $M_i$ une $(p_i,p_{i+1})$-matrice. Soient $\pi _1$ et $\pi _2$ les produits matriciels à parenthésage symétrique suivants : 
$$\pi _1 = (M_1(M_2(M_3(M_4(...(M_{k-1}.M_k))))))$$
$$\pi _2 = ((((((M_1.M_2)M_3)M_4)...)M_{k-1})M_k)$$
Nous admettons que le produit d'une $(p,q)$-matrice et d'une $(q,r)$-matrice peut se faire à l'aide de $p.q.r$ opérations. Tous les produits matriciels envisagés par la suite sont définis car le nombre de colonnes de la première matrice est égal au nombre de lignes de la seconde.

\subsubsection{Nombre d'opérations nécessaires au produit $\pi _1$}
Évaluons le nombre d'opérations nécessaires au produit $\pi _1$ grâce au tableau suivant :
\begin{center}
\resizebox{\textwidth}{!}{
$ \begin{array}{|c|c|c|c|c|c|c|} \hline
	\pi _l & \pi _{1,1} & … & \pi _{1,4} & … & \pi _{1,k-2} & \pi _{1,k-1} \\ \hline
	=  & M_1 \times & … & M_{4} \times  & … & M_{k-2} \times & (M_{k-1} \times M_k) \\
	Nb.\ d'operations & 1.2.(k+1) & … & 4.5.(k+1) & … & (k-2).(k-1).(k+1) & (k-1).k.(k+1) \\
	Taille\ de\ la\ matrice & (1,k+1) & … & (4,k+1) & … & (k-2,k+1) & (k-1,k+1) \\ \hline
 \end{array} $
 }
 \end{center}
 
Nous pouvons alors déduire la formule donnant le nombre d'opérations nécessaires pour calculer le produit $\pi _1$ :
$$ \sum\limits_{i =1}^{i < k} {[i\times (i+1) \times (k+1)]} $$
Décomposons cette somme :
$$ (k+1) \times \left[ \sum\limits_{i =1}^{i < k} {i^2} + \sum\limits_{i =1}^{i < k} {i} \right] $$
En remplaçant les sommes par leur valeur :
$$ (k+1) \times \left(\frac{k.(k-1).(2k-1)}{6} + \frac{k.(k-1)}{2}\right) $$
En développant :
$$ (k+1) \times \left(\frac{2k^3-3k^2+k}{6} + \frac{k^2-k}{2}\right) $$
En réduisant :
$$ \frac{2k^4+2k^3-2k^2-2k}{6} $$
Donc le nombre d’opérations nécessaires pour calculer le produit $\pi_1$ est en $O(k^4)$.



\subsubsection{Nombre d'opérations nécessaires au produit $\pi _2$}
De la même façon, évaluons le nombre d'opérations nécessaires au produit $\pi _2$ grâce au tableau suivant :
\begin{center}
$ \begin{array}{|c|c|c|c|c|c|c|} \hline
	\pi _l & \pi _{2,1} & \pi _{2,2} & \pi _{2,3} & … & \pi _{2,k-2} & \pi _{2,k-1} \\ \hline
	=  & M_1 \times M_2 & \times M_3 & \times M_{4}  & … & \times M_{k-1} & \times M_k \\
	Nb.\ d'operations & 1.2.3 & 1.3.4 & 1.4.5 & … & 1.(k-1).k & 1.k.(k+1) \\
	Taille\ de\ la\ matrice & (1,3) & (1,4) & (1,5) & … & (1,k) & (1,k+1) \\ \hline
 \end{array} $
 \end{center}
 
 Nous pouvons alors déduire la formule donnant le nombre d'opérations nécessaires pour calculer le produit $\pi _2$ :
$$ \sum\limits_{i =1}^{i < k} {[(i+1) \times (i+2)]} $$
Décomposons cette somme :
$$ 2k-2 + \sum\limits_{i =1}^{i < k} {i^2} + 3 \times \sum\limits_{i =1}^{i < k} {i} $$
En remplaçant les sommes par leur valeur :
$$ 2k-2 + \frac{k.(k-1).(2k-1)}{6} + 3 \times \frac{k.(k-1)}{2} $$
En développant :
$$ \frac{12k-12}{6} + \frac{2k^3-3k^2+k}{6} + \frac{9k^2-9k}{6} $$
En réduisant :
$$ \frac{2k^3+6k^2+4k-12}{6} $$
Donc le nombre d’opérations nécessaires pour calculer le produit $\pi_2$ est en $O(k^3)$.


\subsubsection{Comparaison}
Comparons maintenant le nombre d'opérations nécessaires aux produits $\pi_1$ et $\pi_2$ :

\begin{center}
$ \begin{array}{|c|C{4.5cm}|C{4.5cm}|C{4.5cm}|} \hline
	 &  \pi_1 & vs &   \pi_2 \\ \hline
	k=2 & 6 & = & 6 \\
	k=3 & 32 & > & 18 \\ 
	k=4 & 100 & > & 38 \\ \hline
 \end{array} $
 \end{center}

Le premier critère de comparaison est le fait que le nombre d'opérations nécessaires pour calculer le produit $\pi_1$ est en $O(k^4)$ alors que le nombre d'opérations nécessaires pour calculer le produit $\pi_2$ est en $O(k^3)$. De façon évidence, $\pi_2$ est plus intéressant car il coûte $k$ fois moins d'opérations.

D'autre part, nous déduisons du tableau précédent que pour deux matrices, le nombre d'opérations est le même (en effet, il n'y a qu'une seule manière de multiplier deux matrices). Par contre pour tout nombre de matrices strictement supérieur à 2, $\pi_2$ est plus intéressant.

Le nombre d'opérations nécessaires pour calculer un produit matriciel dépendant donc du parenthésage choisi.

 
 
\subsection{Nombre de parenthésages possibles d'un produit de $k$ matrices}
Montrons que le nombre de parenthésages possibles est le suivant : $c(k)=\sum\limits_{i=1}^{k-1}{c(i)c(k-i)}$. Posons $c(1)=1$.

Pour $n=2$, il n'y a qu'une façon de parenthéser le produit de deux matrices donc $c(2)=1$. Or, $c(1).c(2-1) = 1 \times 1 = 1$. Donc la récurrence est initialisée.

Nous admettrons que $\forall k \in \mathbb{N}$ on a : $c(k)=\sum\limits_{i=1}^{k-1}{c(i)c(k-i)}$.

Pour $k = n+1$, on a les mêmes façons de parenthéser le produit que pour $k=n$ plus .

%\subsection{Récurrence}
%TODO

%\subsection{Exemple}
