% !TEX root = ../../Rapport/rapport.tex
\section{Seuil d'approximation pour le problème de la coloration de sommets (resp.
d'arêtes)}\label{ex11}

Soit un graphe quelconque $G = (V,E)$.

\subsubsection{Coloration de sommets}\label{ex11_def_sommets}
Le problème de décision est le suivant :\\
soit $k \in \mathbb{N}$, est-il possible de trouver une fonction $\chi : V \rightarrow
[1,k]$ telle que $\forall (x,y) \in E,\ \chi(x) \neq \chi(y)$.\\
Son problème d'optimisation associé est de trouver $k$ le plus petit possible tel que $G$
soit k-colorable (on note $\chi(G)$ un tel $k$).

\subsubsection{Coloration d'arêtes}\label{ex11_def_aretes}
Le problème de décision est le suivant :\\
soit $k \in \mathbb{N}$, est-il possible de trouver une fonction $\chi' : E \rightarrow
[1,k]$ telle que $\forall (x,y_1),(x,y_2) \in E,\ \chi'((x,y_1)) \neq \chi'((x,y_2))$.\\
Son problème d'optimisation associé est de trouver $k$ le plus petit possible tel que $G$
soit k-arêtes-colorable (on note $\chi'(G)$ un tel $k$).

\subsection{Complexité}\label{ex11_q1}
Ces deux problèmes de coloration appartiennent à la classe $NP$, de plus, celui de
sommets est non approximable, tandis que celui d'arêtes l'est mais n'appartient pas à la 
classe d'approximation $PTAS$.\\
Nous nous intéressons à déterminer un seuil d'approximation pour le problème
de coloration d'arêtes.

\subsection{Algorithme approché avec performance relative meilleure que $\frac{4}{3}$}\label{ex11_q2}
Supposons qu'il existe un algorithme $(\frac{4}{3} - \epsilon)$-approché pour le
problème de coloration des arêtes.
Nous avons alors $sol_A < \frac{4}{3} sol_{opt}$.
De plus, ce problème d'optimisation étant un problème de minimisation, nous avons $sol_A
\geq sol_{opt}$.

\subsubsection{(a)}\label{ex11_q2_a}
Si $sol_{opt} \leq 3$, on a : $sol_A < \frac{4}{3} \times sol_{opt} \leq \frac{4}{3}
\times 3 = 4$,
c'est à dire $sol_A < 4$, et $sol_A \in \mathbb{N}$, ie $sol_A \leq 3$.\\
Cet algorithme renvoie donc la solution optimale.

\subsubsection{(b)}\label{ex11_q2_b}
Si $sol_{opt} \geq 4$, l'algorithme renverra une solution strictement supérieure à 3.
Ainsi il est possible de déterminer si la solution optimale est inférieure à 3.

\subsection{Conclusion}\label{ex11_q3}
Etant donné que si un algorithme de rapport strictement inférieur à $\frac{4}{3}$
existait, nous pourrions répondre en temps polynomial au problème de savoir si un graphe est
3-arêtes-colorable (qui est NP-complet),
le seuil d'approximation minimum pour le problème de coloration d'arêtes est donc
supérieur ou égal à $\frac{4}{3}$.

