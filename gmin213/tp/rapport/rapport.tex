% !TEX encoding = UTF-8 Unicode
\documentclass[a4paper]{report}

%%% Encodage, langues, guillemets, ... %%%
\usepackage[french]{babel}       
\usepackage[utf8]{inputenc}
\usepackage{setspace}
\usepackage{listings}
\usepackage{lscape}
%%% Inclusion images et feuilles pdf %%%
\usepackage{graphicx}
\usepackage{pdfpages}
\usepackage{wrapfig}
\usepackage{geometry}
%%% Inclusion url et liens %%%
\usepackage{hyperref}
\usepackage{url}
% Changement des puces de enumerate
\usepackage{enumerate}
%%% Symboles, Mathématique %%%
% Symbole euro
\usepackage{eurosym}
% Mathématiques
\usepackage{amsmath}
\usepackage{amssymb}
%%% Schémas
% Graphes, géométrie et autres
\usepackage{tikz}
\usetikzlibrary{trees}
%%% Algorithme
\usepackage{algorithm}
\usepackage{algorithmic}

\usepackage{slashbox}

\usepackage{array}
\newcolumntype{C}[1]{>{\centering\arraybackslash$}p{#1}<{$}}
\renewcommand\arraystretch{1.6}

\newcommand{\PLNE}[0]{ Programmation Linéaire en Nombres Entiers }
\newcommand{\card}{\mathrm{Card}}


\pagestyle{headings}
\thispagestyle{empty}
\geometry{a4paper,twoside,left=2.5cm,right=2.5cm,marginparwidth=1.2cm,marginparsep=3mm,top=2.5cm,bottom=2.5cm}
\begin{document}
\large
\setlength{\parskip}{5mm plus2mm minus2mm}
\lstset{language=C, showstringspaces=false, numbers=left, numberstyle=\tiny, tabsize=4}

 
 
{\setlength{\parindent}{0cm}
Chloé DESDOUITS \hfill M1 Informatique \\
Guillerme DUVILLIE \\
Swan ROCHER
}
\vfill
{\centering \Huge \bfseries TP de méthodes de résolution de problèmes NP-complets\par}
\vfill
2012 \hfill UM2

\setcounter{tocdepth}{1}
\tableofcontents
\thispagestyle{empty}
\pagenumbering{arabic}


\chapter{Partie théorique}
\section{Exercice 1 - Sur le problème de la couverture sommet minimale : trois approches différentes :}

\subsection{Première approche : la programmation linéaire en nombres entiers}

\subsubsection{Justification de l'utilisation de la Programmation Linéaire en Nombres Entiers}

On considère le problème de la couverture minimale sous la forme suivante : $$
\left \lbrace \begin{array}{l}
	\min z = \sum_{j=1}^n x_i\\
	x_r + x_s \geq 1, \quad \forall \{v_r, v_s\} \in E \\
	x_j \in \{0,1\} \quad j = 1, \dot, n
\end{array} \right .
$$

La fonction objectif représente le nombre de sommets utilisés par la solution du problème. Le fait
de minimiser la fonction objectif permet d'assurer la couverture minimale. Chacune des clauses est
relative à une arête du graphe, et impose qu'au moins un des sommets adjacents à cette arête soit
dans la couverture.

On a donc bien un problème de Programmation Linéaire en Nombres Entiers permettant de résoudre le problème de la couverture minimale.

\subsubsection{Justification des clauses}

Considérons le graphe donné par la figure \ref{triangle}. 

\begin{figure}
	\begin{center}
		\begin{tikzpicture}
			\tikzset{node/.style={circle, draw=black}};
			\node[node] (A) at (0,2) {$A$};
			\node[node] (B) at (-1, 0) {$B$};
			\node[node] (C) at (1, 0) {$C$};

			\draw (A) -- (B) -- (C) -- (A);
		\end{tikzpicture}
	\end{center}
	\label{triangle}
	\caption{Exemple}
\end{figure}

Sur ce graphe, le programme linéaire en nombres entiers est le suivant :$$
\left \{ \begin{array}{l}
		\min z = x_A + x_B + x_C\\
		x_A + x_B \geq 1 \\
		x_A + x_C \geq 1 \\
		x_B + x_C \geq 1 \\
		x_A, x_B, x_C \in \{0, 1\}
	\end{array} \right .
	$$
Il est très simple ici de comprendre pourquoi il est impossible de considérer le programme linéaire
suivant :$$
\left \{ \begin{array}{l}
		\min z = x_A + x_B + x_C\\
		x_A + x_B = 1 \\
		x_A + x_C = 1 \\
		x_B + x_C = 1 \\
		x_A, x_B, x_C \in \{0, 1\}
	\end{array} \right .
	$$

Ce programmme ne permet pas de résoudre la couverture minimale sur le graphe donné par la figure
\ref{triangle}. Quelque soit le sommet choisi dans un premier lieu pour appartenir à la couverture
minimale, il est impossible d'en choisir un second pour compléter cette dernière. Prenons un
exemple, nous forçons le sommet $A$ à appartenir à la couverture minimale (respectivement $B$ et
$C$). Ce choix force : $x_B = 0$ et $x_C = 0$ (respectivement, $x_A = 0$ et $x_C = 0$, et $x_A = 0$
et $x_B$ = 0). Il est donc impossible de respecter la clause $x_B + x_C = 1$, le problème (au vu de
sa modélisation) n'aurait donc pas de solution, or le graphe de la figure \ref{trisol} montre le
contraire.

\begin{figure}
	\begin{center}
		\begin{tikzpicture}
			\tikzset{node/.style={circle, draw=black}};
			\node[node,draw=red] (A) at (0,2) {$A$};
			\node[node,draw=red] (B) at (-1, 0) {$B$};
			\node[node] (C) at (1, 0) {$C$};

			\draw[green] (A) -- (B) -- (C) -- (A);
		\end{tikzpicture}
	\end{center}
	\label{trisol}
	\caption{Solution}
\end{figure}

\subsubsection{Une borne inférieure des solutions optimales}

On cherche à montrer qu'une solution optimale du programme linéaire en nombres entiers est une borne
inférieure de toute solution optimale du programme relaxé. Raisonnons par l'absurde et considérons
une solution optimale du programme linéaire, notée $n^*$ telle qu'il existe $x^*$ solution optimale du problème
relaxé vérifiant $x^* < n^*$. Toute solution du programme linéaire est solution du programme
relaxé\footnote{Une solution appartenant ) $\mathbb{N}$ appartient aussi à $\mathbb{R}$}. Ceci
implique : $n^*$ solution du programme relaxé, et donc $x^* < n^*$ impossible. On a donc : $n^* \leq
x^*$ ce qui est la définition d'une borne inférieure.

\subsubsection{A propos de la relaxation de contrainte}

Pour démontrer que la relaxation des contraintes d'intégrité implique $x_r \geq \frac{1}{2}$ ou $x_s
\geq \frac{1}{2}$, le raisonnement par l'absurde sera utilisé. 
Soient $x_s$ et $x_r$ les variables relatives aux sommets $r$ et $s$ adjacents à l'arête $(rc)$ et
telles que, après relaxation des contraintes, on a : $x_r < \frac{1}{2}$ \underline{et} $x_s <
\frac{1}{2}$. On en déduit donc que $x_r + x_s < 1$ et donc la contrainte liée à l'arête $(rs)$ est
violée, l'hypothèse de départ est donc fausse. On a donc, $\forall (rs) \in V : x_r \geq \frac{1}{2}
\mbox{ et } x_s \geq{1}{2}$.

\subsubsection{Une 2-approximation}

Mettons en évidence le pire des cas pouvant se présenter : pour une arête $(rs) \in V$, un seul
sommet est nécessaire pour la couverture de cette dernière dans le cas de la couverture minimale,
mais l'algorithme approché retourne : $x_r = x_s = \frac{1}{2}$. Après la phase d'arondis, on a $x_r
= x_s = 1$ est donc les deux sommets appartiennent à la solution approchée, cette phase multiplie
donc au pire le nombre de sommets (pour chaque clause par 2), ce qui implique que le cardinal de la
solution approximée est au plus 2 fois la solution optimale.

Cet algorithme est donc une 2-approximation.

\subsubsection{Dans le cas d'un graphe valué}

\subsection{Seconde approche : la recherche d'un couplage maximal}

\subsubsection{Une 2-approximation}

Commençons par prouver que l'algorithme retourne une couverture des arêtes par les sommets.
Considérons donc une arête $(rs)$ non couverte par l'ensemble de sommets retourné par l'algorithme, par
définition du couplage, il serait donc possible d'ajouter $(rs)$ au couplage. Or le couplage calculé
par l'algorithme est maximal, on en déduit que l'arête $(rs)$ telle qu'elle est définie ne peut
exister et donc que l'ensemble de sommets obtenu couvre l'ensemble des arêtes du graphe.

Appelons $c$ le couplage calculé par l'algorithme et $x^*$ la solution optimale du problème de la
couverture par les sommets, on sait que $\card c \leq \card x^*$, or pour construire la solution
approchée, on ajoute à $C$ les deux extrêmités des arêtes utilisées pour le couplage. On a donc :$$
\card C = 2 \times \card c \leq 2 \times \card x^*$$ $$\Longrightarrow \frac{\card C}{\card x^*} =
2$$

\subsubsection{Exemple de graphe foireux}

Le graphe suivant met en évidence la borne 2 de l'algorithme.

\begin{tikzpicture}
	\tikzset{node/.style={circle}};
	\foreach \x/\xt/\col/\colb/\colc/\cold in {0/A/black/black/black/black,
	4/B/black/green/black/black, 8/C/black/black/red/black, 12/D/green/green/black/black}{
		\node[node, draw=\col] (\xt1) at (\x, 0) {$A$};
		\node[node, draw=\colb] (\xt2) at (\x+2, 0) {$B$};
		\node[node, draw=\col] (\xt3) at (\x+2, 2) {$C$};
		\node[node, draw=\colb] (\xt4) at (\x, 2) {$D$};

		\draw[-, \colc] (\xt1) -- (\xt2);
		\draw[-, \cold] (\xt2) -- (\xt3);
		\draw[-, \colc] (\xt3) -- (\xt4);
		\draw[-, \cold] (\xt4) -- (\xt1);
	}
	\node at (0, -2) {Le graphe};
	\node at (4, -2) {Une couverture minimale};
	\node at (9, -2) {Un couplage maximale};
	\node at (13, -2) {La solution renvoyée};
\end{tikzpicture}

\subsubsection{Application de l'algorithme}

\section{Sur le problème de la couverture d'ensembles}

\subsection{Modélisation du problème à l'aide de la PLNE}

On considère le problème de la couverture d'ensemble, défini par :
Soit $E = \lbrace e_1, \dots, e_n \rbrace$, soient $S_1, \dots, S_m$ des sous-ensembles non vides
de $E$ tels que $\forall i \in \{1, \dots, m\}$, on a : $S_i \subset E$. On associe à chaque
ensemble $S_j$ un poids $w_j \geq 0$. Le problème consiste à trouver une collection de sous-ensemble
de poids minimum et telle que $\bigcup_{i=1}^m S_i = E$.

Ce problème peut s'exprimer à l'aide de la Programmation Linéaire en Nombres entiers de la manière
suivante : $$
\left \lbrace
\begin{array}{l}
	min z = \sum_{i=1}^m w_ix_i \\
<<<<<<< HEAD
	\sum_{x_j : e_i \in S_j} x_j \geq 1 \quad \forall e_i \in E \\
=======
	\sum_{j / e_i \in S_j} x_j \geq 1 \quad \forall e_i \in E \\
>>>>>>> 1fcbeb357cf143821464d4111543a3762fdd056d
	x_i \in \{0, 1\}
\end{array} \right.
$$

\subsection{Procédure d'arrondis}

Soit $f = \max_{i=1,\dots,n} f_i$, avec $f_i$ le cardinal de l'ensemble des sous-ensembles de $E$
contenant $e_i$ : $ f_i = |\{j : e \in S_j\}|$.

Définissons la procédure d'arrondis suivante: $$
x_i = \left \{ \begin{array}{rcl}
		1 & \mbox{ si } & x_i \geq \frac{1}{f} \\
		0 & \mbox{ sinon } &
	\end{array} \right .
	$$

\subsection{Garantie d'une solution réalisable}
Cherchons à démontrer que cette procédure d'arrondis garantit une solution réalisable. Pour ce
<<<<<<< HEAD
faire, nous allons procéder par l'absurde. Supposons qu'il existe un sommet $k$ qui ne repsecte pas
sa contrainte d'intégrité associée, à savoir : $$
\sum_{x_j : e_k \in S_j} < 1 $$

Les variables $x_j$ de cette contrainte étant définies positives et entières, le cas pris en
considération si dessus implique que toutes les variables de l'inéquation sont nulles pour le
programme en nombres entiers et donc : $$
\forall x_j : e_k \in S_j < \frac{1}{f} $$ dans le cas de la version relaxée du problème. Or ceci
n'est possible que si le nombre de sous-ensemble contenant $x_k$ est supérieur à $f$, ce qui est
impossible par définition. Tous les sommets respectent donc leur contrainte d'intégrité et on en
déduit que la procédure d'arrondis garantit une solution réalisable.
=======
faire, considérons l'ensemble $C$ des sous-ensembles sélectionnés par la procédure d'arrondis : $$ C
= \{ S_i : x_i \geq \frac{1}{f} \} $$

Soir un élément quelconque $c$, par définition de $f$, $c$ appartient à, au plus $f$ sous-ensembles.
De plus, par définition du problème, on a : $$
\sum_{j : e \in S_j} x_j \geq 1 $$

On a donc au moins un des $x_j \geq \frac{1}{f}$, et donc $c$ est couvert par la solution. Cette
procédure d'arrondis garantit une solution réalisable.
>>>>>>> 1fcbeb357cf143821464d4111543a3762fdd056d

\subsection{Existence d'un algorithme $f$-approché}

Considérons l'algorithme \ref{algo_pourri}
\begin{algorithm}
	\caption{Approximation couverture par ensemble}
	\label{algo_pourri}
	\begin{algorithmic}[1]
		\STATE Exprimer le problème en programmation linéaire en nombres entiers
		\STATE Résoudre la version relaxée
		\STATE Réaliser la procédure d'arrondis
	\end{algorithmic}
\end{algorithm}

En utilisant la procédure d'arrondis étudiée plus haut, on sait que pour chaque clause, il existe
$x_k$ tel que $x_k \geq \frac{1}{f}$, au pire des cas \footnote{cas similaire à celui de l'exercice
1}, chaque $x_k$ est multiplié par $f$, ce qui implique que la solution obtenue est au plus $f$ fois
plus grande que la solution optimale. Il s'agit donc d'un algorithme f-approché.

\subsection{Cas ou $f=2$}

Si $f=2$ alors le problème devient une couverture minimum par sommets.

% !TEX encoding = UTF-8 Unicode
% !TEX root = ../Rapport/rapport.tex

\section{Sur le problème du couplage maximum de poids minimum}

\subsection{Modélisation du problème}
Soit un graphe $G = (V,E)$ avec $V$ l'ensemble de ses sommets et $E$ l'ensemble de ses arêtes. Soit $\{ \forall{(i,j) \in E}, X_{(i,j)}\}$ un ensemble de variables booléennes qui indiquent le choix de l'arête $(i,j)$ correspondante dans le couplage. Soit $P_{(i,j)}$ le poids de l'arête $(i,j)$.

La première modélisation intuitive est la suivante :

$ Maximiser $ $$ \sum\limits_{(i,j) \in E} (X_{(i,j)} \times H) - \sum\limits_{(i,j) \in E} (X_{(i,j)} \times P_{(i,j)}) $$
$ Sous\ contraintes $
$$ \forall{i \in V}, \sum\limits_{(i,j) \in E} X_{(i,j)} + \sum\limits_{(j,i) \in E} X_{(j,i)} \leq 1 $$
$$ \forall{(i,j) \in E}, X_{(i,j))} \in \{0,1\} $$
$$ \forall{(i,j) \in E}, P_{(i,j))} \geq 0 $$
avec H une constante très grande. Cependant, après quelques instants de réflexion, nous pouvons imaginer une modélisation plus élégante :

$ Minimiser $ $$ \sum\limits_{(i,j) \in E} (X_{(i,j)} \times P_{(i,j)}) $$
$ Sous\ contraintes $
$$ \forall{i \in V}, \sum\limits_{(i,j) \in E} X_{(i,j)} + \sum\limits_{(j,i) \in E} X_{(j,i)} = 1 $$
$$ \forall{(i,j) \in E}, X_{(i,j))} \in \{0,1\} $$
$$ \forall{(i,j) \in E}, P_{(i,j))} \geq 0 $$

En effet, les contraintes forcent le couplage à être maximum tandis que la fonction objectif le force à tendre vers le poids minimum.

\subsection{Modélisation du problème appliquée au graphe de la figure 2}
$ Minimiser $ $$ \epsilon \times X_{ab} + \epsilon \times X_{bc} + \epsilon \times X_{ac} + M \times X_{ae} + M \times X_{cd} + M \times X_{bf} + \epsilon \times X_{df} + \epsilon \times X_{de} + \epsilon \times X_{fe}$$
$ Sous\ contraintes $
$$ X_{ab} + X_{ac} + X_{ae} = 1 $$
$$ X_{ab} + X_{bc} + X_{bf} = 1 $$
$$ X_{ac} + X_{bc} + X_{cd} = 1 $$
$$ X_{cd} + X_{de} + X_{df} = 1 $$
$$ X_{ae} + X_{de} + X_{df} = 1 $$
$$ X_{ef} + X_{df} + X_{bf} = 1 $$
$$ \forall{(i,j) \in E}, X_{(i,j))} \in \{0,1\} $$
$$ \epsilon \geq 0 $$
$$ M \geq 0 $$

\subsection{Solution optimale entière $z(ILP)$}
Sur un exemple de cette taille, il est facile de trouver une solution à la main. Il y a plusieurs solutions optimales de poids total $M+2\epsilon$ sur cet exemple ; l'une d'entre elles est le couplage $\{(a,b), (c,d), (e,f)\}$.

La résolution de ce PLNE par $glpsol$ (solveur de $GLPK$) donne bien la même solution.

\subsection{Solution optimale $z(LP)$ pour le programme relaxé}
Relaxer le programme revient à transformer la contrainte d'intégrité des $X_{(i,j)}$ en la contrainte suivante :
$$ \forall{(i,j) \in E},  0 \leq X_{(i,j))} \leq 1 $$

En rentrant le PL relaxé dans $glpsol$, nous obtenons le couplage de poids total $3\epsilon$ : $$\{X_{ab} = 0.5 ; X_{ac} = 0.5 ; X_{bc} = 0.5 ; X_{df} = 0.5 ; X_{ef} = 0.5 ; X_{de} = 0.5\}$$ 

\subsection{Conclusion sur la pertinence de la formulation}
La solution trouvée au PLNE étant optimale, la formulation du problème semble être pertinente.



\section{Sur le problème de la coupe maximum}

% !TEX root = ../Rapport/rapport.tex

\section{Sur le problème de partition}


% !TEX root = ../../Rapport/rapport.tex
\section{Sur le problème du sac à dos simple}

\subsection{Construction d'un algorithme approché}

\subsubsection{(a) Complexité de l'algorithme}

La complexité de l'algorithme, une fois les nombres triés, est en $O(n)$. Cependant, il est possible
de démontrer que la complexité des algorithmes de tris basés sur des fonctions de comparaisons ne
peut être inférieures à $O(n\log n)$. la complexité de cet algorithme est donc similaire à la
compléxité de l'algorithme de tri utilisé.

\subsubsection{(b) Minoration de $cost(T)$}

Dans un premier temps nous montrerons que l'existence d'un indice $j$ pour lequel $cost(T) + w_{j+1}
> b$ n'est pas systématique, mais que si ce dernier n'existe pas, le problème n'existe pas non plus.

En effet, si cet indice n'existe pas, on a alors : 
$$ \sum_{i=1}^n w_1 \leq b $$
On en déduit donc que la solution optimale est donnée par $ T = \{ 1, \dot, n \} $. Ce qui en soit
même n'a aucun intérêt\footnote{Le problème du sac à dos, dans cette configuration n'est plus
NP-Complet}. Nous admettrons donc l'existence de cet indice $j$.

Supposons, à l'indice $j$, que $cost(T) \leq \frac{b}{2}$. Par définition de l'indice $j$, on a
$cost(T) + w_{j+1} > b$, ce qui implique $w_{j+1} > \frac{b}{2}$ et donc : $$
\begin{array}{lrcl}
	& w_{j+1} & > & cost(T) \\
	\Rightarrow & w_{j+1} & > & \sum_{i=1}^j w_i \\
	\Rightarrow & w_{j+1} & > & w_i
\end{array}
$$

Or les éléments $w_i$ étant triés, on a $w_j \geq w_{j+1}$. L'hypothèse de départ est donc fausse.

\subsubsection{(c) Performance relative de l'algorithme}

Notons $OPT$ la solution optimale au problème du sac à dos, et $A$ la solution approchée donnée par
l'algorithme. Rappelons, par définition du problème, que $OPT \leq b $. La performance relative nous
est donnée par : $$
\frac{OPT}{A} \leq \frac{b}{\frac{b}{2}} = 2 $$

La performance relative de l'algorithme est donc 2.

\subsection{Construction d'un schéma d'approximation}

\subsubsection{(a) Complexité de l'algorithme}

De façon très grossière, la construction de l'ensemble des sous-ensembles $S_k$ de $k$ éléments se
fait en $k\times n^k$ opérations. On a donc une construction des sous-ensembles réalisée en : 
$$\sum_{i=1}^k i . n^i $$
Soit une construction en $O(n^k)$.

L'ensemble étant déjà trié, l'algorithme glouton s'exécute en $O(n)$, d'où une exécution totale de
l'algorithme en $O(n^{k+1})$.


\subsubsection{(b) Comparaison à l'optimal}
% TODO


% !TEX root = ../../Rapport/rapport.tex
% !TEX encoding = UTF-8 Unicode


\section{Programmation dynamique}

\subsection{Sur le problème de la partition}
\subsubsection{Condition nécessaire sur la somme des poids des $n$ objets}

Les poids de chacun des objets étant des valeurs entières, pour qu'il existe deux sous-ensembles
distincts ayant le même poids, il faut que la somme des poids des objets soit paire.

\subsubsection{Récurrence}

Introduisons l'expression booléenne $T(i,j) $ : \emph{Etant donnés les $i$ premiers éléments de la
famille, il existe un sous-ensemble de ces $i$ éléments de poids $j$}. 

Dans un premier temps il paraît évident que si $j=0$, le sous-ensemble existe : il s'agit de
l'ensemble vide. \begin{equation}
\forall i \in \mathbb{[}0, n \mathbb{]}, \quad T(i,0) = \mbox{VRAI} 
\end{equation}

De plus, si $j$ est égal au poids de l'un des $i$ premiers éléments, alors l'existence du
sous-ensemble est avérée. Autrement dit, si l'on appelle $K_i$ l'ensemble formé par les poids des
objets $i$, on a : \begin{equation}
	\forall i \in \mathbb{[}0, n \mathbb{]}, \quad T(i,j \in K_i) = \mbox{VRAI}
\end{equation}

Troisièmement, l'ensemble des $i-1$ premiers objets est un sous ensemble de l'ensemble des $i$
premiers objets : $E_{i-1} \subset E_i$. On en déduit donc, que s'il existe un sous ensemble de
$E_{i-1}$ de poids $j$ tel que $T(i-1, j) = \mbox{VRAI}$, alors ce sous-ensemble est aussi un
sous-ensemble de $E_i$ tel que $T(i, j) = \mbox{VRAI}$ et ce par le principe d'inclusion.
\begin{equation}
	\forall i \in \mathbb{[}0, n \mathbb{]}, \quad \mbox{ si } T(i-1, j) = \mbox{VRAI}, \mbox{
	alors } T(i,j) = \mbox{VRAI}
\end{equation}

Enfin, considérons $S_j$ un sous-ensemble de $E_{i-1}$ de pois $j$ vérifiant donc $T(i-1, j) =
\mbox{VRAI}$, alors $S_j \cup a_i$ est un sous-ensemble de $E_i$ de poids $j+p(a_i)$ et donc
vérifiant $T(i,j+p(a_i)) = \mbox{VRAI}$. On en déduit donc :
\begin{equation}
	\forall i \in \mathbb{[}0, n \mathbb{]}, \quad \mbox{ si } T(i-1, j-p(a_i)) = \mbox{VRAI}, \mbox{
	alors } T(i,j) = \mbox{VRAI}
\end{equation}

En réunissant les équations ci dessus dans une seule expression booléenne, on obtient la formule de
la ligne $i$ en fonction de la ligne $i-1$ et $p(a_i)$ :
\begin{equation}
	T(i,j) = j == 0 \vee j == p(a_i) \vee T(i-1, j) \vee T(i-1, j-p(a_i))
\end{equation}

On peut alors donner un algorithme pour résoudre le problème de la partition : l'algorithme
\ref{solvepart}.

\begin{algorithm}[H]
	\caption{Partition}
	\label{algo_dyn_partition}
	\begin{algorithmic}[1]
		\STATE $P \leftarrow \sum_{i \in \mathbb{[}1, n \mathbb{]}} p(a_i)$
		\IF{$P \equiv 1 \pmod{2}$}
			\RETURN Pas de solution
		\ELSE
			\FOR {$j \in \mathbb{[}0,P/2\mathbb{]}$}
				\IF{$j = 0$}
					\STATE $T(0, j) \leftarrow \TRUE$
				\ELSE
					\STATE $T(0, j) \leftarrow \FALSE$
				\ENDIF
			\ENDFOR
			\FOR{$i \in \mathbb{[}1, n \mathbb{]}$}
				\FOR{$j \in \mathbb{[}0, P/2\mathbb{]}$}
					\STATE $T(i, j) \leftarrow [\left (j = 0) \vee (j = p(a_i)) \vee (T(i-1, j)) \vee (T(i-1, j-p(a_i))) \right]$
				\ENDFOR
			\ENDFOR
			\IF{$T(n, P / 2) = \FALSE$}
				\RETURN Pas de solution
			\ELSE
				\RETURN Partition-Solution($n$,$\frac{P}{2}$,$T$)
			\ENDIF
		\ENDIF
	\end{algorithmic}
\end{algorithm}

\begin{algorithm}
	\caption{Partition-Solution}
	\label{recsol}
	\begin{algorithmic}[1]
	\REQUIRE
		\IF{$i = 0$ \OR $j=0$}
			\RETURN $S$
		\ELSE
			\IF{$T(i-1,j) = \FALSE$}
				\RETURN $a_i \cup$ Partition-Solution($i-1$,$j-a_i$,$T$)
			\ELSE
				\RETURN Partition-Solution($i-1$, $j$, $T$)
			\ENDIF
		\ENDIF
	\end{algorithmic}
\end{algorithm}



Le sous ensemble solution $S_{sol}$ se construit à l'aide du principe suivant : pour une ligne $i>0$ et
un poids $j>0$ donnés, on a $a_i \in S_{sol}$ si $T(i-1, j) = \mbox{FAUX} \wedge T(i,j) =
\mbox{VRAI}$. Ceci n'est vrai que dans un sens et permet d'introduire l'implication suivante :
\begin{equation}
	\label{ineq}
	T(i-1, j) \oplus T(i, j) = \mbox{VRAI} \Rightarrow a_i \in S_{sol} 
\end{equation}

Ceci mérite quelques explications. Séparons les différents cas possibles et présentons les dans un
tableau : \\
\begin{center}
\begin{tabular}{|c|c|c|} \hline
	\backslashbox{$T(i-1, j)$}{$T(i,j)$} & VRAI & FAUX \\	\hline
	VRAI & Cas 1 & Cas 2 \\ \hline
	FAUX & Cas 3 & Cas 4 \\ \hline
\end{tabular}
\end{center}

Analysons les différents cas : \begin{enumerate}
	\item Dans ce cas, l'ensemble des $i-1$ premiers objets possède un sous-ensemble vérifiant les
		contraintes imposées. On peut donc en déduire qu'il existe un sous-ensemble ne contenant pas
		$a_i$ et vérifiant ces mêmes contraintes\footnote{Attention, celà ne veut pas dire qu'il
		n'existe pas de sous-ensembles vérifiant les contraintes et contenant $a_i$}. C'est cet ensemble
		qui nous intéresse
	\item Ce cas est un cas interdit par la formule de calcul de la ligne $i$. En effet, si $T(i-1, j)
		=$ VRAI, alors $T(i,j) =$ VRAI par construction.
	\item C'est ce cas qui est intéressant, puisqu'il nous indique que l'ajout de l'objet $a_i$ permet
		de trouver une solution au problème du sous-ensemble pour les valeurs données. On a donc $a_i$
		appartenant bel et bien à la solution.
	\item Ce cas indique juste que que la solution n'existe pas.
\end{enumerate}
~\\
De ces principes, on peut déduire l'algorithme \ref{recsol} de reconstruction de la solution.

\subsubsection{Complexité}
$O(n.P)$
%TODO

\subsubsection{Jeux d'essais}


\subsection{Le problème du sac à dos}

\begin{algorithm}[H]
	\caption{Sac à Dos}
	\label{algo_dyn_bag}
	\begin{algorithmic}[1]
		\FOR {$j \in \mathbb{\{}1, \ldots, volumeMax \mathbb{\}}$}
				\STATE $T[0, j] \leftarrow 0$
		\ENDFOR
		\FOR{$i \in \mathbb{\{}1, \ldots, n \mathbb{\}}$}
			\FOR{$j \in \mathbb{\{}1, \ldots, volumeMax \mathbb{\}}$}
				\IF{$j \neq 1$}
					\STATE $T[i, j] \leftarrow T[i, j-1]$
				\ENDIF
				\FOR{$k \in \mathbb{\{}0, \ldots, volumeMax/volume[i] \mathbb{\}}$}
					\STATE $T[i,j] \leftarrow \max(T[i,j], T[i-1,j-k \times volume[i]] + k \times utilite[i])$
				\ENDFOR
			\ENDFOR
		\ENDFOR
	
	\RETURN $T[n,volumeMax]$
	\end{algorithmic}
\end{algorithm}



\subsubsection{Justification des formules}

\subsubsection{Exemple}

Considérons le problème suivant :
Le sac à dos a une capacité de 6 et les propriétés des objets sont définies dans le tableau suivant
:
\begin{center}
\begin{tabular}{|c|c|c|c|c|} \hline
	utilité & 1 & 3 & 4 & 5 \\ \hline
	volume & 1 & 4 & 2 & 2 \\ \hline
\end{tabular}
\end{center}

L'algorithme va donc créer la matrice suivante :
\begin{center}
$$\begin{array}{|c|c|c|c|c|c|c|} \hline
  _{objet}^{volume} & 1 & 2 & 3 & 4 & 5 & 6 \\ \hline
	0		  & 0 & 0 & 0 & 0 & 0 & 0 \\ \hline
	1			& 1 \times i_1 = 1 & 2 \times i_1 = 2 & 3 \times i_1 = 3 & 4 \times i_1 = 4 & 5 \times i_1 = 5 & 6 \times i_1 = 6 \\ \hline
	2			& 1 \times i_1 = 1 & 2 \times i_1 = 2 & 3 \times i_1 = 3 & 4 \times i_1 = 4 & 5 \times i_1 = 5 & 6 \times i_1 = 6 \\ \hline
	3			& 1 \times i_1 = 1 & 1 \times i_3 = 4 & 1 \times i_1 + 1 \times i_3 = 5 & 2 \times i_3 = 8 & 1 \times i_1 + 2 \times i_3 = 9 & 3 \times i_3 = 12 \\ \hline
	4			& 1 \times i_1 = 1 & 1 \times i_3 = 4 & 1 \times i_1 + 1 \times i_4 = 6 & 2 \times i_4 = 10
	& 1 \times i_1 + 2 \times i_4 = 11  & 3 \times i_4 = 15 \\ \hline
\end{array}$$
\end{center}

Il ne nous reste plus qu'à lire la solution optimale dans la case [4,6].


\subsubsection{Complexité}
Notre algorithme génère une matrice de taille $n \times V$, et pour chaque case dans le pire des cas (si $v_i = 1$), il faudra $V$ 
itérations. Ainsi nous obtenons la complexité en temps suivante : $\bigcirc(n\times V^2)$.
De plus, si la matrice est stockée entièrement, la complexité en espace est donc en $\bigcirc(n\times V)$.

\subsection{Le problème du voyageur de commerce}

\begin{algorithm}[H]
	\caption{Voyageur de Commerce}
	\begin{algorithmic}[1]
		\FOR{$i \in \{1,...,n\}$}
			\STATE $C(\{0\},i) \leftarrow \omega(0,i)$
		\ENDFOR
		\FOR{$i \in \{2,...,n\}$}
			\FOR{$S \subseteq \{1,...,n\} : |S| = i$}
				\FOR{$j \in V \setminus S$}
					\STATE $C(S,j) \leftarrow \min\limits_{k \in S \setminus \{0\}}(C(S\setminus\{k\},k) + \omega(k,j))$
				\ENDFOR
			\ENDFOR
		\ENDFOR
		\RETURN $\min\limits_{i\in V\setminus\{0\}}(C(V\setminus\{i\},i) + \omega(i,0))
	\end{algorithmic}
\end{algorithm}



\subsubsection{Exemple}
\input{../theorique/Chloe/exemple_tsp.tex}

{\bf i = 1}\\
\begin{center}
$$\begin{array}{|c|c|} \hline
	_i^S & \{0\} \\ \hline
	1 & \{0,1\} = 1\\ \hline
	2 & \{0,2\} = 2\\ \hline
	3 & \{0,3\} = 1\\ \hline
	4 & \{0,4\} = 0\\ \hline
\end{array}$$
\end{center}


{\bf i = 2}\\
\begin{center}
$$\begin{array}{|c|c|c|c|c|} \hline
	_i^S & \{0,1\} & \{0,2\} & \{0,3\} & \{0,4\} \\ \hline
	1 & $\ding{55}$ & \{0,2,1\} = 5 & \{0,3,1\} = 6& \{0,4,1\} = 0\\ \hline
	2 & \{0,1,2\} = 4 & $\ding{55}$ & \{0,3,2\} = 3& \{0,4,2\} = 1\\ \hline
	3 & \{0,1,3\} = 6 & \{0,2,3\} = 4 & $\ding{55}$ & \{0,4,3\} = 4\\ \hline
	4 & \{0,1,4\} = 1 & \{0,2,4\} = 3 & \{0,3,4\} = 5& $\ding{55}$\\ \hline
\end{array}$$
\end{center}

{\bf i = 3}\\
\begin{center}
$$\begin{array}{|c|c|c|c|c|c|c|} \hline
	_i^S & \{0,1,2\} & \{0,1,3\} & \{0,1,4\} & \{0,2,3\}
& \{0,2,4\}& \{0,3,4\} 
\\ \hline
	1 & $\ding{55}$ & $\ding{55}$ & $\ding{55}$ & \{0,3,2,1\} = 6
& \{0,2,4,1\} = 3 & \{0,3,4,1\} = 5 \\ \hline
	2 & $\ding{55}$ & \{0,1,3,2\} = 8 & \{0,1,4,2\} = 2 & $\ding{55}$ 
& $\ding{55}$ & \{0,3,4,2\} = 6 \\ \hline
	3 & \{0,1,2,3\} = 6 & $\ding{55}$ & \{0,1,4,3\} = 5 & $\ding{55}$
& \{0,4,2,3\} = 3 & $\ding{55}$ \\ \hline
	4 & \{0,1,2,4\} = 5 & \{0,3,1,4\} = 6 & $\ding{55}$ & \{0,3,2,4\} = 4 
& $\ding{55}$ & $\ding{55}$ \\ \hline
\end{array}$$
\end{center}

{\bf i = 4}\\
\begin{center}
$$\begin{array}{|c|c|c|c|c|} \hline
	_i^S & \{0,1,2,3\} & \{0,1,2,4\} & \{0,1,3,4\} & \{0,2,3,4\} \\ \hline
	1 & $\ding{55}$ & $\ding{55}$ & $\ding{55}$ & \{0,3,2,4,1\} = 4 \\ \hline
	2 & $\ding{55}$ & $\ding{55}$ & \{0,1,4,3,2\} = 7 & $\ding{55}$ \\ \hline
	3 & $\ding{55}$ & \{0,1,4,2,3\} = 4 & $\ding{55}$ & \ding{55} \\ \hline
	4 & \{0,3,2,1,4\} = 5 & $\ding{55}$ & $\ding{55}$ & \ding{55} \\ \hline
\end{array}$$
\end{center}

Solution optimale : $\{0,3,2,1,4,0\} = 5$.

\subsubsection{Complexité}
La matrice générée ici est de taille $n,2^n$, et il faut $\bigcirc(n)$ opérations pour remplir une case.
Les complexités sont donc :
\begin{itemize}
	\item temps : $\bigcirc(n^2\times 2^n)$
	\item espace : $\bigcirc(n\times 2^n)$
\end{itemize}


% !TEX encoding = UTF-8 Unicode
% !TEX root = ../Rapport/rapport.tex

\section{Sur le produit matriciel}

\subsection{Nombre d'opérations nécessaires pour un produit}
Soit $M_i$ une $(p_i,p_{i+1})$-matrice. Soient $\pi _1$ et $\pi _2$ les produits matriciels à parenthèsage symétrique suivants : 
$$\pi _1 = (M_1(M_2(M_3(M_4(...(M_{k-1}.M_k))))))$$
$$\pi _2 = ((((((M_1.M_2)M_3)M_4)...)M_{k-1})M_k)$$
Nous admettons que le produit d'une $(p,q)$-matrice et d'une $(q,r)$-matrice peut se faire à l'aide de $p.q.r$ opérations. Tous les produits matriciels envisagés par la suite sont définis car le nombre de colonnes de la première matrice est égal au nombre de lignes de la seconde.

Évaluons le nombre d'opérations nécessaires au produit $\pi _1$ grâce au tableau suivant :
\begin{center}
\resizebox{\textwidth}{!}{
$ \begin{array}{|c|c|c|c|c|c|c|} \hline
	\pi _l & \pi _{1,1} & … & \pi _{1,4} & … & \pi _{1,k-2} & \pi _{1,k-1} \\ \hline
	=  & M_1 \times & … & M_{4} \times  & … & M_{k-2} \times & (M_{k-1} \times M_k) \\
	Nb.\ d'operations & 1.2.(k+1) & … & 4.5.(k+1) & … & (k-2).(k-1).(k+1) & (k-1).k.(k+1) \\
	Taille\ de\ la\ matrice & (1,k+1) & … & (4,k+1) & … & (k-2,k+1) & (k-1,k+1) \\ \hline
 \end{array} $
 }
 \end{center}
 
 Nous pouvons alors déduire la formule donnant le nombre d'opérations nécessaires pour calculer le produit $\pi _1$ :
$$ \sum\limits_{i =1}^{i < k} {[i\times (i+1) \times (k+1)]} $$

De la même façon, évaluons le nombre d'opérations nécessaires au produit $\pi _2$ grâce au tableau suivant :
\begin{center}
$ \begin{array}{|c|c|c|c|c|c|c|} \hline
	\pi _l & \pi _{1,1} & \pi _{1,2} & \pi _{1,3} & … & \pi _{1,k-2} & \pi _{1,k-1} \\ \hline
	=  & M_1 \times M_2 & \times M_3 & \times M_{4}  & … & \times M_{k-1} & \times M_k \\
	Nb.\ d'operations & 1.2.3 & 1.3.4 & 1.4.5 & … & 1.(k-1).k & 1.k.(k+1) \\
	Taille\ de\ la\ matrice & (1,3) & (1,4) & (1,5) & … & (1,k) & (1,k+1) \\ \hline
 \end{array} $
 \end{center}
 
 Nous pouvons alors déduire la formule donnant le nombre d'opérations nécessaires pour calculer le produit $\pi _2$ :
$$ \sum\limits_{i =1}^{i < k} {[(i+1) \times (i+2)]} $$

Nous remarquons tout d'abord que le produit $\pi_1$ dépend de $k$ alors que le produit $\pi_2$ n'en dépend pas. Comparons maintenant le nombre d'opérations nécessaires aux produits $\pi_1$ et $\pi_2$ :

\begin{center}
$ \begin{array}{|c|C{4.5cm}|C{4.5cm}|C{4.5cm}|} \hline
	 & \sum\limits_{i =1}^{i < k} {[i\times (i+1) \times (k+1)]} & vs & \sum\limits_{i =1}^{i < k} {[(i+1) \times (i+2)]} \\ \hline
	k=1  & 4 & < & 6  \\
	k=2 & 24 & > & 18 \\
	k=3 & 80 & > & 38 \\ \hline
 \end{array} $
 \end{center}
 
Nous pouvons en déduire que $\forall k \geq 2$ le nombre d'opérations nécessaires pour calculer le produit $\pi_1$ est supérieur au nombre d'opérations nécessaires pour calculer le produit $\pi_2$.

 
 
\subsection{Nombre de parenthèsages possibles d'un produit de $k$ matrices}


\subsection{Récurrence}


\subsection{Exemple}

\section{Résolution numérique par la méthode primal-dual}
Joker !

% !TEX root = ../../Rapport/rapport.tex
\section{Seuil d'approximation pour le problème Bin Packing}\label{ex10}

\subsection{Définition du problème}\label{ex10_q1}
Soient un ensemble $O = \{o_i : i \in [1;n]\}$ de $n$ objets, une fonction $q : O \rightarrow \mathbb{N}$, une taille de boîte $Q$.

Le problème de décision Bin Packing est le suivant :\\
Existe-t-il une k-partition $O_1,...,O_k \subseteq O$ telle que :
\begin{itemize}
	\item $O = \cup_{i = 1}^{k}O_i$,
	\item $\forall i,j \in \mathbb{N}$, $i \neq j$, $O_i \cap O_j = \emptyset$,
	\item $\sum_{o_{ij} \in O_i} \leq Q\ \forall i \in [1;k]$.
\end{itemize}
~\\
Le problème d'optimisation associé est :\\
Minimiser $k \in \mathbb{N} : \exists$ k-partition qui respecte les conditions ci-dessus.

Dans la suite on notera $B(O_i) = \sum\limits_{o_{ij} \in O_i}q(o_i)$ la somme des poids des
éléments de $O_i$, $B = B(O)$, et $K^*$ la solution optimale au problème d'optimisation.

\subsection{Construction d'une instance de Bin Packing à partir d'une instance de Partition}\label{ex10_q2}
Remarque : pour les définitions et notations concernant le problème de partition, voir
l'\href{ex5}{exercice \ref{ex5}}.

Soit $x$ une instance du problème partition, nous construisons
une instance $x'$ de Bin Packing de la manière suivante :
\begin{itemize}
	\item $O = A$,
	\item $q(o_i) = \frac{2 \times p(a_i)}{w(A)}$,
	\item $Q = 1$.
\end{itemize}
%TODO


\subsection{Instance positive}\label{ex10_q3}
Nous devons montrer que si $x$ est positive alors $K^*(x') = 2$ et que $K^*(x') = 3$
sinon.
Commençons par calculer $B = \sum_{a_i \in A} \frac{2 \times p(a_i)}{w(a)} =
\frac{2}{w(a)} \times \sum_{a_{i} \in A} = \frac{2}{w(a)} \times w(a) = 2$.

Montrons tout d'abord que $K^*(x') \leq 3$.\\
Nous supposons que $K(x') = 4 > 3$. Nous savons alors qu'il existe quatre ensembles
$O_1,O_2,O_3,O_4$ tels que $B(O_i) \leq 1$.
De plus, puisque les $O_i$ sont disjoints, nous avons $B = B(O_1)+B(O_2)+B(O_3)+B(O_4)$,
et nous pouvons énumérer les $O_i$ de sorte que $B(O_i) \geq B(O_{i+1})$.\\
Or, $B(O_i) + B(O_j) > 1\ \forall i \neq j$, en effet si ce n'était pas le cas, alors on
pourrait unir les deux ensembles et $K^*(x')$ serait égal à 3.
En particulier nous avons : $B(O_1)+B(O_2) > 1$ et $B(O_3)+B(O_4) > 1$, c'est à dire :
$2 = B(O_1) + B(O_2) + B(O_3) + B(O_4) > 2$, ce qui est absurde.\\
Nous avons donc que $K(x') \leq 3$.

Supposons que $x$ soit une instance positive.\\
Soit la solution $P$ de $x$ $Y_1,Y_2$, on a donc $w(Y_1) = w(Y_2) = \frac{w(A)}{2} = P$.
Soit $O_1$ (resp. $O_2$) l'image de $Y_1$ (resp. $Y_2$) dans le problème $x'$.\\
On a donc : $B(O_2) = B(O_1) = \frac{2 \times w(Y_1)}{w(A)} = \frac{2 \times w(A)}{2
\times w(A)} = 1 \leq Q$.\\
Ainsi $K^* = 2$.

Supposons maintenant que $x$ soit une instance négative et on suppose que $K(x') < 3$.
Si $K(x') = 1$, il est trivial de voir que $B = 2$ et que $B \leq Q = 1$ ce qui est
absurde.\\
Si $K(x') = 2$, on a alors $O = O_1 \cup O_2$. 
De plus, $B = 2$, $B = B(O_1) + B(O_2)$ et $B(O_i) \leq 1$.
On a donc $B(O_1) = B(O_2) = 1$.\\
Soit $Y_1$ (resp. $Y_2$) l'ensemble correspondant à $O_1$ (resp. $O_2$) dans le problème
x. On a :
$1 = w(O_1) = \frac{2}{w(A)} \times w(Y_1) \leftrightarrow w(Y_1) = \frac{w(A)}{2}$.
Donc il existe une solution $Y_1,Y_2$ au problème $x$, ce qui est absurde.\\
Ainsi si l'instance $x$ est négative, $K^*(x') = 3$.

\subsection{Classe d'approximation du Bin Packing}\label{ex10_q5}
Supposons qu'il existe un algorithme A qui donne une solution $(\frac{3}{2} -
\epsilon)$-approchée au problème Bin Packing.\\
Soit $x$ un problème de Partition.
Grâce à la réduction polynomiale étudiée ci-dessus, nous avons donc :
$sol_{A}(x') < \frac{3}{2}sol_{opt}(x')$.\\
Et si le problème partition admet une solution alors $sol_{opt}(x') = 2$, nous en
déduisons : $sol_{A}(x') < 3$, ie $sol_A(x') = 2 = sol_{opt}(x')$, et si ce n'est pas le
cas $sol_A(x') = 3$.
A permet donc de résoudre le problème Partition en temps polynomial, celui-ci étant
NP-complet, il n'existe pas un tel algorithme.\\
Ainsi le problème Bin Packing admet un seuil d'approximation en $\frac{3}{2}$ et donc 
n'appartient pas à la classe d'approximation $PTAS$.


\section{Seuil d'approximation pour le problème de la coloration de sommets (resp.
d'arêtes)}\label{ex11}

\subsection{Question 1}\label{ex11_q1}
Complexités.

\subsection{Question 2}\label{ex11_q2}

\subsection{Question 3}\label{ex11_q3}

% !TEX encoding = UTF-8 Unicode
% !TEX root = ../Rapport/rapport.tex

\section{Comparaison de \emph{branch and bound} et \emph{branch and cut}}

Nous considérons le  programme linéaire suivant :
$$
PL_0 \begin{cases}
max\ z(x_1,x_2) = 2x_1 + x_2 \\
 2x_1 + 5x_2 \leq 17 \\
 3x_1 + 2x_2 \leq 10 \\
 x_1, x_2 \geq 0 \\
\end{cases}
$$

\subsection{Polytope associé à ${PL}_0$}
\begin{tikzpicture}[scale = 0.7]

        \filldraw [smooth,draw=gray!20,fill=gray!20] plot[id=f3,domain=0:16.0/11.0]  function {(17.0/5.0) - (2.0/5.0)*x}
            -- (10.0/3.0,0)  -- (0,0) -- cycle;

    \draw[very thin,color=gray] (-5,-5) grid (10, 10);
    \draw[->] plot[id=axeX1] (-5,0) -- (10,0) node[right] {$x_1$};
    \draw[->] plot[id=axeX2] (0,-5) -- (0,10) node[above] {$x_2$};
    
    
    \draw[color=red, domain=-5:2.5, dashed] plot[id=obj] function{-2*x} 
        node[below] {$max\ z = 2x_1 + x_2$};
    \draw[color=blue, domain=-5:10] plot[id=c1] function{(17.0/5.0) - (2.0/5.0)*x} 
        node[right] {$2x_1 + 5x_2 \leq 17$};
    \draw[color=orange, domain=-10.0/3.0:20.0/3.0] plot[id=c2] function{(5.0)-(3.0/2.0)*x)} 
        node[right] {$3x_1+2x_2 \leq 10$};
   	
\end{tikzpicture}


\subsection{Résolution graphique}
\begin{tikzpicture}[scale = 0.7]

        \filldraw [smooth,draw=gray!20,fill=gray!20] plot[id=f3,domain=0:16.0/11.0]  function {(17.0/5.0) - (2.0/5.0)*x}
            -- (10.0/3.0,0)  -- (0,0) -- cycle;

    \draw[very thin,color=gray] (-5,-5) grid (10, 10);
    \draw[->] plot[id=axeX1] (-5,0) -- (10,0) node[right] {$x_1$};
    \draw[->] plot[id=axeX2] (0,-5) -- (0,10) node[above] {$x_2$};
    
    
    \draw[color=red, domain=-5:2.5, dashed] plot[id=obj] function{-2*x} 
        node[below] {$max\ z = 2x_1 + x_2$};
    \draw[color=red, domain=(-5+1.7):(2.5+1.7), dashed] plot[id=obj] function{17.0/5.0-2*x};
    \draw[color=red, domain=(-5+63.0/22.0):(2.5+63.0/22.0), dashed] plot[id=obj] function{63.0/11.0-2*x};
    \draw[color=red, domain=(-5+10.0/3.0):(2.5+10.0/3.0), dashed] plot[id=obj] function{20.0/3.0-2*x};
        
    \draw[color=blue, domain=-5:10] plot[id=c1] function{(17.0/5.0) - (2.0/5.0)*x} 
        node[right] {$2x_1 + 5x_2 \leq 17$};
    \draw[color=orange, domain=-10.0/3.0:20.0/3.0] plot[id=c2] function{(5.0)-(3.0/2.0)*x)} 
        node[right] {$3x_1+2x_2 \leq 10$};
   	
\end{tikzpicture}

Nous obtenons grâce à la résolution graphique : 
$$ \begin{cases}
x_1 = \frac{16}{11} \\
x_2 = 0 \\
z = \frac{20}{3}
\end{cases} $$


\subsection{Résolution par la méthode du simplexe}
Programme linéaire :
$$
PL_0 \begin{cases}
max\ z(x_1,x_2) = 2x_1 + x_2 \\
 2x_1 + 5x_2 \leq 17 \\
 3x_1 + 2x_2 \leq 10 \\
 x_1, x_2 \geq 0 \\
\end{cases}
$$

Forme standard :
$$
PL_0 \begin{cases}
max\ z(x_1,x_2) = 2x_1 + x_2 \\
 2x_1 + 5x_2 + y_1 = 17 \\
 3x_1 + 2x_2 +y_2 = 10 \\
 x_1, x_2, y_1, y_2 \geq 0 \\
\end{cases}
$$


Tableaux :

$$ \begin{array}{|C{1cm}|C{2cm}|C{1cm} C{1cm} C{1cm} C{1cm}|} \hline
	 &  & 2 & 1 & 0 & 0 \\ \hline
	0 & y_1 = 17 & 2 & 5 & 1 & 0 \\ 
	0 & y_2 = 10 & 3 & 2 & 0 & 1 \\ \hline
	 & z = 0 & -2 & -1 & 0 & 0 \\ \hline
 \end{array} $$
 
 $$ \begin{array}{|C{1cm}|C{2cm}|C{1cm} C{1cm} C{1cm} C{1cm}|} \hline
	 &  & 2 & 1 & 0 & 0 \\ \hline
	0 & y_1 = \frac{31}{3} & 0 & \frac{11}{3} & 1 & -\frac{2}{3} \\ 
	2 & x_1 = \frac{10}{3} &1 &  \frac{2}{3} & 0 & \frac{1}{3} \\ \hline
	 & z = \frac{20}{3} & 0 & \frac{1}{3} & 0 & \frac{2}{3} \\ \hline
 \end{array} $$


\subsection{Recherche d'une solution à valeurs entières}

\subsubsection{Branch and bound}

\subsubsection{Coupes de Gomory}


\chapter{Partie pratique}
% !TEX encoding = UTF-8 Unicode
% !TEX root = ../Rapport/rapport.tex


Nous avons décidé d'implémenter tous les algorithmes suivants en langage C pour des raisons d'efficacité.

\section{Programmation dynamique}
La programmation dynamique est un paradigme d'algorithmes exacts qui consiste à plonger le problème dans sa généralisation. Le principe est d'utiliser une fonction de récursion pour calculer la solution optimale d'un sous-problème du problème courant. À partir de la solution de ce sous-problème, on obtient la solution d'un sous-problème plus grand, etc, jusqu'à résoudre le problème de départ. Ainsi, tous les calculs sont effectués une et une seule fois.

%Espace de recherche, sous-ensemble de l'espace de recherche

\subsection{Problème de la partition}

\subsubsection{Description du problème}
Le problème de la 2-partition de $n$ objets consiste à trouver deux sous-ensembles de ces $n$ objets tel que chacun des sous-ensembles ait le même poids. Les poids de chacun des objets étant des valeurs entières, pour qu’il existe deux sous-ensembles distincts ayant le même poids, il faut que la somme des poids des objets soit paire.

\subsubsection{Algorithme de programmation dynamique}
Prenons l'algorithme de résolution présenté dans la section théorique (cf \ref{solvepart}) et rappelé ci-dessous. Cet algorithme est un algorithme exact admettant une complexité en $O(n.P)$ avec $n$ le nombre d'objets et $P$ la somme des poids de tous les objets.
\begin{algorithm}[H]
	\caption{solve-partition}
	\begin{algorithmic}[1]
		\STATE $P \leftarrow \sum_{i \in \mathbb{[}1, n \mathbb{]}} p(a_i)$
		\IF{$P \equiv 1 \pmod{2}$}
			\STATE Pas de solutions
		\ELSE
			\FOR {$j \in \mathbb{[}0,P/2\mathbb{]}$}
				\IF{$j = 0$}
					\STATE $T(0, j) \leftarrow \TRUE$
				\ELSE
					\STATE $T(0, j) \leftarrow \FALSE$
				\ENDIF
			\ENDFOR
			\FOR{$i \in \mathbb{[}1, n \mathbb{]}$}
				\FOR{$j \in \mathbb{[}0, P/2\mathbb{]}$}
					\STATE $T(i, j) \leftarrow [\left (j = 0) \vee (j = p(a_i)) \vee (T(i-1, j)) \vee (T(i-1, j-p(a_i))) \right]$
				\ENDFOR
			\ENDFOR
			\IF{$T(n, P / 2) = \FALSE$}
				\STATE Pas de solutions
			\ELSE
				\STATE Construire le sous ensemble solution
			\ENDIF
		\ENDIF
	\end{algorithmic}
\end{algorithm}

\subsubsection{Tests}

\begin{figure}[H]
	\includegraphics[width=\linewidth]{../pratique/prog_dynamique_dev/res/partition.png}
\end{figure}


\subsection{Problème du sac à dos}

\subsubsection{Description du problème}
Le problème du sac à dos consiste à vouloir mettre des objets ayant chacun différents volumes et utilités dans un sac à dos à volume limité. Le but est d'obtenir un sac à dos rempli avec une utilité maximum.

\subsubsection{Algorithme de programmation dynamique}
Prenons l'algorithme de résolution présenté ci-dessous. Cet algorithme est un algorithme exact admettant une complexité en $O(n.V^2)$ avec $n$ le nombre d'objets et $V$ le volume du sac à dos. Notons que la borne (pire des cas) est atteinte si tous les objets sont de volume 1.
\begin{algorithm}[H]
	\caption{sac à dos}
	\begin{algorithmic}[1]
		\FOR {$j \in \mathbb{\{}1, \ldots, volumeMax \mathbb{\}}$}
				\STATE $T[0, j] \leftarrow 0$
		\ENDFOR
		\FOR{$i \in \mathbb{\{}1, \ldots, n \mathbb{\}}$}
			\FOR{$j \in \mathbb{\{}1, \ldots, volumeMax \mathbb{\}}$}
				\IF{$j \neq 1$}
					\STATE $T[i, j] \leftarrow T[i, j-1]$
				\ENDIF
				\FOR{$k \in \mathbb{\{}0, \ldots, volumeMax/volume[i] \mathbb{\}}$}
					\STATE $T[i,j] \leftarrow \max(T[i,j], T[i-1,j-k \times volume[i]] + k \times utilite[i])$
				\ENDFOR
			\ENDFOR
		\ENDFOR
	
	\RETURN $T[n,volumeMax]$
	\end{algorithmic}
\end{algorithm}

\subsubsection{Implémentation}\label{bag_impl}

\subsubsection{Tests}

\begin{figure}[H]
	\includegraphics[width=\linewidth]{../pratique/prog_dynamique_dev/res/bag.png}
\end{figure}


\subsection{Problème du voyageur de commerce}

\subsubsection{Description du problème}
Le problème du voyageur de commerce consiste à trouver un cycle hamiltonien de poids minimum.
\subsubsection{Algorithme de programmation dynamique}

\begin{algorithm}[H]
	\caption{TSP}
	\begin{algorithmic}[1]
		\FOR {$i \in \mathbb{\{}1, \ldots, n-1 \mathbb{\}}$}
			\STATE $MeilleureChaine(\{v_i\},v_i) = (v_0, v_i)$
			\STATE $ValeurChaine(\{v_i\},v_i) = d(v_0, v_i)$
		\ENDFOR
		\FOR{$j \in \mathbb{\{}1, \ldots, n-2 \mathbb{\}}$}
			\FOR{$V^{'} \subseteq \mathbb{\{ } v_1, \ldots, v_{n-1} \mathbb{ \} } : |V^{'}|=j$}
				\FOR{$v \in V^{'}$}
					\STATE $ValeurChaine(V^{'},v) = \min\limits_{w \in V^{'}}\{ValeurChaine(V^{'}\backslash \{v\},w)+d(w,v)\}$
					\STATE $MeilleureChaine(V^{'},v) = (MeilleureChaine(V^{'}\backslash \{v\},w_0),v)$ (où $w_0$ est le sommet qui atteint ce minimum)
				\ENDFOR
			\ENDFOR
		\ENDFOR
		
		\RETURN $T=arg \min\limits_{v=v_2}^{v_n}(ValeurChaine(\{v_1,\ldots,v_{n-1}\}\backslash \{v\},v)+d(v,v_0))$
	\end{algorithmic}
\end{algorithm}


\subsubsection{Tests}

\begin{figure}[H]
	\includegraphics[width=\linewidth]{../pratique/prog_dynamique_dev/res/tsp.png}
\end{figure}



% !TEX encoding = UTF-8 Unicode
% !TEX root = ../Rapport/rapport.tex

\section{Branch and bound}

\subsection{Principe général du branch\&bound}
La méthode du branch and bound consiste à énumérer les solutions d'un problème sous forme d'arbre binaire et à conserver à chaque pas la meilleure solution réalisable. Certaines branches de l'arbre ne sont pas évaluée lorsqu'elles donnent de moins bonnes solutions que la meilleure solution courante. Une méthode de séparation est appliquée à chaque nœud de l'arbre pour le partitionner en deux sous-problèmes et ainsi générer ses fils. Une méthode d'évaluation, quant à elle, sert à déterminer la solution optimale associée à un nœud de l'arbre.

\subsection{Méthode de séparation}
La méthode de séparation est utilisée pour séparer le problème en plusieurs sous-problèmes qui forment une partition de l'ensemble des solutions initiales. Les sous-problèmes générés sont les fils dans l'arbre du problème de départ.


\subsection{Méthode d'évaluation}
La méthode d'évaluation, quant à elle, est utilisée pour déterminer la solution optimale associée à un nœud de l'arbre.


\subsection{Stratégie de parcours}
Enfin, la stratégie de parcours de l'arbre consiste à générer les nœuds en largeur ou en profondeur.


\subsection{Solution initiale du voyageur de commerce}


\subsubsection{Chaine de poids le plus faible}
\begin{figure}[H]
	\includegraphics[width=\linewidth]{../pratique/branch_and_bound_dev/tsp_bb.png}
\end{figure}

\subsubsection{voisinage 2-opt}

\subsubsection{voisinage 3-opt}


\subsection{Comparaisons}


\subsection{Algorithme $\frac{3}{2}$ approché}



\section{Comparaison du Branch\&Bound et de la programmation dynamique sur l'exemple du TSP}


\begin{figure}[H]
	\includegraphics[width=\linewidth]{../pratique/comp.png}
\end{figure}





\vfill
{\raggedleft Réalisé avec \LaTeX{} \par}

\end{document}
