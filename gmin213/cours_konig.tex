\documentclass[a4paper, 11pt]{article}

\usepackage[utf8x]{inputenc}
\usepackage[T1]{fontenc}
\usepackage[french]{babel}
\usepackage{graphicx}
\usepackage{moreverb}
\usepackage{subfigure}
\usepackage{pdfpages}
\usepackage{eurosym}
\usepackage{fullpage}
\usepackage[fleqn]{amsmath}
\usepackage{bbm}
\usepackage{amssymb}
\usepackage{tikz}
\usepackage{algorithm}
\usepackage{algorithmic}
\usepackage{pgf}

\usetikzlibrary{arrows,automata}

\newtheorem{lemma}{Lemme}[subsection]
\newtheorem{thrm}{Théorème}[section]

\begin{document}

On cherche à prouver qu'il n'existe pas d'algorithme $\rho$-approché permettant de résoudre le
problème du voyageur de commerce sans l'inégalité triangulaire. Pour ce faire, raisonnons par 
l'absurde.

\underline{Hypothèse :} il existe un algorithme $\rho$-approché, $\rho \in \mathbb{R}$.

On s'intéresse au problème du cycle Hamiltonien :
\begin{algorithm}
	\caption{Problème du cycle Hamiltonien}
	\begin{algorithmic}[1]
		\STATE $\forall (i, j) \in V^2$
		\IF {$\{i,j\} \in  E$}
			\STATE $d_{ij} \leftarrow 1$
		\ELSE 
			\STATE $d_{ij} \leftarrow \rho|V| + 1$
		\ENDIF
		\STATE $\mu \leftarrow A(V,D)$
		\IF {longueur($\mu$) $\leq \rho|V|$}
			\STATE G est Hamiltonien
		\ELSE
			\STATE G n'est pas Hamiltonien
		\ENDIF
	\end{algorithmic}
\end{algorithm}

En effet, si longueur$(\mu) \leq \rho|V|$, alors il n'y a aucune arête de valeur $\rho|V| + 1$, donc
toutes les distances entre 2 villes consécutives de $\mu$ sont égales à 1, donc $G$ est Hamiltonien.

Si longueur$(\mu) > \rho$, comme l'algorithme est $\rho$-approché, on a : $$
\begin{array}{lrcl}
	&\displaystyle \frac{l(\mu)}{c^*} & \leq & \rho \\
	\Rightarrow & c^* & \geq & \frac{l(\mu)}{\rho} > |V|
\end{array} $$
Donc le meilleur cycle n'utilise pas que des arêtes de poids 1, et donc $G$ n'est pas Hamiltonien.

\section{Comment trouver un schéma d'approximation polynomial ?}

Trouver un algorithme pseudo polynomial (la complexité dépend de la valeur de la donnée de façon
polynomiale). Pour accélérer le calcul, on divise la valeur de la donnée, mais on introduit des
erreurs liées à la division : on perd l'optimalité, la division n'est pas forcément pertinente.

Une autre technique utilisée, repose sur le calcul d'échantillons, c'est à dire une partie des
solutions et les solutions \emph{oubliées} ont un représentant dans l'échantillon.

\subsection{Problème du sac à dos}

On énonce le problème du sac à dos ainsi : $$
KP \left \lbrace \begin{array}{rcl}
	\max p_ix_i \\
	\displaystyle \sum a_i x_i & < & B 
\end{array} \right . $$

$p_i$ (resp. $a_i$) est le poids (resp. le volume) de l'objet $i$, avec $B$ le volume du sac à dos,
$x_i \in \{0,1\}$ et :
\begin{itemize}
	\item $x_i = 0 \rightarrow $ l'objet $i$ n'est pas mis dans le sac à dos
	\item $x_i = 1 \rightarrow $ l'objet $i$ est mis dans le sac à dos
\end{itemize}

Si on a une instance du problème de la partition, on pose : $$
\begin{array}{rcl}
	p_i & = & a_i \quad (\mbox{ = valeur de l'entier } i \\
		B & = & \displaystyle \frac{\sum_{i=1}^n p_i}{2}
\end{array} $$

On résouds avec un algorithme polynomial $KP$. Soit S l'ensemble des objetrs mis dans le sac à dos.
La réponse est OUI pour le problème de la partition ssi $\displaystyle \sum_{i\in S} p_i = B$. On
voit alors que $KP$ est plus difficile que partition, il est NP-difficile.


Soit $F_j(i) =$ volume minimal pour atteindre le poids $i$ avec les objets de $1$ à $j$. On a alors
: $F_1(0) = 0$, $F_1(p_1) = a_1$ et $F_1(i) = +\infty$ avec $i \neq 0$ et $i \neq p_1$. De là on
peut donc calculer $F_j(i)$ en fonction de $F_{j-1}(j)$ : $$
F_j(i) =  min (F_{j-1}(i), F_{j-1} (i - p_j) + a_j) $$. 

On cherche le maximum pour $i$ tel que $F_n (i) \leq B$, on calcule donc $F_j(i)$ pour $j$ allant de
$1$ à $n$ et pour $i$ allant de $0$ à $\displaystyle \sum_{k=1}^n p_k$. La complexité est égale à
$O(n \displaystyle\sum_{i=1}^n p_i)$ c'est à dire : $O (n^2 P_{\max})$ ou $P_{\max} = \max p_i$.

Pour cet algorithme, on a une complexité en $0 (n^2 P_{\max})$. Pour accélérer, le calcul, on pose :
$\forall j P_j(K) = \displaymath \left \lfloor\frac{P_j}{K} \right \rfloor$, la complexité se ramène
à une complexité en $\displaystyle 0 (n^2 \frac{P_{\max}}{K})$. 

À partir de maintenant, on considère le problème : $$
KP(K= \left \lbrace \begin{array}{rcl}
	\max \displaystyle \sum \left \lfloor \frac{P_i}{K} \right \rfloor x_i \\
	\displaystyle \sum a_i x_i & \leq B 
\end{array} \right .
$$

Soit $S(K)$ la solution optimale obtenue en résolvant $KP(K=$, soit $S^*$ la solution optimale de
$KP$. On a : $$
\sum_{j\in S(k) P_j \geq K \sum_{j\inS(K)} \left \lfloor \frac{P_j}{K} \right \rfloor \geq K \sum_{j
	\in S^*} \left \lfloor \frac {P_j}{K} \right \rfloor \geq K \sum_{j\inS^*} \left ( \frac{P_j}{K} -
	1 \right ) \geq \sum_{j\in S^*} P_j - \sum_{j\in S^*} K $$

	Ce qui nous donne : $$
	\frac{\sum_{j\in S(K)} P_j}{\sum_{j \in S^*} P_j \geq 1 - \frac{\sum_{j\in S^*} K}{\sum_{j\in S^*}
		P_j} \geq 1 - \frac{KN}{P_{\max}} $$
\end{document}
