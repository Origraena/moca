%ex16.tex

\section{Exercice 16 - Coloration des Sommets d'un Graphe}\label{ex16}

\subsection{Question 1}\label{ex16_q1}
Nous devons montrer que l'algorithme \ref{ex14_algo} n'est pas r-approch\'e quel que soit
$r \in N$.

Soit la propri\'et\'e $P_r$ : "l'algorithme n'est pas r-approch\'e".
Par r\'ecurrence sur $r \in N$ :

$P_1$ : Si $P_1$ est fausse, la solution renvoy\'ee est optimale, l'algorithme \'etant polynomial et le
probl\`eme de coloration de sommets appartenant \`a la classe $NP$, il est \'evident que
$P_1$ est vraie (si $P \neq NP$).

Sous l'hypoth\`ese que $P_r$ est vraie, nous allons montrer que $P_{r+1}$ l'est
\'egalement :\\
On suppose que $P_{r+1}$ est fausse (d\'emonstration par l'absurde).\\
L'algorithme renvoie donc $k \times (r+1)$ couleurs.
Soit $G' = G \setminus \{v : couleur(v) > k \times r\}$ (c'est \`a dire qu'on retire de G
tous les sommets color\'es par les $k$ derni\`eres couleurs).\\
Il est trivial de voir que l'algorithme renvoie donc une solution r-approch\'ee pour
$G'$ (tout en restant polynomial en fonction du temps).\\
Or ceci est impossible par hypoth\`ese de r\'ecurrence $\rightarrow$ absurde.\\
$P_{r+1}$ est donc vraie.

\subsection{Question 2}\label{ex16_q2}
Nous devons maintenant montrer qu'il existe toujours un ordre des sommets tel que
l'algorithme \ref{ex14_algo} renvoie une solution optimale.

Soit $G$ un graphe quelconque.\\
Soit $couleur(x_i) \forall x_i \in G$ une coloration optimale de $G$.\\
Soit la num\'erotation $\{y_i\} \forall y_i \in G$ telle que $couleur(y_i) <=
couleur(y_i+1)$.\\
Il est \'evident qu'une telle num\'erotation existe et qu'en suivant celle-ci
l'algorithme renverra toujours la solution optimale.

\subsection{Question 3}\label{ex16_q3}
Le probl\`eme de savoir si un graphe est 2-colorable est polynomial puisque l'algorithme
suivant permet toujours de donner la r\'eponse :\\
On attribue au premier sommet la couleur 1, puis la couleur 2 \`a tous ses voisins, \`a
nouveau la couleur 1 \`a leurs voisins respectifs, et on continue ainsi tant qu'il
n'existe pas deux sommets adjacents partageant la m\^eme couleur.\\
Si aucune erreur n'est rep\'er\'ee a la fin de l'algorithme, le graphe est 2-colorable.

\subsection{Question 4}\label{ex16_q4}
Nous devons montrer que si $G$ est 3-colorable, alors $\forall x \in G\ voisinage(x)$ est
un graphe biparti.

Soit $G$ un graphe 3-colorable.\\
Soit $x \in G$ un sommet quelconque. On note 1 la couleur de x.\\
Soit $G_x$ le graphe du voisinage de $x$.\\
Puisque tous les sommets $y \in G_x$ sont adjacents \`a $x$ par construction, ils ne
peuvent pas \^etre de sa couleur. C'est \`a dire que $couleur(y) > 1$.\\
Or $G$ est 3-colorable, donc $couleur(y) <= 3$.\\
On a donc que $couleur(y) \in \{2,3\}$.
Ie, $G_x$ est 2-colorable, et tout graphe 2-colorable admet une bipartition de ses
sommets (cf \ref{ex4}).\\
$G_x$ est donc un graphe biparti.

\subsection{Question 5}\label{ex16_q5}
Il est question de trouver un algorithme $\sqrt{n}$-approch\'e permettant de colorer un
graphe 3-colorable.
Soit l'algorithme suivant :

\begin{center}
\begin{algorithm}[H]
\caption{Coloration de graphe 3-colorable}\label{ex16_algo1}
\algsetup{indent=2em,linenodelimiter= }
\begin{algorithmic}[1]
\REQUIRE $G$ : graphe de n sommets 3-colorable
\ENSURE $couleur(x)$ : couleur attribu\'e au sommet x $\forall x \in G$ 
	\STATE trier les $x_i \in G$ par ordre d\'ecroissant de degr\'e
	\STATE $k \leftarrow 1$
	\STATE $couleur \leftarrow \emptyset$
	\REPEAT
		\STATE $x \leftarrow sommet(G)$
		\STATE $G' \leftarrow voisinage(x)$
		\STATE $couleur \leftarrow couleur \cup 2coloration(G',k)$
		\STATE $k \leftarrow k+2$
		\STATE $G \leftarrow G \setminus G'$
	\UNTIL {$deg(sommet(G)) < \sqrt{n}$}
	\STATE $couleur \leftarrow couleur \cup DMaxPlus1coloration(G,k)$
\RETURN $couleur$
\end{algorithmic}
\end{algorithm}
\end{center}

Il faut maintenant prouver que cet algorithme renvoie bien une $\sqrt{n}$-approximation.

Tout d'abord l'op\'eration $2coloration(G,k)$ (qui colore un graphe 2-colorable avec
les couleurs k et k+1) est expliqu\'ee en question 3 (\ref{ex16_q3}).\\
Quant \`a $DMaxPlus1coloration(G,k)$ qui colore un graphe avec les couleurs
$[k,k+deg_{max}(G)]$, il suffit de voir qu'il est simple de colorer G si on s'autorise 1
couleur de plus que le degr\'e maximum de G.
En effet puisque chaque sommet a au plus D voisins, il reste toujours au moins une
couleur non utilis\'ee.\\
De plus, ces deux op\'erations sont \'evidemment polynomiales.

Il faut donc compter le nombre de couleurs utilis\'ees.\\
A chaque fois qu'on entre dans la boucle on \'elimine $\sqrt{n}$ sommets (puisque le
sommet $x$ a un degr\'e d'au moins $\sqrt{n}$).\\
On entrera donc dans cette boucle au plus $\frac{n}{\sqrt{n}} = \sqrt{n}$ fois.
En effet, une fois ce nombre de passages effectu\'es le degr\'e maximum de G est
forc\'ement inf\'erieur \`a $\sqrt{n}$.\\
Or \`a chaque it\'eration on utilise seulement deux couleurs, en effet $G'$ est
2-colorable (\ref{ex16_q4}).\\
Ce qui am\`ene le nombre de couleurs \`a $2 \times \sqrt{n}$ \`a la fin de la boucle.\\
Une fois celle-ci termin\'ee il ne nous reste plus que la derni\`ere coloration qui
utilise donc une couleur de plus que le degr\'e maximum de G.\\
Or $deg_{max}(g) < \sqrt{n}$, ce qui au pire ajoutera encore $\sqrt{n}$ couleurs.\\
En faisant la somme nous arrivons donc \`a $3 \times \sqrt{n}$ couleurs.

Le graphe \'etant 3-colorable, le nombre de couleurs optimal est 3, donc en calculant le
ration de cet algorithme par rapport \`a la meilleure solution nous obtenons bien :
$\frac{3 \times \sqrt{n}}{3} = \sqrt{n}$.

Cet algorithme est bien $\sqrt{n}$-approch\'e.

\subsection{Question 6}\label{ex16_q6}
Si un graphe dont le voisinage de tout sommet est un graphe biparti impliquait que ce
graphe est 3-colorable, alors il exxisterait un algorithme polynomial permettant de
savoir si un graphe est 3-colorable (probl\`eme $NP$) :

\begin{center}
\begin{algorithm}[H]
\caption{Graphe 3-colorable ?}\label{ex16_algo2}
\algsetup{indent=2em,linenodelimiter= }
\begin{algorithmic}[1]
\REQUIRE $G$ : graphe de n sommets
\ENSURE vrai ssi G est 3-colorable, faux sinon  
	\FORALL {sommet $x \in G$}
		\STATE $G_x \leftarrow voisinage(x)$
		\IF {$G_x$ n'est pas 2-colorable}
			\RETURN faux
		\ENDIF
	\ENDFOR
	\RETURN vrai
\end{algorithmic}
\end{algorithm}
\end{center}

Et donc en supposant que $P \neq NP$, ceci est impossible.

