% !TEX root = ../main.tex

\chapter{Contraintes}\label{chap6}

\section{Exercice 19 - Mélangeons les reines et les sudokus}

\subsection{Modèles}
Pour le modèle orienté case de l'échiquier à remplir, on pose l'ensemble des $X_{ij} \in \{0,1\}, \forall i,j \in \{1,\ldots,n\}$. Ces variables booléennes $X_{ij}$ représentent la présence ou non d'une reine sur une case $ij$. Les contraintes sont les suivantes :
$$ Lignes : \forall i \in \{1,\ldots,n\}, \sum\limits_{j=1}^{n}{X_{ij}} = 1 $$
$$ Colonnes : \forall j \in \{1,\ldots,n\}, \sum\limits_{i=1}^{n}{X_{ij}} = 1 $$
%$$ Diagonale\ NO - SE : \sum\limits_{i=1,j=1}^{n}{X_{ij}} = 1 $$
%$$ Diagonale\ SO - NE : \sum\limits_{i=n,j=1}^{i \geq 1, j \leq n}{X_{ij}} = 1 $$
$$ Diagonales : \forall i,j,k,l : X_{ij} = 1 \wedge X_{kl} = 1 , |i - k| \ne |j-l| $$

Pour le modèle orienté position de la reine sur une colonne, on pose l'ensemble des $X_{i} \in \{1,\ldots,n\}, \forall i \in \{1,\ldots,n\}$. Ces variables $X_{i}$ représentent la position d'une reine sur la colonne $i$. Les contraintes sont les suivantes :
$$ Lignes : \forall i,j : 1 \leq i < j \leq n, X_i \ne X_j $$
$$ Diagonales : \forall i,j : 1 \leq i < j \leq n, |X_i - X_j| \ne |j-i| $$

\subsection{Dual du modèle position sur la colonne}
Le dual de ce modèle est celui de la position des reines sur les lignes. On pose l'ensemble des $Y_{i} \in \{1,\ldots,n\}, \forall i \in \{1,\ldots,n\}$. Ces variables $Y_{i}$ représentent la position d'une reine sur la ligne $i$. Les contraintes sont les suivantes :
$$ Colonnes : \forall i,j : 1 \leq i < j \leq n, Y_i \ne Y_j $$
$$ Diagonales : \forall i,j : 1 \leq i < j \leq n, |Y_i - Y_j| \ne |j-i| $$
On remarque que les deux modèles sont strictement équivalents grâce à la relation suivante : $X_i = j \iff Y_j = i$.

\subsection{Stratégie de recherche}
Comme ce problème génère un grand espace de recherche, il est nécessaire de chercher à en explorer le moins possible (dans le cas de recherche d'une seule solution). Afin de réduire le facteur de branchement, on va chercher à affecter en priorité les variables dont le domaine est le plus réduit par les contraintes posées précédemment.

D'autre part, on va également chercher à affecter d'abord les variables qui génèrent le plus de contraintes car ce sont celles qui réduiront le plus les possibilités restantes.

\subsection{Modèle pour la coloration des reines}
Les modèles proposés précédemment permettent de résoudre ce problème avec une seule couleur, il faut donc ajouter une dimension à notre modèle : la dimension des couleurs. Reprenons donc le modèle orienté case de l'échiquier à remplir avec des variables entières cette fois et les contraintes suivantes :
$$ Lignes : \forall i,j,k \in \{1,\ldots,n\} : j \ne k, X_{ij} \ne X_{ik} $$
$$ Colonnes : \forall i,j,k \in \{1,\ldots,n\} : i \ne j, X_{ik} \ne X_{jk} $$
$$ Diagonales : \forall i,j,k,l : |i - k| = |j-l| , X_{ij} \ne X_{kl}$$
Notons que chaque $X_{ij}$ est affectée à la plus petite valeur possible de façon à minimiser le nombre de couleurs utilisées.

\subsection{Solutions pour $n = 5$}
Voici la solution donnée par l'énoncé :

$\begin{array}{|c|c|c|c|c|}
\hline
0 & 1 & 2 & 3 & 4 \\ \hline
3 & 4 & 0 & 1 & 2 \\ \hline
1 & 2 & 3 & 4 & 0 \\ \hline
4 & 0 & 1 & 2 & 3 \\ \hline
2 & 3 & 4 & 0 & 1 \\ \hline
\end{array}$

Il existe d'autres solutions comme celle-ci par exemple :

$\begin{array}{|c|c|c|c|c|}
\hline
0 & 2 & 4 & 1 & 3 \\ \hline
1 & 3 & 0 & 2 & 4 \\ \hline
2 & 4 & 1 & 3 & 0 \\ \hline
3 & 0 & 2 & 4 & 1 \\ \hline
4 & 1 & 3 & 0 & 2 \\ \hline
\end{array}$

\subsection{Solution pour $n > 5$}

Pour $n = 6$, il n'y a pas de solution. Pour $n = 7$ en voici une :

$\begin{array}{|c|c|c|c|c|c|c|}
\hline
0 & 2 & 4 & 6 & 1 & 3 & 5 \\ \hline
1 & 3 & 5 & 0 & 2 & 4 & 6 \\ \hline
2 & 4 & 6 & 1 & 3 & 5 & 0 \\ \hline
3 & 5 & 0 & 2 & 4 & 6 & 1 \\ \hline
4 & 6 & 1 & 3 & 5 & 0 & 2 \\ \hline
5 & 0 & 2 & 4 & 6 & 1 & 3 \\ \hline
6 & 1 & 3 & 5 & 0 & 2 & 4 \\ \hline
\end{array}$

\subsection{Conclusion}
Nous remarquons que ces solutions répètent les mêmes motifs d'une ligne sur l'autre et que par conséquent, elles doivent être étendables à des grilles plus grandes.



