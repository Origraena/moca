%ex4.tex

\section{Exercice 4 - Coloration des Sommets}\label{ex4}
Montrons que : $G = (V,E)$ est biparti $\leftrightarrow G$ est 2-colorable.\\
Un graphe biparti est d\'efini de la mani\`ere suivante :
\begin{itemize}
	\item $V = V_1 \cup V_2$ avec $V_1 \cap V_2 = \emptyset$,
	\item $\forall x_1,y_1 \in V1 (x_1,y_1) \notin E$,
	\item $\forall x_2,y_2 \in V2 (x_2,y_2) \notin E$.
\end{itemize}

$\rightarrow$ Soit $G = (V = V_1 \cup V_2,E)$.\\
Soit la coloration $C$ suivante :
\begin{itemize}
	\item $C(x) = 1 \forall x \in V_1$,
	\item $C(x) = 2 \forall x \in V_2$.
\end{itemize}
Par d\'efinition d'un graphe biparti, cette 2-coloration est bien valide (puisque il
n'existe aucune ar\^ete \`a l'int\'erieur de chaque partition).

$\leftarrow$ Soit $G = (V,E)$ un graphe 2-colorable.\\
Soit $C$ une 2-coloration valide de ses sommets.\\
On pose $V_1 = \{x \in V : C(x) = 1\}$ et $V_2 = \{x \in V : C(x) = 2\}$.
Puisque cette coloration est valide $\forall x,y \in V$ si $C(x) = C(y)$ alors
$(x,y) \notin E$.\\
Ainsi on peut noter $V = V_1 \cup V_2$, leur intersection est bien vide (puisque un
m\^eme sommet ne peut pas avoir deux couleurs diff\'erentes) et il n'existe aucune
ar\^ete ayant ses deux extr\'emit\'es dans le m\^eme sous-ensemble de sommets.\\
$G$ est donc un graphe biparti.

