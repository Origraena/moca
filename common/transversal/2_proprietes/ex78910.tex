% !TEX root = ../main.tex
%ex6.tex

%\section{Exercice 7 - Coloration des Sommets}\label{ex7}
%
%\subsection{Montrons que $\chi(G) \times \chi(\overline{G}) \geq n$}
%Soit $\{s_1,\ldots,s_k\}$ une partition minimale de $G$ en stables c'est à dire $k=\chi(G)$.
%Les $s_i$ sont des cliques de $G$ d'où $|s_i| \leq \chi(G)$. Donc $n = \sum\limits_{1}^{k}{|s_i|} \leq k \times \chi(G) = \chi(G) \times \chi(\overline{G}) $.
%
%\subsection{Montrons que $\chi(G) + \chi(\overline{G}) \leq n + 1$ (par récurrence)}
%Le résultat est trivialement vrai pour $n=1$ ou $2$ (voire même $3$). Supposons $n > 2$. Soit $x_0$ un sommet quelconque de $G$. Notons $G_0 = G\backslash \{x_0\}$. On a $\chi(G) \leq \chi(G_0) + 1$ et $\chi(\overline{G}) \leq \chi(\overline{G}_0) + 1$ (voir figure 1). Si ces inégalités sont des égalités, on a en fait $d_G(x_0) \geq \chi(G_0)$ et  $d_\overline{G}(x_0) \geq \chi(\overline{G}_0)$.
%
%
