%ex5.tex

\section{Exercice 5 - Coloration des Sommets}\label{ex5}
Soit $A$ le plus grand stable d'un graphe quelconque $G$ \`a n sommets.
On note $\alpha(G) = |A|$. \\
Montrons que $\lceil\frac{n}{\alpha(G)}\rceil \leq \chi(G) \leq n - \alpha(G) + 1$.
On pose (i) la premi\`ere in\'egalit\'e et (ii) la seconde.

Montrons (i) par r\'ecurrence sur la taille $\alpha(G)$ de A.\\
Si $\alpha(G) = 1$, on a : $\lceil\frac{n}{1}\rceil = n \leq \chi(G) \leq n$.\\
On suppose que (i) est vraie pour $\alpha(G) = c$ et on cherche \`a prouver que c'est
toujours le cas pour $\alpha(G) = c+1$.
Il est \'evident que $\lceil\frac{n}{c+1}\rceil \leq \lceil\frac{n}{c}\rceil$,
or par hypoth\`ese on a : $\lceil\frac{n}{c}\rceil \leq \chi(G)$.\\
(i) est donc v\'erifi\'ee.

Montrons maintenant (ii) :\\
On sait que $\chi(G) \leq n$ (il suffit d'attribuer une couleur diff\'erente \`a chaque
sommet).\\
Soit $G' = G \setminus A$, on voit que le nombre de sommets de $G'$ est \'egal \`a $n -
\alpha(G)$, et on en d\'eduit que $\chi(G') \leq n - \alpha(G)$.
$A$ \'etant stable, on a $\chi(A) = 1$.\\
De plus, $\chi(G) \leq \chi(G') + \chi(A)$, et donc on peut conclure :\\
$\chi(G) \leq n - \alpha(G) + 1$.


