%ex22.tex

\section{Exercice 22 - Homomorphisme et Satisfaction de Contraintes}\label{ex22}
Ici, nous nous int\'eressons \`a la r\'eduction polynomiale du probl\`eme de l'existence
d'une solution d'un r\'eseau de contraintes (Constraint Satisfaction Problem)
vers le probl\`eme de recherche d'homomorphisme dans un graphe (HOM).

Soit le CSP suivant : C = (X,D,C,R) avec :
\begin{itemize}
	\item $X = \{x_i : \forall i \in [1,n]\}$ l'ensemble des variables de C,
	\item $D = \cup_{i \in [1,n]}D_i$ o\`u chaque $D_i$ est le domaine de la variable
	$x_i \in X$,
	\item $C = \{C_j : \forall j \in [1,p]\}$ o\`u chaque $C_j$ est une contrainte
	d\'efinie par un ensemble ordonn\'e de variables $x_{ji} \in X_j \in X \forall i \in
	[1,q]$,
	\item $R = \{R_j : \forall j \in [1,q]\}$ o\`u $R_j \in \{\times_{i \in [1,q]}D_{ji}\}$ est
	la d\'efinition de la contrainte $C_j$.
\end{itemize}
R\'esoudre $C$ revient \`a d\'eterminer si ce r\'eseau est consistant.

Nous allons construire deux graphes bipartis diff\'erents $G_{X} = (V_X,E_X,\omega_X)$ (la
"requ\^ete") et $G_R = (V_R,E_R,\omega_R)$ (la base de connaissance).

Soit $G_X$ d\'efini comme suit :
\begin{itemize}
	\item $V_X = \{x_i \in X\} \cup \{C_j \in C\}$,
	\item $E_X = \{(x_i,C_j) : x_i$ est une variable apparaissant dans $C_j \}$,
	\item $\omega_X : E_X \rightarrow N$ une fonction d\'efinie telle que
	$\omega_X(x_i,C_j) = p$ avec $p$ la position de $x_i$ dans $C_j$.
\end{itemize}

Soit $G_R$ d\'efini comme suit :
\begin{itemize}
	\item $V_R = \{D_{ji} \in R\} \cup \{C_j \in C\}$,
	\item $E_R = \{(D_{ji},C_j)\}$,
	\item $\omega_R : E_R \rightarrow N$ une fonction d\'efinie telle que
	$\omega_R(D_{ji},C_j) = q$ avec $q$ la position de $D_{ji}$ dans $C_j$.
\end{itemize}

$C$ est consistant $\leftrightarrow$ $\exists h$ un homomorphisme de $G_X$ dans $G_R$.

Preuve ?
Bijection entre h et la fonction solution de C...

