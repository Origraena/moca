\documentclass[a4paper, 11pt, fleqn]{article}

\usepackage[utf8x]{inputenc}
\usepackage[T1]{fontenc}
\usepackage[french]{babel}
\usepackage{graphicx}
\usepackage{moreverb}
\usepackage{subfigure}
\usepackage{pdfpages}
\usepackage{eurosym}
\usepackage{fullpage}
\usepackage{amsmath}
\usepackage{bbm}
\usepackage{amssymb}
\usepackage{tikz}

\newcommand{\ssi}[0]{si et seulement si }
\newcommand{\theoreme}[1]{\textbf{\underline{Théorème :}} \\ \begin{center}\parbox{0.90\textwidth}{#1}\end{center}}
\newcommand{\definition}[1]{\textbf{\underline{Définition :}}\\\begin{center}\parbox{0.90\textwidth}{#1}\end{center}}
\newcommand{\proposition}[1]{\textbf{\underline{Proposition :}}\\\begin{center}\parbox{0.90\textwidth}{#1}\end{center}}
\newcommand{\corollaire}[1]{\textbf{\underline{Corollaire :}}\\\begin{center}\parbox{0.90\textwidth}{#1}\end{center}}
\newcommand{\lemme}[1]{\textbf{\underline{Lemme :}}\\\begin{center}\parbox{0.90\textwidth}{#1}\end{center}}
\newcommand{\remarque}[0]{\textbf{\underline{Remarque :}}\\}
\newcommand{\qos}[0]{Qualité de Service }
\newcommand{\inve}[1]{\frac{1}{#1}}
\newcommand{\fa}[0]{file d'attente }
\newcommand{\fas}[0]{files d'attente }
\newcommand{\VA}[0]{variable aléatoire }
\newcommand{\VAs}[0]{variables aléatoires }
\newcommand{\Prob}[0]{\mathbb{P} }

\begin{document}

\section{Exercice 4}

% Fig Exo 4

\subsection{Débit d'arrivée et stabilité}

Pour chacune des files du système, on cherche le débit d'arrivée de la file:
\begin{itemize}
	\item \textbf{File 1}, débit = $\lambda$
	\item \textbf{File 2}, débit = $\lambda p_{1,2}$
	\item \textbf{File 3}, débit = $\lambda p_{1,3}$
	\item \textbf{File 4}, débit = $\lambda (p_{1,2} + p_{1,3} + p_{1,4})$
\end{itemize}

Un système est dit stable si le débit d'arrivée est inférieur au temps de traitement, on obtient
alors ce qui suit : $$
\left \lbrace \begin{array}{rcl}
	\lambda &<& \mu_1 \\
	\lambda p_{1,2} &<& \mu_2 \\
	\lambda p_{1,3} &<& \mu_3 \\
	\lambda (p_{1,2} + p_{1,3} + p_{1,4}) &<& \mu_4 \\
\end{array} \right.
\qquad \Longrightarrow \qquad \lambda < \min (\mu_1, \frac{\mu_2}{p_{1,2}}, \frac{\mu_3}{p_{1,4}}, \mu_4) $$

\subsection{Application Numérique}

On a : $$
\inve{\mu_1} = 10 \mbox{ ms}, \quad \inve{\mu_2} = 100 \mbox{ ms}, \quad \inve{\mu_3} = 100 \mbox{
ms}, \quad \inve{\mu_4} $$

Il faut faire attention attention aux unités! On obtient donc les valeurs suivantes : $$
\begin{array}{rcccl}
	\mu_1 & = & \displaystyle \inve{10. 10^{-3}} & = & 100 \\~\\
	\mu_2 & = & \displaystyle \inve{100. 10^{-3}} & = & 10 \\~\\
	\mu_3 & = & \displaystyle \inve{200. 10^{-3}} & = & 5 \\~\\
	\mu_4 & = & \displaystyle \inve{50. 10^{-3}} & = & 20 \\~\\
	p_{1,2} & = & 1 - 0,25 & = & 0,75 \\
	p_{1,3} & = & 0,25 \times 0,6 & = & 0,15 \\
	p_{1,4} & = & 0,25 \times 0,4 & = & 0,1 \\
\end{array} $$

Après calcul on obtient : $$
\lambda < min (100; 100; 33,33; 20) \Longrightarrow \lambda \leq 19 $$

\subsection{Temps de réponse moyen}

Le temps de réponse moyen est donné par : $$
T = \inve{X} \sum_{i=1}^{n} \frac{\lambda_i}{\mu_i - \lambda_i} $$

Ce qui donne une valeur de $165$ ms.

\section{Exercice 6}


\end{document}
