\documentclass[a4paper, 11pt, fleqn]{article}

\usepackage[utf8x]{inputenc}
\usepackage[T1]{fontenc}
\usepackage[french]{babel}
\usepackage{graphicx}
\usepackage{moreverb}
\usepackage{subfigure}
\usepackage{pdfpages}
\usepackage{eurosym}
\usepackage{fullpage}
\usepackage{amsmath}
\usepackage{bbm}
\usepackage{amssymb}
\usepackage{tikz}
\usetikzlibrary{shapes}

\newcommand{\ssi}[0]{si et seulement si }
\newcommand{\theoreme}[1]{\textbf{\underline{Théorème :}} \\ \begin{center}\parbox{0.90\textwidth}{#1}\end{center}}
\newcommand{\definition}[1]{\textbf{\underline{Définition :}}\\\begin{center}\parbox{0.90\textwidth}{#1}\end{center}}
\newcommand{\proposition}[1]{\textbf{\underline{Proposition :}}\\\begin{center}\parbox{0.90\textwidth}{#1}\end{center}}
\newcommand{\corollaire}[1]{\textbf{\underline{Corollaire :}}\\\begin{center}\parbox{0.90\textwidth}{#1}\end{center}}
\newcommand{\lemme}[1]{\textbf{\underline{Lemme :}}\\\begin{center}\parbox{0.90\textwidth}{#1}\end{center}}
\newcommand{\remarque}[0]{\textbf{\underline{Remarque :}}\\}
\newcommand{\qos}[0]{Qualité de Service }
\newcommand{\inve}[1]{\frac{1}{#1}}
\newcommand{\fa}[0]{file d'attente }
\newcommand{\fas}[0]{files d'attente }
\newcommand{\VA}[0]{variable aléatoire }
\newcommand{\VAs}[0]{variables aléatoires }
\newcommand{\Prob}[0]{\mathbb{P} }

% Commandes tikz
\newcommand{\openrect}[4]{\draw (#1, #2) -- (#3, #2);
\draw (#3, #2) -- (#3, #4);
\draw (#3, #4) -- (#1, #4);}

\begin{document}

\section{Exercice 4}

% Fig Exo 4

\begin{tikzpicture}
	\tikzset{fleche/.style={->, thick, >=latex}, 
	noeud/.style={ellipse, draw=black}};

	\openrect{1}{0.5}{3}{-0.5}
	\openrect{6}{2.5}{8}{1.5}
	\openrect{6}{-2.5}{8}{-1.5}
	\openrect{11}{0.5}{13}{-0.5}

	\node (finit) at ( 0,  0) {};
	\node (fend) at ( 2,  0) {};
	\node (midr1) at ( 4,  0) {};
	\node (midr2) at ( 9,  2) {};
	\node (midr3) at ( 9, -2) {};
	\node (midr4) at (14,  0) {};
	\node (midr5) at (11,  0) {};
	\node (emu1a) at ( 6,  2) {};
	\node (emu1b) at ( 6, -2) {};
	\node[noeud] (mu1) at ( 5,  0) {$\mu_1$};
	\node[noeud] (mu2) at (10,  2) {$\mu_2$};
	\node[noeud] (mu3) at (10, -2) {$\mu_3$};
	\node[noeud] (mu4) at (15.5, 0){$\mu_4$};

	\draw[fleche] (finit) to (fend);
	\draw[fleche] (midr1) to (mu1);
	\draw[fleche] (mu1) to[out=90,in=180] (emu1a);
	\draw[fleche] (mu1) to[out=270,in=180] (emu1b);
	\draw[fleche] (mu1) to (midr5);


\end{tikzpicture}

\subsection{Débit d'arrivée et stabilité}

Pour chacune des files du système, on cherche le débit d'arrivée de la file:
\begin{itemize}
	\item \textbf{File 1}, débit = $\lambda$
	\item \textbf{File 2}, débit = $\lambda p_{1,2}$
	\item \textbf{File 3}, débit = $\lambda p_{1,3}$
	\item \textbf{File 4}, débit = $\lambda (p_{1,2} + p_{1,3} + p_{1,4})$
\end{itemize}

Un système est dit stable si le débit d'arrivée est inférieur au temps de traitement, on obtient
alors ce qui suit : $$
\left \lbrace \begin{array}{rcl}
	\lambda &<& \mu_1 \\
	\lambda p_{1,2} &<& \mu_2 \\
	\lambda p_{1,3} &<& \mu_3 \\
	\lambda (p_{1,2} + p_{1,3} + p_{1,4}) &<& \mu_4 \\
\end{array} \right.
\qquad \Longrightarrow \qquad \lambda < \min (\mu_1, \frac{\mu_2}{p_{1,2}}, \frac{\mu_3}{p_{1,4}}, \mu_4) $$

\subsection{Application Numérique}

On a : $$
\inve{\mu_1} = 10 \mbox{ ms}, \quad \inve{\mu_2} = 100 \mbox{ ms}, \quad \inve{\mu_3} = 100 \mbox{
ms}, \quad \inve{\mu_4} $$

Il faut faire attention attention aux unités! On obtient donc les valeurs suivantes : $$
\begin{array}{rcccl}
	\mu_1 & = & \displaystyle \inve{10. 10^{-3}} & = & 100 \\~\\
	\mu_2 & = & \displaystyle \inve{100. 10^{-3}} & = & 10 \\~\\
	\mu_3 & = & \displaystyle \inve{200. 10^{-3}} & = & 5 \\~\\
	\mu_4 & = & \displaystyle \inve{50. 10^{-3}} & = & 20 \\~\\
	p_{1,2} & = & 1 - 0,25 & = & 0,75 \\
	p_{1,3} & = & 0,25 \times 0,6 & = & 0,15 \\
	p_{1,4} & = & 0,25 \times 0,4 & = & 0,1 \\
\end{array} $$

Après calcul on obtient : $$
\lambda < min (100; 100; 33,33; 20) \Longrightarrow \lambda \leq 19 $$

\subsection{Temps de réponse moyen}

Le temps de réponse moyen est donné par : $$
T = \inve{X} \sum_{i=1}^{n} \frac{\lambda_i}{\mu_i - \lambda_i} $$

Ce qui donne une valeur de $165$ ms.

\section{Exercice 6}

\subsection{Trafic engendré par une connexion}

Une connexion engendre un trafic descendant de $4,0 Mb.s^-1$ ce qui donne en cellules : $$
\frac{4.10^6}{8 \times 50} = 10. 10^3 s^{-1} $$. A raison d'une cellule envoyée toutes les 50
cellules reçues, le traffic montant s'estime de la sorte : $$
\frac{10.10^3}{50} = 200 $$

Le traffic total s'écrit alors : $$
10.10^3 + 200 = 10200 \mbox{ cellules}.s^{-1} $$

\subsection{Capacités des commutateurs}

Chacun des commutateurs présente une borne de capacité qui lui est propre, en ajoutant la borne de
capacité maximale du système on obtient le système d'inéquations suivant (en Mb puis en cellules) : $$
\left \lbrace \begin{array}{rcl}
	a \times 4,08 & \leq & 250 \\
	b \times 4,08 & \leq & 250 \\
	c \times 4,08 & \leq & 150 \\
	(a+b+c) 4,08 & \leq & 250 \\
\end{array} \right .
\Longrightarrow 
\left \lbrace \begin{array}{rcl}
	a \times 10200 & \leq & 625. 10^3 \\
	b \times 10200 & \leq & 625. 10^3 \\
	c \times 10200 & \leq & 375. 10^3 \\
	(a+b+c) 10200 & \leq & 625. 10^3 \\
\end{array} \right .
\Longrightarrow
\left \lbrace \begin{array}{rcl}
	a & \leq & 61\\
	b & \leq & 61 \\
	c & \leq & 36 \\
	(a+b+c) & \leq & 61 \\
\end{array} \right .
$$

\subsection{Temps moyen de réponse}

On a $\mu_1 = \mu_2 = \mu_E = 625.10^3$ et $\mu_3 = 375.10^-3$. On note $\hat{\lambda_1},
\hat{\lambda_2}, \hat{\lambda_3}, \hat{\lambda_E}$, le trafic total arrivant aux noeuds $A, B, C \mbox{ et
} E$. On a alors : $$
\begin{array}{rcl}
	\hat{\lambda_1} & = & a \lambda \\
	\hat{\lambda_2} & = & b \lambda \\
	\hat{\lambda_3} & = & c \lambda \\
	\hat{\lambda_E} & = & \lambda(a+b+c) \\
\end{array} $$

Soit $\lambda_0$ la valeur du traffic sur une route passant par $A$ et $E$, on a : $$
F_A = \inve{\lambda_0} \left ( \frac{\lambda_0}{625000 - 10200 a} + \frac{\lambda_0}{625000 -
10200(a+b+c)} \right . $$

Pour le temps de réponse moyen de tous les clients confondus, on calcule le traffic total : $$
\Lambda = \lambda (a+b+c) $$
D'où :$$
\bar T = \inve{\Lambda}\left ( \frac{\hat{\lambda_1}}{\mu_1 - \hat{\lambda_1}} +
\frac{\hat{\lambda_2}}{\mu_2 - \hat{\lambda_2}} + \frac{\hat{\lambda_3}}{\mu_3 - \hat{\lambda_3}} +
\frac{\hat{\lambda_E}}{\mu_E - \hat{\lambda_E}} \right ) $$
Ce qui implique : $$
\bar T_A = 17,53 \mu s \quad \mbox{ et } \quad F = 20,62 \mu s $$

\section{Exercice 5}

\subsection*{Multiplexage en fréquence}

On a $N$ file d'attente en parallèle. Chacune est alimentée par des clients de débit : $\lambda =
\theta $ et le temps de service moyen est donné ainsi : $$
\begin{array}{rcl}
	\displaystyle \inve{\mu} & = & \frac{L N}{C} \\~\\
	\mu & = & \displaystyle \frac{C}{LN}
\end{array} $$

On a alors : $$
\begin{array}{rcl}
	\displaystyle T_{Rf} & = & \inve{\mu - \lambda} \\
				 \displaystyle & = & \inve{C/LN} - \theta \\
				 \displaystyle & = & \frac{LN}{C - \theta LN} \\
\end{array}$$

\subsection*{Multiplexage statistique}

On a alors : $\lambda = N \theta$, le temps moyen de chaque client : $$
	\inve{\mu} = \frac{L}{C} \quad \Longrightarrow \mu = \frac{C}{L} $$
Avec le temps de réponse : $$
T_{Rs} = \inve{\mu - \lambda} = \inve{C/L - N\theta} = \frac{L}{C- \theta L N} = \inve{N} T_{Rf} $$

\end{document}
