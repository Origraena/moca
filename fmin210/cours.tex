\documentclass[a4paper, 11pt, fleqn]{article}

\usepackage[utf8x]{inputenc}
\usepackage[T1]{fontenc}
\usepackage[french]{babel}
\usepackage{graphicx}
\usepackage{moreverb}
\usepackage{subfigure}
\usepackage{pdfpages}
\usepackage{eurosym}
\usepackage{fullpage}
\usepackage{amsmath}
\usepackage{bbm}
\usepackage{amssymb}
\usepackage{tikz}

\newcommand{\ssi}[0]{si et seulement si }
\newcommand{\theoreme}[1]{\textbf{\underline{Théorème :}} \\ \begin{center}\parbox{0.90\textwidth}{#1}\end{center}}
\newcommand{\definition}[1]{\textbf{\underline{Définition :}}\\\begin{center}\parbox{0.90\textwidth}{#1}\end{center}}
\newcommand{\proposition}[1]{\textbf{\underline{Proposition :}}\\\begin{center}\parbox{0.90\textwidth}{#1}\end{center}}
\newcommand{\corollaire}[1]{\textbf{\underline{Corollaire :}}\\\begin{center}\parbox{0.90\textwidth}{#1}\end{center}}
\newcommand{\lemme}[1]{\textbf{\underline{Lemme :}}\\\begin{center}\parbox{0.90\textwidth}{#1}\end{center}}
\newcommand{\remarque}[0]{\textbf{\underline{Remarque :}}\\}
\newcommand{\qos}[0]{Qualité de Service }
\newcommand{\inve}[1]{\frac{1}{#1}}
\newcommand{\fa}[0]{file d'attente }
\newcommand{\fas}[0]{files d'attente }
\newcommand{\VA}[0]{variable aléatoire }
\newcommand{\VAs}[0]{variables aléatoires }
\newcommand{\Prob}[0]{\mathbb{P} }

% Page web www.lrimm.fr/~baert/Cours/FMIN210
% Cours :
%		* Files d'attentes - Réseaux de file d'attente (processus Markoviens, probabilité)
%		* Réseaux et services - Routage dans les réseaux (ospf et rip), Tolérance aux pannes, qualités
%		de service
% Création de modèles optimaux à partir de la théorie des files d'attente

% Objectif réaliser de la simulation de réseaux (simulation des différents protocoles [TCP - UDP -
% RED]. Langage utilisé TCL sur un logiciel : network simulator

% TPs notés, rapport d'expérience à réaliser (courbes, analyse de courbes, conclusions, critiques,
% ...)

\begin{document}

\part{Introduction à la \qos - Métrologie}

\section{Définitions}

\subsection{Débit :}

On a des évènements discrets qui arrivent selon un certain processus
d'arrivéequi suit une certaine loi à des dates successives\footnote{on est dans un principe de
discètisation des évènements}. Soient $ \displaystyle a_0, a_1, a_2, \dots, a_n $, les temps
d'arrivées, on va noter la $n^{ieme}$ interarrivée : $$
\tau_n = a_{n+1} - a_n $$
\begin{figure}[h!]
	\begin{center}
		\begin{tikzpicture}
			\draw[->] (-1, 0) -- (13,0) node [right] {$t$};
			\foreach \x in {0, 1, 2, 3} {
				\draw (3 * \x, 0.1) -- (3 * \x , -0.1) node[anchor = north] {$a_\x$};
				\draw[<->] (3*\x + 0.2, 0.3) -- node[above] {$\tau_{\x}$} (3*\x + 2.8, 0.3);
			}
			\draw (12, 0.1) -- (12 , -0.1) node[anchor = north] {$a_4$};
		\end{tikzpicture}
	\end{center}
	\caption{Evolution du temps}
\end{figure}

On notera $N(t) = $ le nombre d'arrivées avant $t$ : $$
N(t) = card \{ n\ /\ a_n \leq t \} $$
Le débit d'arrivée est défini par : $$
\theta = \lim_{y \rightarrow \infty} \frac{N(t)}{t} $$
C'est le nombre moyen d'arrivées par unité de temps.

On estime le débit d'arrivées par :
$$ \bar\theta = \frac{N(T)}{T} \mbox{pour T grand} $$
Le nombremoyen d'arrivées par unité de temps : $$
\bar\tau = \lim_{m \rightarrow \infty} \frac{1}{n} \sum_{i=0}^{n} \tau_i = \lim_{n \rightarrow
\infty} \frac{a_n}{n} $$
On peut estimer : $$
\bar\tau = \frac{a_N}{N} \quad \quad \mbox{pour N grand} $$
On voit alors que : $$
\theta = \frac{1}{\tau} \quad \mbox{ou} \quad \tau = \frac{1}{\theta} $$

\subsection{Débit d'information :} 

Chaque paquet a une quantité d'information donnée. On
définit alors le débit d'information comme le rapport de la quantité total d'information
transmise par unité de temps. Soit $\sigma_n$ la quantité d'information du paquet $n$. La
quantité totale d'information est : $$
\sum_{i=1}^{n} \sigma_i $$
La durée total est : $$
\sum_{i=1}^{n} \tau_i $$
On définit alors $\bar\delta$, le débit moyen d'information, ainsi : $$
\bar\delta = \lim_{n \rightarrow \infty} \frac{\sum_{i=1}^{n} \sigma_i}{\sum_{i=1}^n \tau_i}
$$
On peut définir la quantité moyenne d'information d'un paquet : $$
\bar\sigma = \lim_{n \rightarrow \infty} \frac{\sum_{i=1}^n \sigma_i}{n} $$
On peut alors écrire ce qui suit : $$
\bar\delta = \bar\theta \times \bar\sigma $$
Ce qui signifie que le débit d'information ($bit.s^{-1}$) est le produit du débit
($paquet.s^{-1}$) par la taille moyenne des paquets ($bit$).

\subsection{Débit crête :} 

Le débit est irrégulier dans la plupart des systèmes. On considère
des processus de type ON/OFF, soit $\thêta '$ le débit crête, on a :

\begin{figure}[h!]
	\begin{center}
		\begin{tikzpicture}
			\draw (-1, 0) -- (0, 0) ;
			\foreach \x in {1, 2, 3} {
				\foreach \y/\ytext/\ytest in {0/X/ON, 1/Y/OFF} {
					\draw (4*\x + 2*\y - 4, 0.1) --  (4*\x + 2*\y -4, -0.1);
					\draw (4*\x + 2*\y - 4, 0)  -- node[above] {\ytest} node[below] {$\ytext_\x$} (4*\x + 2*\y -2, 0);
				}
			}
			\draw (12, 0.1) -- (12, -0.1);
			\draw[->] (12, 0) -- (13, 0);
		\end{tikzpicture}
	\end{center}
	\caption{Processus de type ON/OFF}
\end{figure}

On appelle : $ X_i = $ périodes d'activité, $Y_i = $ périodes d'inactivité, $\bar X = $ période
moyenne d'activité et $\bar Y$ période moyenne d'inactivité. On peut alors définir le débit
moyen comme : $$
\bar\theta = \theta ' \times \frac{\bar X}{\bar X + \bar Y} $$

\subsection{Lois de conservation de débit}

La loi est valable uniquement dans un système stable : à savoir aucun paquet ne se perd dans le
réseau.\\

\begin{figure}
	\begin{center}
		\begin{tikzpicture}
			\draw[->] (0, 2) -- node[above] {$\lambda_1$} (3, 0.3);
			\draw[->] (0, -2) -- node[below] {$\lambda_2$} (3, -0.3);
			\node [circle, draw = black, minimum size=12mm] at (3.5, 0) {S};
			\draw[->] (4.1, 0) -- node[above] {$\lambda_1 + \lambda_2$} node[below] {$\lambda_3$} (7, 0);
		\end{tikzpicture}
	\end{center}
	\caption{Conservation du débit (1)}
\end{figure}
\begin{figure}
	\begin{center}
		\begin{tikzpicture}
			\draw[->] (0, 0) -- node[above] {$\lambda$} (2.9, 0);
			\node [circle, draw = black, minimum size=12mm] at (3.5, 0) {S};
			\draw[->] (4.1, 0.3) -- node[above] {$\lambda_p$} (7, 2);
			\draw[->] (4.1, -0.3) -- node[below] {$1 - \lambda_p$} (7, -2);
		\end{tikzpicture}
	\end{center}
	\caption{Conservation du débit (2)}
\end{figure}
\begin{figure}
	\begin{center}
		\begin{tikzpicture}
			\node at (0,1) { };
			\draw[->] (0, -1) -- node[above] {$\lambda$} (2.9, -1);
			\node [circle, draw = black, minimum size=12mm] at (3.5, -1) {S};
			\draw[->] (4.1 , -1) -- node[above] {$\mu$} node[below] {$\lambda = \mu$} (7, -1);
		\end{tikzpicture}
	\end{center}
	\caption{Conservation du débit (3)}
\end{figure}

%fig 3, 4, 5
\subsubsection*{Un petit exercice}

Soit le réseau présenté sur la figure \ref{ex}. \\

\begin{figure}
	\begin{center}
		\begin{tikzpicture}
			\draw (0, 0) -- node[above]{$\lambda$} (2, 0) ;
			\draw[->] (2, 0) -- node[below]{$\lambda '$} (3, 0);
			\draw (3, -0.5) rectangle (6, 0.5);
			\node at (4.5, 0) {S};
			\draw (6, 0) -- node[below] {$\lambda '$} (7, 0) ;
			\draw[->] (7, 0) -- node[above]{$(1-p) \lambda '$} (9, 0);
			\draw (7, 0) -- (7, 1.5);
			\draw (7, 1.5) -- node[above] {$p - \lambda '$} (2, 1.5);
			\draw[->] (2, 1.5) -- (2, 0);
		\end{tikzpicture}
	\end{center}
	\label{ex}
	\caption{Exercice : petit réseau}
\end{figure}

Calculer $\lambda '$.\\

Par a loi de conservation des débits, on a : $$
\begin{array}{rccl}
	& ( 1 - p ) \lambda ' &=& \lambda \\
	\Longrightarrow & \lambda ' & =& \displaystyle \frac {\lambda}{1 - p} \\
\end{array} $$

\subsection{Temps de réponse}

$a_n$ est le début d'activité de la $n^e$ activité. On note $d_n$ la date de fin de la $n^e$
activité. Le temps de réponse est alors défini par : $$
r_n = d_n - a_n $$
Le temps de réponse moyen est défini par : $$
\bar r = \lim_{n \rightarrow \infty} \inve{n} \sum_{i=1}^n r_i$$
qui peut aussi être défini ainsi : $$
\bar r_N = \inve{N} \sum_{i=1}^N r_i \quad \mbox{pour N grand}$$
On peut alors calculer la variance de $r$ : $$
var (r) = \inve{N-1} \sum_{i=1}^n (r_m - \bar r_N) ^ 2$$
L'objectif est alors de minimiser la variance.

\subsection{Taux d'utilisation}

Reconsidérons les processus ON/OFF. On définit le taux d'utilisation $u$ par : $$
u = \frac{\bar X}{\bar X + \bar Y} $$
$u$ est alors le taux d'utilisation du système. En pratique, il faut estimer le temps d'utilisation.
On comptabilise la durée totale d'activité : $$
\bar u_T = \inve{T} \times D \quad D \in [0, T], D \mbox{ étant la durée d'activité } $$

Où $\bar u_T$ est le taux moyen d'utilisation sur la période $T$.

\subsection{Taux de perte}

Le taux de perte définit la proportion de requêtes qui ont échoué, on peut aussi le voir comme la
probabilité qu'une requête quelconque échoue. Considérons un système de traitement qui peut perdre
des paquets, il y a alors 2 flux de données : 
\begin{itemize}
	\item Celui des paquets perdus
	\item Celui des paquets servis ou acceptés\footnote{appelé débit efficace}
\end{itemize}

La somme des deux est appelé débit offert. Le taux de perte $\Pi_p$ est donné par : $$
\Pi_p = \frac{\lambda - \lambda_{eff}}{\lambda}$$ $$
\Longrightarrow 1 - \Pi_p = \frac {\lambda_{eff}}{\lambda} $$

\section{Description des files d'attente}

\subsection{Caractéristiques communes}

On peut définir un modèle de \fa par ce qui sit :
\begin{enumerate}
	\item des entités défilent dans le système
	\item utilisation de ressources
	\item des comportements pas complètement prévisibles
\end{enumerate}

Une \fa est constituée :
\begin{itemize}
	\item d'un ou plusieurs serveurs
	\item de clients demandant un service
	\item d'un lieu où les clients attendent d'être servis lorsqu'aucun serveur n'est disponible
	\item d'arrivées étant des processus stochastiques\footnote{Il s'agit de modèles probabilistes dont on
		connaît les distributions : moyenne, variances, écart-type, ...}
	\item de processus de renouvellement des populations\footnote{Une \VA modèlisant les arrivées}
	\item durée de services \footnote{Aussi définie par une \VA}
\end{itemize}

\begin{figure}
	\begin{center}
		\begin{tikzpicture}
			\draw[->] (0, 0) -- node[above]{Arrivées} node[below] {$a_1, a_2, \dots, a_n$} (3, 0);
			\draw (3, 0.5) -- ( 7, 0.5 );
			\draw (3, -0.5) -- ( 7, -0.5 );
			\draw (7, 0.5) -- (7, -0.5);
			\node at (5, 0) {Lieu Attente};
			\draw (7, 0) -- (10, 0);
			\node [circle, draw=black, minimum size=10mm] at (10.5, 0){S};
			\node [below=5mm] at (10.5, 0) {Serveurs};
			\draw[->] (11, 0) -- node[above] {Départs} node[below]{$d_1, d_2, \dots, d_n$} (14, 0);
		\end{tikzpicture}
	\end{center}
	\caption{Une file d'attente}
\end{figure}


\subsection{Notation de Kendall}

$A/S/P/K/D$

Un peu d'éclaircissement :
\begin{itemize}
	\item $A$ : décrit la loi des arrivées
	\item $S$ : décrit la loi des services
	\item $P$ : nombre de serveurs (par défaut $P=1$)
	\item $K$ : capacité de traitement du système (i.e. le nombre maximum de clients $K = \infty$)
	\item $D$ : politique de services (FIFO, LIFO, Priorités\footnote{LIFO et Priorités sont
			préemptifs (i.e. si un client est servi et qu'un autre arrive avec une priorité supérieure, le
		premier client sort de la file d'attente)})
\end{itemize}

\subsubsection{Loi d'inter-arrivée et processus de Poisson}

La loi des inter-arrivées peut être une loi déterministe ($D$), on observe en général ce cas lorsque le
processus est périodique. 

Elle peut être une loi d'Erlang ($E_m$), elle est alors définie ainsi :
$$ Q(N) = \frac{\frac{x^N}{\mu^N N!}}{1 + \frac{x}{\mu} + \dots + \frac{x^N}{\mu^N N!}} $$
$\inve{\mu}$ est alors la durée moyenne d'un appel\footnote{Cette loi ressemble à un développement
	limité d'exponentielle, les lois exponentielles sont en général utilisées pour les évènements sans
	mémoire : l'évènement reçu à un temps $t$ est indépendant du précédent et n'influera pas sur le
suivant}. $\inve{x}$ représente le temps moyen entre 2 appels, et $Q(N)$ est la probabilité qu'un
appel trouve toutes les lignes occupées. 

Elle peut être de distribution générale ($G$), elle englobe tous les cas.

La loi qui nous intéresse est notée $M$ pour markov, il s'agit d'une loi de distribution
exponentielle, dans ce cas le processus d'arrivée est un processus de Poisson.

Considérons une suite de \VAs : $$
0 \leq t_1 < t_2 < \dots < t_n < t_{n+1} < \dots $$
La \VA enregistre la date d'occurence du $n^e$ évènement dans une expérience\footnote{On peut par
exemple considérer la date d'arrivée du $n^e$ paquet dans un réseau}. Ce processus est un processus
stochastique de Poisson avec un taux (paramètre) de $\lambda > 0$ si :
\begin{enumerate}
	\item $\Prob (\mbox{ un seul évènement dans un intervalle h }) = \lambda \times h + o(h)$
	\item $\Prob (\mbox{ plus d'un évènement durant h }) = O(h)$
	\item Le nombre d'évènements qui arrivent dans des intervalles qui ne chevauchent pas sont
		indépendants les uns des autres
\end{enumerate}

La probabilité d'avoir $n$ arrivées dans un intervalle de durée $h$ est alors donnée par : $$
\begin{array}{lcr}
	\Prob_n(h) &= &\Prob (N (h) = n) \\
						 			&=& \frac{(\lambda h) ^ n}{n !} e^{-\lambda h}
\end{array} $$

La probabilité nous sert juste à calculer le nombre moyen d'arrivées dans un intervalle de temps $h$
qui nous est donnée par : $$
\mathbb{E} (N(h)) = \lambda \times h $$

Intuitivement on comprends alors que le nombre moyen d'arrivées par unité de temps est $\lambda$. La
probabilité que la durée d'attente séprant deux arrivées régies par un processus de Poisson soit
inférieure à $x$ est donnée par : $$
\Prob(\tau \leq x) = 1 - e^{-\lambda x} $$

Lors d'arrivées régies par une loi de Poisson, la loi d'inter-arrivée suit une loi exponentielle.

\subsection{Evolution de la file d'attente}

On appelle $N(t)$ : le nombre de clients par unités de temps. Considérons le tableau des processus
de la figure \ref{tab}.

% fig 8
% fig 9
% fig 10

\begin{figure}
	\begin{center}
	\begin{tabular}{|c|c|c|c|}
		\hline
		Num. Client & Arrivée & Tps Service & Prio \\
		\hline
		1 & 0 & 4 & \\
		\hline 
		2&2&8&\\
		\hline 
		3&6&4& \\
		\hline 
		4&8&4& \\
		\hline 
		5 & 15 & 2 & \\
		\hline
	\end{tabular}
	\end{center}
	\caption{Tableau des clients}
	\label{tab}
\end{figure}

On prend la courbe de charge et on regarde le nombre de clients par unités de temps.

\begin{figure}
	\begin{center}
		\begin{tikzpicture}
			% Dessin des axes
			\draw[->] (-1, 0) -- (12, 0) node[right] {$t$};
			\draw[->] (0, -1) -- (0, 6) node[above] {Nb Clients};
			\foreach \x in {2, 4, 6, 8, 10, 12, 14, 16, 18, 20, 22} {
				\draw (0.5 * \x, 0.1) -- (0.5 * \x, -0.1) node[below] {\x};
			}
			\foreach \x in {1, 2, 3, 4, 5 } {
				\draw (0.1, \x) -- (-0.1, \x) node[left] {\x};
			}
			\node[below] at (-0.3, 0) {0};

			\coordinate (A) at (0, 1);
			\coordinate (B) at (1, 1);
			\coordinate (C) at (1, 2);
			\coordinate (D) at (2, 2);
			\coordinate (E) at (2, 1);
			\coordinate (F) at (3, 1);
			\coordinate (G) at (3, 2);
			\coordinate (H) at (4, 2);
			\coordinate (I) at (4, 3);
			\coordinate (J) at (6, 3);
			\coordinate (K) at (6, 2);
			\coordinate (L) at (7.5, 2);
			\coordinate (M) at (7.5, 3);
			\coordinate (N) at (8, 3) ;
			\coordinate (O) at (8, 2) ;
			\coordinate (P) at (10, 2);
			\coordinate (Q) at (10, 1);
			\coordinate (R) at (11, 1);
			\coordinate (S) at (11, 0);

			\draw (A) -- (B) -- (C) -- (D) -- (E) -- (F) -- (G) -- (H) -- (I) -- (J) -- (K) -- (L) -- (M) -- (N) -- (O) -- (P) -- (Q) -- (R) -- (S);
			\draw[<->] (0, -1) -- node[above] {1} (4, -1);
			\draw[<->] (1, -1.5) -- node[above] {2} (6, -1.5);
			\draw[<->] (3, -2) -- node[above] {3} (8, -2);
			\draw[<->] (4, -2.5) -- node[above] {4} (10, -2.5);
			\draw[<->] (7.5, -1) -- node[above] {5} (11, -1);
		\end{tikzpicture}
	\end{center}
	\caption{Nombre de clients en fonction du temps en mode FIFO}
\end{figure}

\begin{figure}
	\begin{center}
		\begin{tikzpicture}[scale=0.7]
			% Dessin des axes
			\draw[->] (0, 0) -- (17, 0) node[right] {$t$};
			\draw[->] (0, 0) -- (0, 6) node[above] {Nb Clients};
			\foreach \x in {2, 4, 6, 8, 10, 12, 14, 16, 18, 20, 22, 24, 26, 28, 30, 32} {
				\draw (0.5 * \x, 0.1) -- (0.5 * \x, -0.1) node[below] {\x};
			}
			\foreach \x in {1, 2, 3, 4, 5 } {
				\draw (0.1, \x) -- (-0.1, \x) node[left] {\x};
			}
			\node[below] at (-0.3, 0) {0};

			\coordinate (A) at (0, 1);
			\coordinate (B) at (1, 1);
			\coordinate (Z) at (1, 2);
			\coordinate (C) at (3, 2);
			\coordinate (D) at (3, 3);
			\coordinate (E) at (4, 3);
			\coordinate (F) at (4, 4);
			\coordinate (G) at (6, 4);
			\coordinate (H) at (6, 3);
			\coordinate (I) at (7.5, 3);
			\coordinate (J) at (7.5, 4);
			\coordinate (K) at (8.5, 4);
			\coordinate (L) at (8.5, 3);
			\coordinate (M) at (10.5, 3);
			\coordinate (N) at (10.5, 2) ;
			\coordinate (O) at (14.5, 2) ;
			\coordinate (P) at (14.5, 1);
			\coordinate (Q) at (16.5, 1);
			\coordinate (R) at (16.5, 0);

			\draw (A) -- (B) -- (Z) -- (C) -- (D) -- (E) -- (F) -- (G) -- (H) -- (I) -- (J) -- (K) -- (L) -- (M) -- (N) -- (O) -- (P) -- (Q) -- (R);
			\draw[<->] (0, -1) -- node[above] {1} (16.5, -1);
			\draw[<->] (1, -1.5) -- node[above] {2} (14.5, -1.5);
			\draw[<->] (3, -2) -- node[above] {3} (10.5, -2);
			\draw[<->] (4, -2.5) -- node[above] {4} (6, -2.5);
			\draw[<->] (7.5, -2.5) -- node[above] {5} (8.5, -2.5);
		\end{tikzpicture}
	\end{center}
	\caption{Nombre de clients en fonction du temps en mode LIFO sans mémoire de travail}
\end{figure}

\begin{figure}
	\begin{center}
		\begin{tikzpicture}
			% Dessin des axes
			\draw[->] (-1, 0) -- (12, 0) node[right] {$t$};
			\draw[->] (0, -1) -- (0, 6) node[above] {Nb Clients};
			\foreach \x in {2, 4, 6, 8, 10, 12, 14, 16, 18, 20, 22} {
				\draw (0.5 * \x, 0.1) -- (0.5 * \x, -0.1) node[below] {\x};
			}
			\foreach \x in {1, 2, 3, 4, 5 } {
				\draw (0.1, \x) -- (-0.1, \x) node[left] {\x};
			}
			\node[below] at (-0.3, 0) {0};

			\coordinate (A) at (0, 1);
			\coordinate (B) at (1, 1);
			\coordinate (C) at (1, 2);
			\coordinate (D) at (3, 2);
			\coordinate (E) at (3, 3);
			\coordinate (F) at (4, 3);
			\coordinate (G) at (4, 4);
			\coordinate (H) at (6, 4);
			\coordinate (I) at (6, 3);
			\coordinate (J) at (7, 3);
			\coordinate (K) at (7, 2);
			\coordinate (L) at (7.5, 2);
			\coordinate (M) at (7.5, 3);
			\coordinate (N) at (8.5, 3) ;
			\coordinate (O) at (8.5, 2) ;
			\coordinate (P) at (10, 2);
			\coordinate (Q) at (10, 1);
			\coordinate (R) at (11, 1);
			\coordinate (S) at (11, 0);

			\draw (A) -- (B) -- (C) -- (D) -- (E) -- (F) -- (G) -- (H) -- (I) -- (J) -- (K) -- (L) -- (M) -- (N) -- (O) -- (P) -- (Q) -- (R) -- (S);
			\draw[<->] (0, -1) -- node[above] {1} (11, -1);
			\draw[<->] (1, -1.5) -- node[above] {2} (10, -1.5);
			\draw[<->] (3, -2) -- node[above] {3} (7, -2);
			\draw[<->] (4, -2.5) -- node[above] {4} (6, -2.5);
			\draw[<->] (7.5, -2.5) -- node[above] {5} (8.5, -2.5);
		\end{tikzpicture}
	\end{center}
	\caption{Nombre de clients en fonction du temps en mode LIFO avec mémoire de travail}
\end{figure}

\begin{figure}
	\begin{center}
		\begin{tikzpicture}
			% Dessin des axes
			\draw[->] (-1, 0) -- (12, 0) node[right] {$t$};
			\draw[->] (0, -1) -- (0, 7) node[above] {Charge restante};
			\foreach \x in {2, 4, 6, 8, 10, 12, 14, 16, 18, 20, 22} {
				\draw (0.5 * \x, 0.1) -- (0.5 * \x, -0.1) node[below] {\x};
			}
			\foreach \x in {2, 4, 6, 8, 10, 12} {
				\draw (0.1, 0.5 * \x) -- (-0.1, 0.5 * \x) node[left] {\x};
			}
			\node[below] at (-0.3, 0) {0};

			\coordinate (A) at (0, 2);
			\coordinate (B) at (1, 1);
			\coordinate (C) at (1, 5);
			\coordinate (D) at (3, 3);
			\coordinate (E) at (3, 5);
			\coordinate (F) at (4, 4);
			\coordinate (G) at (4, 6);
			\coordinate (H) at (7.5, 2.5);
			\coordinate (I) at (7.5, 3.5);
			\coordinate (J) at (11, 0);

			\draw (A) -- (B) -- (C) -- (D) -- (E) -- (F) -- (G) -- (H) -- (I) -- (J);
		\end{tikzpicture}
	\end{center}
	\caption{Courbe de charge en mode FIFO}
\end{figure}

On définit la charge du système à l'instant $t$ comme la quantité de travail qui reste à effectuer
par le système à cet instant.

\subsection{Etude de la stabilité d'un système}

Le nombre moyen de client est donné par : $$
E (N) = \lim_{T \rightarrow \infty} \frac{1}{T} \int_0^T N(t) dt $$

On a aussi vu que le temps de réponse moyen est donné par : $$
\bar r = \lim_{n\rightarrow \infty} \frac{1}{n} \sum_{i=1}^n r_i $$

On peut alors écrire le \underline{théorème de Little} :

Pour tout système stable on a : $$
\bar N = \lambda \bar r \quad \mbox{où } \lambda \mbox{ est le débit et } \bar r \mbox{ est le temps
de réponse moyen. }$$

Si le serveur est occupé avec une robabilité de $u$ l'utilisation moyenne (dans le cas d'un système ON/OFF), la formule de Little devient : $$
\bar u = \lambda \bar r $$

Comme $\bar u \in [0, 1]$, le débit ne peut être supérieur à $\frac{1}{\bar r}$, on appelle alors
cette valeur, la capacité de traitement et on la note $\mu$.

%fig 12

Si $\lambda < \mu$ on a : \begin{itemize}
	\item système stable
	\item le débit de sortie est exactement $\lambda$
	\item le taux d'utilisation est alors : $ u = \frac{\lambda}{\mu} $
\end{itemize}

Si $\lambda > \mu$ on a : \begin{itemize}
	\item le système est instable
	\item $N(t) \underset{t \rightarrow \infty}{\longrightarrow} \infty$
	\item le débit de sortie est $\mu$
	\item $\bar u = 1$
\end{itemize}

\subsection{Etudes des files M/M/1}

Ce genre de file permet de modéliser des serveurs ftp et web. Commençons par en énoncer les
caractéristiques : 
\begin{itemize}
	\item arrivées poissonniennes
	\item le temps entre deux arrivées suit une loi exponentielle ($1 - e^{-\lambda t}$)
	\item services poissonniens
	\item le temps de service suit une loi exponentielle ($1 - e^{- \mu t}$)
	\item un seul serveur
	\item traitement FIFO
\end{itemize}

Le débit moyen est $\lambda$, le temps de service moyen est $\inve{\mu}$ et la capacité de
traitement est $\mu$. D'après la formule de Little, le système est stable si $\lambda < \mu$.
On cherche maintenant à calculer le nombre moyen de clients.

Pour ce faire, notons $P_i$ la probabilité d'avoir $i$ clients dans le système. Soit la chronologie
suivante : 
\begin{enumerate}
	\item J'ai 0 client
	\item 1 client arrive avec un débit de $\lambda$
	\item 1 client part avec un débit $\mu$
\end{enumerate}

%fig 13

D'un point de vue stationnaire on a : \begin{itemize}
	\item le nombre de clients qui arrivent est $\lambda P_0$
	\item le nombre de clients qui partent est $\mu P_1$
\end{itemize}

Généralisons à $n$ clients :
%fig 14

L'état stationnaire du noeud $k$ nous donne (par conservation du débit), l'égalité suivante :$$
\lambda P_k + \mu P_k = \lambda P_{k-1} + \mu P_{k+1} $$

Sur l'ensemble du système on peut écrire : $$
\left \lbrace \begin{array}{rcl}
	\lambda P_0 &=& \mu P_1 \\
	\vdots & = & \vdots \\
	\lambda P_{k-1} + \mu P_{k-1} & = & \lambda P_{k-2} + \mu P {k}\\
	\lambda P_{k} + \mu P_{k} & = & \lambda P_{k-1} + \mu P_{k+1}\\
	\vdots & = & \vdots \\
	\mu P_{n} & = & \lambda P_{n-1}\\
\end{array} \right .
$$

Par l'énoncé on a : $$
P_n = \left ( \frac{\lambda}{\mu} \right )^n . P_0 $$

\textbf{Preuve :}
Procédons par récurrence.  De par la première ligne, on a : $$
P_1 = \frac{\lambda}{\mu} P_0 $$
Et de par la seconde ligne on a : $$
\begin{array}{rrcl}
	&\lambda P_1 + \mu P_1 &=& \lambda P_0 + \mu P_2 \\
	\Rightarrow & \frac{\lambda^2}{\mu} P_1 + \lambda P_0 - \lambda P_0 & = & \mu P_2 \\
	\Rightarrow & P_2 & = & \displaystyle \left ( \frac{\lambda}{\mu} \right )^2
\end{array} $$

Supposons la propriété vraie au rang $k-1$ et au rang $k$ et intéressons nous au rang $k+1$,
$\forall k \in [1;n-1]$. : $$
\begin{array}{rrcl}
	&\lambda P_k + \mu P_k &=& \lambda P_{k-1} + \mu P_{k+1} \\
	\Rightarrow & \displaystyle P_0 \left ( \frac{\lambda}{\mu} \right )^k \left [ \lambda + \mu -
\lambda \left ( \frac{\mu}{\lambda} \right ) \right ] &=& \mu P_{k+1} \\
	\Rightarrow & \displaystyle P_0 \lambda \left( \frac{\lambda}{\mu} \right )^k &=& \mu P_{k+1} \\
	\Rightarrow & P_{k+1} &=& \displaystyle\left (\frac{\lambda}{\mu} \right )^{k+1}
\end{array} $$


On définit de plus la valeur : $$
\rho = \frac{\lambda}{\mu} $$

Par définition des probabilités, on a : $$
\begin{array}{rlcl}
	&\displaystyle \sum_{k = 1}^n P_k & = & 1\\
	\Longrightarrow & P_0 \displaystyle \sum_{k = 1}^n \rho^k & = & 1 \\
	\Longrightarrow & P_0 . \displaystyle \inve{1 - \rho} & = & 1 \\
\end{array} $$

De la même manière on peut écrire $P_n$ sous la forme suivante : $$
P_n = (1-\rho) . \rho $$

On peut alors calculer le nombre moyen de clients : $$
\begin{array}{rcl}
	E(N) & = & \displaystyle \sum_{i=0}^{n-1} i P_i \\
			 	 & = & \displaystyle \sum_{i=0}^{n-1} i (1 - \rho) \rho^i \\
				 & = & \displaystyle \frac{\rho}{1 - \rho} \\
\end{array}$$

Par la formule de Little, on a : $$
\bar r = \frac{\bar N}{\lambda} = \frac{\rho}{\lambda (1 - \rho)} = \inve{\mu - \lambda} $$
\end{document}
