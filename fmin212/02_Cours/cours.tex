\documentclass[a4paper, 11pt]{article}

\usepackage[utf8x]{inputenc}
\usepackage[T1]{fontenc}
\usepackage[french]{babel}
\usepackage{graphicx}
\usepackage{moreverb}
\usepackage{subfig}
\usepackage{pdfpages}
\usepackage{eurosym}
\usepackage{fullpage}
\usepackage[fleqn]{amsmath}
\usepackage{bbm}
\usepackage{amssymb}
\usepackage{tikz}

\begin{document}

\section{Généraux Byzantins}

Deux hypothèses :
\begin{itemize}
	\item \underline{$IC_1$} : Si les lieutenants sont loyaux, ils obéïssent aux mêmes ordres.
	\item \underline{$IC_2$} : Si le général est loyal, alors chaque lieutenant loyal obéït à l'ordre
		qu'il a reçu.
\end{itemize}

%fig1 et fig2
%Théorème

Pour pouvoir repérer les félons, il faut que le nombre de sites soit supérieur à $3m$, $m$ étant le
nombre de félons.

\subsection{Oral Message}



\end{document}
