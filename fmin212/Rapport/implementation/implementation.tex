% !TEX encoding = UTF-8 Unicode
% !TEX root = ../rapport.tex

\chapter{Implémentation}\label{implementation}

\section{Stratégie et mise en \oe uvre}

L'implémentation des sites participant à l'algorithme a été réalisée de manière à obtenir une
architecture réellement distribuée pouvant s'exécuter sur des machines distinctes.
Le choix du langage, quant à lui, s'est tout naturellement porté sur le C\footnote{Afin d'éviter
	toute forme de digression déplacée dans ce rapport, la justification sera laissée au bon sens du
lecteur}. 
 Cette
implémentation permet une plus grande modularité pour les phases de test, ainsi que la soumission de
l'algorithme à des conditions physiques réelles lors de son exécution.

Le développement a été réalisée sous Vim et sous XCode et a été versionné à l'aide de git.

\begin{figure}[h]
	\begin{center}
		\begin{tikzpicture}[scale=1.7]
			\tikzset{debfin/.style={ellipse, draw, text=red}, aff/.style={rectangle, draw,
				fill=yellow!50}, test/.style={diamond, aspect=2.5, thick, draw=blue, fill=yellow!50,
				text=blue}, act/.style={rectangle, draw, rounded corners=4pt, fill=blue!25},
			fl/.style={->,>=latex, thick, rounded corners=4pt}}

			\node[debfin] (beg) at (1,0) {Debut};
			\node[debfin] (end) at (-4, 0) {Fin};
			\node[act] (init) at (1, -1) {Initialisation};
			\node[test] (sc) at (1, -2.5) {Demande SC?};
			\node[act] (csrq) at (4, -2.5) {critSecRequest()};
			\node[test] (mes) at (-2, -2.5) {Message ?};
			\node[test] (quit) at (-4, -2.5) {Quitte ?};
			\node[circle, fill=black] (inter) at (-4, -7.75) {$\bullet$};
			\node[act] (ayalive) at (-2, -5.5) {ARE\_YOU\_ALIVE};
			\node[act] (iaalive) at (-2, -4) {Envoi I\_AM\_ALIVE};
			\node[act] (conn) at (-2, -7) {CONNECT};
			\node[aff] (next) at (0, -7) {Next$ = -1$};
			\node[act] (sq) at (-2, -8.5) {SEARCH\_QUEUE};
			\node[test] (pos1) at (1, -8.5) {Position$ >0$?};
			\node[act] (ack1) at (4, -8.5) {Envoi ACK\_SQ};
			\node[act] (sp) at (-2, -10) {SEARCH\_PREV};
			\node[test] (pos2) at (1, -10) {Meilleure pos?};
			\node[act] (ack2) at (4, -10) {Envoi ACK\_SP};

			\draw[fl] (beg) -- (init);
			\draw[fl] (init) -- (sc);
			\draw[fl] (sc) --node[above] {N} (mes);
			\draw[fl] (sc) --node[above] {O} (csrq);
			\draw[fl] (mes) --node[above] {N} (quit);
			\draw[fl] (quit) -- (-4, -1.5) --node[above left] {N} (1, -1.5) |- (sc.north);
			\draw[thick, rounded corners=4pt] (mes) |-node[above left] {Y} (-4, -3.5) -- (inter);
			\draw[fl] (inter) -- (ayalive);
			\draw[fl] (inter) -- (conn);
			\draw[fl] (inter) -- (sp);
			\draw[fl] (inter) -- (sq);
			\draw[fl] (ayalive) -- (iaalive);
			\draw[fl] (conn) -- (next);
			\draw[fl] (sq) -- (pos1);
			\draw[fl] (sp) -- (pos2);
			\draw[fl] (quit) --node[above left] {O} (end);

			\draw[fl] (iaalive.east) --(1, -4) -- (sc.south);
			\draw[fl] (next.east) --(1, -7) -- (sc.south);
			\draw[fl] (pos1.north) --node[left] {N} (sc.south);
			\draw[fl] (pos1.east) --node[above] {O}  (ack1);
			\draw[fl] (pos2.east) --node[above] {O} (ack2);
			\draw[fl] (ack1.east) -- (5, -8.5) -- (5, -4) -- (1, -4) -- (sc.south);
			\draw[fl] (ack2.east) -- (5, -10) -- (5, -4) -- (1, -4) -- (sc.south);
			\draw[fl] (pos2.south) -- (1, -11) --node[below] {N} (5, -11) -- (5, -4) -- (1, -4) -- (sc.south);
			\draw[fl] (csrq.north) -- (4, -1.5) -- (1, -1.5) -- (sc.north);
		\end{tikzpicture}
	\end{center}
	\caption{Procédure principale}
\end{figure}

\begin{figure}
	\begin{center}
		\begin{tikzpicture}[scale=1.7]
			\tikzset{debfin/.style={ellipse, draw, text=red}, aff/.style={rectangle, draw,
				fill=yellow!50}, test/.style={diamond, aspect=2.5, thick, draw=blue, fill=yellow!50,
				text=blue}, act/.style={rectangle, draw, rounded corners=4pt, fill=blue!25},
			fl/.style={->,>=latex, thick, rounded corners=4pt}}

			\node[debfin] (begin) at (0,0) {Début};
			\node[act] (sr) at (0, -1) {Demande Connection};
			\node[test] (ci) at (0, -2.5) {Commit reçu?};
			\node[act] (surv) at ( 0, -4) {Surveille pred};
			\node[test] (alive) at (0, -5.5) {Pred en vie?};
			\node[test] (token) at (0, -7) {Token?};
			\node[act] (tkcs) at (0, -8.5) {Prendre SC};
			\node[act] (m3) at (3, -2.5) {Cas M3};
			\node[act] (apred) at (3, -5.5) {Questionne pred};
			\node[test] (kpred) at (6, -5.5) {+ k pannes?};
			\node[act] (m2) at (6, -4) {M2};
			\node[act] (m1) at (6, -7) {M1};
			\node[debfin] (end) at (0, -9.5) {Fin};

			\draw[fl] (begin) -- (sr);
			\draw[fl] (sr) -- (ci);
			\draw[fl] (ci) --node[above] {N} (m3);
			\draw[fl] (ci) --node[left] {O} (surv);
			\draw[fl] (surv) -- (alive);
			\draw[fl] (alive) --node[above] {N} (apred);
			\draw[fl] (alive) --node[left] {O} (token);
			\draw[fl] (token) -- (-1.5, -7) --node[left] {N} (-1.5, -4) -- (surv);
			\draw[fl] (token) --node[left] {O} (tkcs);
			\draw[fl] (tkcs) -- (end);
			\draw[fl] (apred) -- (kpred);
			\draw[fl] (kpred) --node[right] {O} (m2);
			\draw[fl] (kpred) --node[right] {N} (m1);

		\end{tikzpicture}
	\end{center}
	\caption{Procédure critSeqRequest}
\end{figure}

\begin{figure}
	\begin{center}
		\begin{tikzpicture}[scale=1.7]
			\tikzset{debfin/.style={ellipse, draw, text=red}, aff/.style={rectangle, draw,
				fill=yellow!50}, test/.style={diamond, aspect=2.5, thick, draw=blue, fill=yellow!50,
				text=blue}, act/.style={rectangle, draw, rounded corners=4pt, fill=blue!25},
			fl/.style={->,>=latex, thick, rounded corners=4pt}}

			\node[debfin] (begin) at (0,0) {Début};
			\node[act] (conn) at (0, -1) {Demande Connexion};
			\node[aff] (lasr) at (0, -2) {Last$ = \max(pos)$};
			\node[act] (con) at (0, -3) {critSecRequest};
			\node[debfin] (end) at (0, -4) {Fin};

			\draw[fl] (begin) -- (conn);
			\draw[fl] (conn) -- (lasr);
			\draw[fl] (lasr) -- (con);
			\draw[fl] (con) -- (end);
		\end{tikzpicture}
	\end{center}
	\caption{Procédure M1}
\end{figure}

\begin{figure}
	\begin{center}
		\begin{tikzpicture}[scale=1.7]
			\tikzset{debfin/.style={ellipse, draw, text=red}, aff/.style={rectangle, draw,
				fill=yellow!50}, test/.style={diamond, aspect=2.5, thick, draw=blue, fill=yellow!50,
				text=blue}, act/.style={rectangle, draw, rounded corners=4pt, fill=blue!25},
			fl/.style={->,>=latex, thick, rounded corners=4pt}}

			\node[debfin] (begin) at (0, 0) {Début};
			\node[act] (sp) at (0, -1) {Broadcast SEARCH\_PREV};
			\node[act] (wait) at (0, -2) {Attends réponse};
			\node[test] (viv) at (0, -3.5) {Réponses?};
			\node[act] (regen) at (3, -3.5) {Regénération TOKEN};
			\node[act] (tksc) at (6, -3.5) {Prendre SC};
			\node[act] (conn) at (0, -5) {Demande Connexion};
			\node[aff] (last) at (0, -6) {Last$ = \max(pos)$};
			\node[act] (con) at (0, -7) {critSecRequest};
			\node[debfin] (end1) at (0, -8) {Fin};

			\draw[fl] (begin) -- (sp);
			\draw[fl] (sp) -- (wait);
			\draw[fl] (wait) -- (viv);
			\draw[fl] (viv) --node[above] {N} (regen);
			\draw[fl] (viv) --node[right] {O} (conn);
			\draw[fl] (regen) -- (tksc);
			\draw[fl] (tksc) |- (end1);
			\draw[fl] (conn) -- (last);
			\draw[fl] (last) -- (con);
			\draw[fl] (con) -- (end1);

		\end{tikzpicture}
	\end{center}
	\caption{Procédure M2}
\end{figure}

\begin{figure}
	\begin{center}
		\begin{tikzpicture}[scale=1.7]
			\tikzset{debfin/.style={ellipse, draw, text=red}, aff/.style={rectangle, draw,
				fill=yellow!50}, test/.style={diamond, aspect=2.5, thick, draw=blue, fill=yellow!50,
				text=blue}, act/.style={rectangle, draw, rounded corners=4pt, fill=blue!25},
			fl/.style={->,>=latex, thick, rounded corners=4pt}}

			\node[debfin] (begin) at (0,0) {Début};
			\node[act] (bsq) at (0, -1) {Broadcast SEARCH\_QUEUE};
			\node[test] (mes) at (0, -3.5) {Message reçu?};
			\node[act] (regen) at (-3, -3.5) {Regénération TOKEN};
			\node[act] (tksc) at (-3, -2) {Prend SC};
			\node[act] (sq) at (3, -3.5) {SEARCH\_QUEUE};
			\node[test] (nbacc) at (3, -5) {nb\_acc sup?};
			\node[aff] (last) at (3, -6.5) {Last$ = $emetteur};
			\node[act] (con) at (3, -7.5) {critSecRequest};
			\node[act] (ack) at (0, -5) {ACK\_SEARCH\_QUEUE};
			\node[test] (next) at (0, -6.5) {Next?};
			\node[act] (conn) at (-3, -6.5) {Envoi une requete};
			\node[act] (aconn) at (0, -8) {Demande Connexion};
			\node[aff] (flast) at (0, -9) {Last$ = $emetteur};
			\node[act] (dd) at (0, -10) {critSecRequest};
			\node[debfin] (end) at (0, -11) {Fin};

			\draw[fl] (begin) -- (bsq);
			\draw[fl] (bsq) -- (mes);
			\draw[fl] (mes) --node[above] {Aucun} (regen);
			\draw[fl] (mes) --node[above] {Search Q} (sq);
			\draw[fl] (mes) --node[right] {ACK} (ack);
			\draw[fl] (regen) -- (tksc);
			\draw[fl] (tksc.west) -- (-5,  -2) |- (end);
			\draw[fl] (sq) -- (nbacc);
			\draw[fl] (nbacc.east) -- (5, -5) --node[right] {N} (5, -2) -- (0, -2) -- (mes);
			\draw[fl] (nbacc) --node[right] {O} (last);
			\draw[fl] (last) -- (con);
			\draw[fl] (con) |- (end);
			\draw[fl] (ack) -- (next);
			\draw[fl] (next) --node[right] {O} (aconn);
			\draw[fl] (next) --node[above] {N} (conn);
			\draw[fl] (aconn) -- (flast);
			\draw[fl] (flast) -- (dd);
			\draw[fl] (dd) -- (end);



		\end{tikzpicture}
	\end{center}
	\caption{Procédure M3}
\end{figure}

\subsection{Phase d'initialisation}
Une phase d'initialisation est nécessaire à l'algorithme de Naimi-Trehel. Lorsqu'un site se connecte, il broadcast un message de type HELLO. Pour garantir l'unicité du jeton, nous avons différencié deux cas :
\begin{itemize}
\item le site n'obtient aucune réponse à son message : il génère donc le jeton.
\item le site obtient au moins une réponse : une élection est faite et le site élu gagne le jeton. Il devient le père des autres sites dans l'arbre logique.
\end{itemize}


\subsection{Naimi-Trehel}
Afin d'implémenter l'algorithme, nous avons utilisé deux sockets pour chaque site : une pour la lecture et une pour l'écriture. La boucle d'exécution est en attente passive (select). Enfin, les différents mécanismes ont été gérés grâces aux protocoles réseau standards.

\subsection{Extension}
Le mécanisme de surveillance du prédécesseur lors de la demande de section critique a été réalisée à l'aide d'un thread, permettant le traitement parallèle des messages provenant des autres sites. Cette architecture nous a forcé à utiliser des signaux lors de la détection d'une panne permettant de mettre en suspens toute autre activité pour se concentrer sur la résolution du problème. 

Nous avons aussi redéfini la structure de données transitant sur le réseau de façon à pouvoir changer les adresses des prédécesseurs ainsi que d'autres informations tels que le nombre d'accès à la section critique (utile pour l'élection), ou la présence ou non d'un next dans la  file d'attente.


\subsection{Application pratique}
Nous avons décidé d'utiliser notre structure distribuée afin de résoudre un problème pratique : une approximation au problème de {\em Partition}.
Pour cela nous avons mis en place un algorithme naïf 2-approché (qui consiste à parcourir les objets et à les mettre dans le sous-ensemble le plus léger).
De plus afin de permettre la résolution distribuée, nous avons dû rendre cet algorithme non déterministe, ainsi l'ordre des objets est aléatoire pour chaque exécution.
Chaque site va donc résoudre le problème avec un ordre donné dans un thread séparé, puis une fois la solution trouvée, va tenter d'accéder à la section critique (qui est représentée par un nouveau site {\em ressource}) afin d'y déposer sa solution.
Une fois cette étape effectuée, le site génère un nouvel ordre sur les objets et recommence.

La ressource garde uniquement la meilleure solution trouvée, ainsi au bout d'un laps de temps donné, il est possible de récupérer celle-ci.


\section{Problèmes rencontrés}

Le premier problème s'est posé, peu après l'implémentation de l'algorithme de Naimi-Trehel, pour
lequel des chaînes de caractères avaient été utilisées dans les échanges de message. L'utilisation
de ces dernières étant plus sujette à débat lors de l'ajout de l'extension de tolérance aux pannes,
nous avons décidé l'utilisation d'une structure de données spécifique aux messages permettant une
gestion plus claire des paramètres à envoyer. L'adaptation de la sous-couche réseau à cette
structure particulière s'est faite non sans peine en grande partie pour des raisons d'espace
d'adressage.

Un autre problème s'est présenté, lors des phases de debug à cause du caractère distribué de
l'application, ce dernier ne permettant pas l'utilisation d'outils classiques tels que valgrind et
gdb. Il a donc été difficile de pouvoir surveiller l'évolution dans le temps des variables d'états
de chacun des sites. Dans ce genre de situation, il est facile de se rendre compte que quelque chose
ne fonctionne pas, mais beaucoup plus de comprendre le problème en lui-même et les causes de ce
dernier. 
