% !TEX encoding = UTF-8 Unicode
% !TEX root = ../rapport.tex

\chapter{Implémentation et tests}\label{implementation}

\section{Stratégie et mise en \oe uvre}

L'implémentation des sites participant à l'algorithme a été réalisée en C, afin d'obtenir un
architecture réellement distribuée pouvant s'exécuter sur des machines distinctes. Cette
implémentation permet une plus grande modularité pour les phases de test, ainsi que la soumission de
l'algorithme à des conditions physiques réelles lors de son exécution.

Le choix du langage, quant à lui, s'est tout naturellement porté sur le C\footnote{Afin d'éviter
	toute forme de digression déplacée dans ce rapport, la justification sera laissée au bon sens du
lecteur}. 

Une grande partie du développement a été réalisée sous Vim et sous XCode.

\section{Phase d'initialisation}

Je t'en laisse le soin

\section{Problèmes rencontrés}

Le premier problème s'est posé, peu après l'implémentation de l'algorithme de Naimi-Trehel, pour
lequel des chaînes de caractères avaient été utilisées dans les échanges de message. L'utilisation
de ces dernières étant plus sujette à débat lors de l'ajout de l'extension de tolérance aux pannes,
nous avons décidé l'utilisation d'une structure de données spécifique aux messages permettant une
gestion plus claire des paramètres à envoyer. L'adaptation de la sous-couche réseau à cette
structure particulière s'est faite non sans peine en grande partie pour des raisons d'espace
d'adressage.

Un autre problème s'est présenté, lors des phases de debug à cause du caractère distribué de
l'application, ce dernier ne permettant pas l'utilisation d'outils classiques tels que valgrind et
gdb. Il a donc été difficile de pouvoir surveiller l'évolution dans le temps des variables d'états
de chacun des sites. Dansce genre de situation, il est facile de se rendre compte que quelque chose
ne fonctionne pas, mais beaucoup plus de comprendre le problème en lui-même et les causes de ce
dernier. 
